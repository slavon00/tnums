Implementace matematických operací je různě složitá. Zatímco aditivní operace jsou celkem jednoduché, multiplikativní jsou řádově složitější, tak mocninné dokončíme až na konci následující kapitoly.

\subsection{Aditivní operace}
Operace \texttt{+} bere přirozený počet argumentů. Pro žádný vrací nulu (neutrální prvek aditivní grupy \cite{RachALG1}), pro jeden vrací tento a pro více pak jejich součet (je třeba doprogramovat součet). Operace \texttt{-} potom vyžaduje alespoň jeden argument, v případě zadání pouze tohoto se vrací tnum k němu opačný, v případě více pak jejich rozdíl (je třeba doprogramovat rozdíl).

\begin{theorem}[O součtu tnumů]
Nechť $\tnum{x_0}\in\Tnum{x_0}, \tnum{x_1}\in\Tnum{x_1}, \ldots, \tnum{x_n}\in\Tnum{x_n}$ a funkce $\tnum{}(\tnum{x_0},\tnum{x_1},\ldots,\tnum{x_n})$ má předpis
\begin{equation}
\tnum{}(\tnum{x_0},\tnum{x_1},\ldots,\tnum{x_n})(\varepsilon)=\sum_{i=0}^n\tnum{x_i}\left(\frac{\varepsilon}{n+1}\right),
\end{equation}
pak $\tnum{}(\tnum{x_0},\tnum{x_1},\ldots,\tnum{x_n})\in\Tnum{x_0+x_1+\ldots{+}x_n}$.

\begin{proof}
Předpokládejme $\tnum{x_i}(\varepsilon) + \varepsilon \geq x_i$ a $\tnum{x_i}(\varepsilon) - \varepsilon \leq x_i$ pro $i \in \{0, 1, \ldots , n\}$ a ukažme $\sum_{i=0}^n\tnum{x_i}\left(\frac{\varepsilon}{n+1}\right) + \varepsilon \geq \sum_{i=0}^nx_i$ a $\sum_{i=0}^n\tnum{x_i}\left(\frac{\varepsilon}{n+1}\right) - \varepsilon \leq \sum_{i=0}^nx_i$.

Z definice tnumu předpokládáme
\begin{equation}
\begin{aligned}
\tnum{x_0}\left(\frac{\varepsilon}{n+1}\right)+\frac{\varepsilon}{n+1}\geq x_0 &\land \tnum{x_0}\left(\frac{\varepsilon}{n+1}\right)-\frac{\varepsilon}{n+1}\leq x_0,\\
&\ldots \\
\tnum{x_n}\left(\frac{\varepsilon}{n+1}\right)+\frac{\varepsilon}{n+1}\geq x_n &\land \tnum{x_n}\left(\frac{\varepsilon}{n+1}\right)-\frac{\varepsilon}{n+1}\leq x_n.\\
\end{aligned}
\end{equation}
Sečtěme teď všechny výrazy a získáváme
\begin{equation}
\sum_{i=0}^n\tnum{x_i}\left(\frac{\varepsilon}{n+1}\right) + \sum_{i=0}^n\frac{\varepsilon}{n+1} \geq \sum_{i=0}^nx_i \land \sum_{i=0}^n\tnum{x_i}\left(\frac{\varepsilon}{n+1}\right) - \sum_{i=0}^n\frac{\varepsilon}{n+1} \leq \sum_{i=0}^nx_i,
\end{equation}
dále $\sum_{i=0}^n\frac{\varepsilon}{n+1} = \varepsilon$, takže
\begin{equation}
\sum_{i=0}^n\tnum{x_i}\left(\frac{\varepsilon}{n+1}\right) + \varepsilon \geq \sum_{i=0}^nx_i \land \sum_{i=0}^n\tnum{x_i}\left(\frac{\varepsilon}{n+1}\right) - \varepsilon \leq \sum_{i=0}^nx_i,
\end{equation}
což jsme chtěli ukázat.
\end{proof}
\end{theorem}

\begin{lispcode}{\texttt{tnum+}}{Funkce na součet tnumů}
(\textcolor{funkcionalni}{defun} \textcolor{pojmenovan}{tnum+} (&rest tnums)
  (\textcolor{funkcionalni}{if} (\textcolor{funkcionalni}{null} tnums)
      (\textcolor{moje}{num-to-tnum} 0)
    (\textcolor{funkcionalni}{lambda} (eps)
      (\textcolor{vedlejsi}{let} ((new-eps (\textcolor{matematicke}{/} eps (\textcolor{funkcionalni}{list-length} tnums))))
        (\textcolor{funkcionalni}{apply} \textquotesingle\textcolor{moje}{+} 
               (\textcolor{funkcionalni}{mapcar} (\textcolor{funkcionalni}{lambda} (tnum) 
                         (\textcolor{moje}{tnum-to-num} tnum new-eps))
                       tnums))))))
\end{lispcode}

\begin{fact}[Rozdíl tnumů]
Nechť $\tnum{x_{i}}\in\Tnum{x_{i}}, x_i\in\mathbb{R}$ pro $i\in\{-1,0,\ldots,n\}$ a funkce $\tnum{}(\tnum{x_{-1}},\tnum{x_0},\ldots,\tnum{x_n})$ má předpis
\begin{equation}
\tnum{}(\tnum{x_{-1}},\tnum{x_0},\ldots,\tnum{x_n})(\varepsilon)=\tnum{x_{-1}-\sum_{i=0}^nx_n}(\varepsilon)=\tnum{x_{-1}+\left(-\sum_{i=0}^nx_n\right)}(\varepsilon),
\end{equation}
pak $\tnum{}(\tnum{x_{-1}},\tnum{x_0},\ldots,\tnum{x_n})\in\Tnum{x_{-1}-x_0-\ldots -x_n}$.
\end{fact}

\begin{lispcode}{\texttt{tnum-}}{Funkce pro opačný tnum a rozdíl tnumů}
(\textcolor{funkcionalni}{defun} \textcolor{pojmenovan}{tnum-} (tnum1 &rest tnums)
  (\textcolor{funkcionalni}{if} (\textcolor{funkcionalni}{null} tnums)
      (\textcolor{moje}{-tnum} tnum1)
    (\textcolor{moje}{tnum+} tnum1 (\textcolor{moje}{-tnum} (\textcolor{funkcionalni}{apply} \textquotesingle\textcolor{moje}{tnum+} tnums)))))
\end{lispcode}

\subsection{Multiplikativní operace}
Jako první multiplikativní operaci představím převrácení hodnoty tnumu. Je to unární operace jako Opačný tnum, jen jde o jinou inverzi. Protože převrácení pracuje jen s nenulovými čísly, přidáme ještě aparát pro tnumy nenulových čísel.

\begin{definition}[Nenulový tnum]
Tnum $\tnum{}$, který nikdy nenabývá nulové hodnoty, neboli $(\forall \varepsilon \in (0,1))(\tnum{}(\varepsilon) \neq 0)$ budeme nazývat \textit{nenulový tnum} a budeme jej značit $\tnum{}_\emptyset$.
\end{definition}

Nenulová čísla tedy budeme moci reprezentovat nenulovými tnumy.

\begin{definition}[Bezpečné epsilon]\label{def:bezpecne_epsilon}
K libovolnému $\varepsilon\in (0,1)$ a tnumu $\tnum{x}\in\Tnum{x}, x\neq 0$ uvažujeme číslo $\varepsilon_\emptyset(\tnum{x},\varepsilon)$ tak, že
\begin{enumerate}
\item{$0<\varepsilon_\emptyset(\tnum{x},\varepsilon)\leq\varepsilon$},
\item{$\tnum{x}(\varepsilon_\emptyset(\tnum{x},\varepsilon)) \neq 0$} a
\item{$|\tnum{x}(\varepsilon_\emptyset(\tnum{x},\varepsilon))| > \varepsilon_\emptyset(\tnum{x},\varepsilon)$}
\end{enumerate}
a nazýváme jej \textit{bezpečným epsilonem}.
\end{definition}

\begin{lemma}[O nenulovém tnumu nenulového čísla]\label{vet:nenul}
Tnum nenulového čísla lze vyčíslit nenulově, čili $(\forall x \in(\mathbb{R}\setminus\{0\}))((\exists \tnum{x}) \rightarrow (\exists \tnum{x}_\emptyset))$. 
\begin{proof}
Vezměme za $\tnum{x}_\emptyset$ funkci $\tnum{}(\tnum{x})$, s předpisem $\tnum{}(\tnum{x})(\varepsilon) = \tnum{x}(\varepsilon_\emptyset(\tnum{x},\varepsilon))$. Pak díky podmínce $1$ v definici \ref{def:bezpecne_epsilon} vyčíslení proběhne v pořádku a díky bodu $2$ ve stejné definici bude vyčíslení nenulové, díky čemuž se jedná o nenulový tnum.
\end{proof}
\end{lemma}

Bezpečných epsilonů je nekonečně mnoho, stačí nám najít jediné. Funkce pro jeho výpočet potřebuje tnum a epsilon. To je ve shodě se zavedeným symbolem $\varepsilon_\emptyset(\tnum{x}, \varepsilon)$. Dále musí kvůli kontrole nenulovosti vypočítat i num zadaného tnumu a musí také vracet nové epsilon. Aby se tnum nevyčísloval vícekrát, když už jeho hodnotu známe, vrací funkce i tento num.

\begin{lispcode}{\texttt{get-nonzero-num+eps}}{Funkce pro nalezení bezpečného epsilonu a numu nenulového tnumu}
(\textcolor{funkcionalni}{defun} \textcolor{pojmenovan}{get-nonzero-num+eps} (tnum eps)
  (\textcolor{vedlejsi}{let} ((num (\textcolor{moje}{tnum-to-num} tnum eps)))
    (\textcolor{funkcionalni}{if} (\textcolor{funkcionalni}{or} (\textcolor{funkcionalni}{zerop} num) (\textcolor{matematicke}{<=} (\textcolor{matematicke}{abs} num) eps))
        (\textcolor{moje}{get-nonzero-num+eps} tnum (\textcolor{matematicke}{/} eps 10))
      (\textcolor{matematicke}{values} num eps))))
\end{lispcode}

\begin{theorem}[O převráceném tnumu]\label{hyp:prevraceni_tnumu}
Nechť $\tnum{x}\in\Tnum{x},x\neq 0$ a funkce $\tnum{}(\tnum{x})$ má předpis
\begin{equation}
\tnum{}(\tnum{x})(\varepsilon)=/\left[\tnum{x}(\varepsilon_\emptyset(\tnum{x}, (\varepsilon*|\tnum{x}(\varepsilon_\emptyset(\tnum{x}, \varepsilon))|*(|\tnum{x}(\varepsilon_\emptyset(\tnum{x}, \varepsilon))|-\varepsilon_\emptyset(\tnum{x}, \varepsilon)))))\right],
\end{equation}
pak $\tnum{}(\tnum{x})\in\Tnum{/x}$.
\begin{proof}

Podle rovnice \eqref{rov:def:tnum} platí 
\begin{equation}
\left|\txe-x\right|\leq \varepsilon,
\end{equation}
což lze díky lemmatu \ref{vet:nenul} a předpokladu nenulovosti $x$ přepsat na
\begin{equation}
\left|\tnum{x}(\varepsilon_\emptyset(\tnum{x}, \varepsilon))-x\right|\leq \varepsilon,
\end{equation}
díky absolutní hodnotě pak platí
\begin{equation}
\left| x-\tnum{x}(\varepsilon_\emptyset(\tnum{x}, \varepsilon))\right|\leq \varepsilon.
\end{equation}
Nerovnici vydělíme kladným číslem $|\tnum{x}(\varepsilon_\emptyset(\tnum{x}, \varepsilon))*x|$
\begin{equation}
\frac{\left| x-\tnum{x}(\varepsilon_\emptyset(\tnum{x}, \varepsilon))\right|}{|\tnum{x}(\varepsilon_\emptyset(\tnum{x}, \varepsilon))*x|}\leq \frac{\varepsilon}{|\tnum{x}(\varepsilon_\emptyset(\tnum{x}, \varepsilon))*x|}
\end{equation}
a protože $|a|*|b|=|a*b|$, po dvojí aplikaci platí
\begin{equation}
\left|\frac{x-\tnum{x}(\varepsilon_\emptyset(\tnum{x}, \varepsilon))}{\tnum{x}(\varepsilon_\emptyset(\tnum{x}, \varepsilon))*x}\right|\leq \frac{\varepsilon}{|\tnum{x}(\varepsilon_\emptyset(\tnum{x}, \varepsilon))|*|x|}
\end{equation}
a po roztržení levého výrazu na rozdílné jmenovatele dostáváme
\begin{equation}
\left|\frac{1}{\tnum{x}(\varepsilon_\emptyset(\tnum{x}, \varepsilon))}-\frac{1}{x}\right|\leq \frac{\varepsilon}{|\tnum{x}(\varepsilon_\emptyset(\tnum{x}, \varepsilon))|*|x|}.
\end{equation}
Dále díky předpokladu $3$ z definice \ref{def:bezpecne_epsilon} $|x|\geq|\tnum{x}(\varepsilon_\emptyset(\tnum{x}, \varepsilon))|-\varepsilon_\emptyset(\tnum{x}, \varepsilon)$ a proto
\begin{equation}
\left|\frac{1}{\tnum{x}(\varepsilon_\emptyset(\tnum{x}, \varepsilon))}-\frac{1}{x}\right|\leq \frac{\varepsilon}{|\tnum{x}(\varepsilon_\emptyset(\tnum{x}, \varepsilon))|*(|\tnum{x}(\varepsilon_\emptyset(\tnum{x}, \varepsilon))|-\varepsilon_\emptyset(\tnum{x}, \varepsilon))},
\end{equation}
takže po úpravě přesnosti dostáváme
\begin{equation}
\left|\frac{1}{\tnum{x}(\varepsilon_\emptyset(\tnum{x}, (\varepsilon * |\tnum{x}(\varepsilon_\emptyset(\tnum{x}, \varepsilon))|*(|\tnum{x}(\varepsilon_\emptyset(\tnum{x}, \varepsilon))|-\varepsilon_\emptyset(\tnum{x}, \varepsilon)))))}-\frac{1}{x}\right|\leq \varepsilon.
\end{equation}
\end{proof}
\end{theorem}

\begin{lispcode}{\texttt{/tnum}}{Funkce pro převrácenou hodnotu tnumu}
(\textcolor{funkcionalni}{defun} \textcolor{pojmenovan}{/tnum} (tnum)
  (\textcolor{funkcionalni}{lambda} (eps)
    (\textcolor{matematicke}{multiple-value-bind} (num eps0)
        (\textcolor{moje}{get-nonzero-num+eps} tnum eps)
      (\textcolor{vedlejsi}{let*} ((absnum (\textcolor{matematicke}{abs} num))
             (neweps (\textcolor{matematicke}{*} eps absnum (\textcolor{matematicke}{-} absnum eps0))))
        (\textcolor{matematicke}{/} (\textcolor{funkcionalni}{if} (\textcolor{matematicke}{>=} neweps eps) num
             (\textcolor{moje}{get-nonzero-num+eps} tnum neweps)))))))
\end{lispcode}

Operace \texttt{*} bere přirozený počet argumentů. Pro žádný vrací jedničku (neutrální prvek multiplikativní grupy \cite{RachALG1}), pro jeden vrací tento a pro více pak jejich součin (doprogramovat zbývá součin).

\begin{theorem}[O součinu dvou tnumů]\label{vet:soucin_dvou_tnumu}
Nechť $\tnum{x}\in\Tnum{x},x\in\mathbb{R}$ a $\tnum{y}\in\Tnum{y},y\in\mathbb{R}$ a funkce $\tnum{}(\tnum{x},\tnum{y})$ má předpis
\begin{equation}
\tnum{}(\tnum{x},\tnum{y})(\varepsilon)=\tnum{x}\left(\frac{\varepsilon}{2*\mathrm{max}\left\{(|\tnum{y}(\varepsilon)| + \varepsilon);1\right\}}\right)*\tnum{y}\left(\frac{\varepsilon}{2*\mathrm{max}\left\{(|\tnum{x}(\varepsilon)| + \varepsilon);1\right\}}\right),
\end{equation}
pak $\tnum{}(\tnum{x},\tnum{y})\in\Tnum{x*y}$.

\begin{proof}
\begin{itemize}\item{
Nejprve si dokažme nerovnici
\begin{equation}\label{ner:abst}
|x|\leq|\tnum{x}(\varepsilon)|+\varepsilon.
\end{equation}
Odečtením $|\tnum{x}(\varepsilon)|$ získáváme
\begin{equation}
|x|-|\tnum{x}(\varepsilon)|\leq\varepsilon,
\end{equation}
což bude platit, pokud dokážeme silnější tvrzení
\begin{equation}
||x|-|\tnum{x}(\varepsilon)||\leq\varepsilon.
\end{equation}
To díky trojúhelníkové nerovnosti $||a|-|b||\leq|a-b|$ lze přepsat na
\begin{equation}
|x-\tnum{x}(\varepsilon)|\leq\varepsilon,
\end{equation}
v absolutní hodnotě můžeme prohodit sčítance beze změny její hodnoty. Získaný vztah
\begin{equation}
|\tnum{x}(\varepsilon)-x|\leq\varepsilon
\end{equation}
platí přímo z definice tnumu.}

\item{Nerovnice
\begin{equation}\label{ner:abst2}
\frac{|y|}{\mathrm{max}\left\{(|\tnum{y}(\varepsilon)| + \varepsilon);1\right\}}\leq 1
\end{equation}
vyplývá z nerovnice \eqref{ner:abst}, ekvivalentně je ji totiž možné zapsat $\frac{|x|}{|\tnum{x}(\varepsilon)|+\varepsilon}\leq 1$ a maximum ve jmenovateli jen tvrzení posiluje.}

\item{Nerovnice
\begin{equation}\label{ner:abst3}
\frac{\left|\tnum{x}\left(\frac{\varepsilon}{2*\mathrm{max}\left\{(|\tnum{y}(\varepsilon)| + \varepsilon);1\right\}}\right)\right|}{\mathrm{max}\left\{(|\tnum{x}(\varepsilon)| + \varepsilon);1\right\}}\leq 1
\end{equation}
vyplývá ze skutečnosti, že $0\leq\frac{\varepsilon}{2*\mathrm{max}\{(|\tnum{y}(\varepsilon)+\varepsilon);1\}}\leq\varepsilon$, a proto
\begin{equation}
\frac{\left|\tnum{x}\left(\frac{\varepsilon}{2*\mathrm{max}\left\{(|\tnum{y}(\varepsilon)| + \varepsilon);1\right\}}\right)\right|}{\mathrm{max}\left\{(|\tnum{x}(\varepsilon)| + \varepsilon);1\right\}}\leq\frac{|\tnum{x}(\varepsilon)|+\varepsilon}{\mathrm{max}\left\{(|\tnum{x}(\varepsilon)| + \varepsilon);1\right\}}\leq 1,
\end{equation}což platí.}

\item{Nyní přejděme k důkazu věty. Aby věta platila, musí být splněna nerovnost
\begin{equation}
\left| \tnum{x}\left(\frac{\varepsilon}{2*\mathrm{max}\left\{(|\tnum{y}(\varepsilon)| + \varepsilon);1\right\}}\right)*\tnum{y}\left(\frac{\varepsilon}{2*\mathrm{max}\left\{(|\tnum{x}(\varepsilon)| + \varepsilon);1\right\}}\right) -xy \right|\leq\varepsilon.
\end{equation} V dalším kvůli přehlednosti zápisu budu označovat $\varepsilon_x := \frac{\varepsilon}{2*\mathrm{max}\left\{(|\tnum{y}(\varepsilon)| + \varepsilon);1\right\}}$ a $\varepsilon_y := \frac{\varepsilon}{2*\mathrm{max}\left\{(|\tnum{x}(\varepsilon)| + \varepsilon);1\right\}}$.

Rozepíšeme levou stranu, odečtením a přičtením členu $\tnum{x}(\varepsilon_x)*y$ dostáváme
\begin{equation}
|\tnum{x}(\varepsilon_x)*\tnum{y}(\varepsilon_y) - \tnum{x}(\varepsilon_x)*y + \tnum{x}(\varepsilon_x)*y - x*y|\leq
\end{equation}
a z trojúhelníkové nerovnosti
\begin{equation}
\leq|\tnum{x}(\varepsilon_x)*\tnum{y}(\varepsilon_y) - \tnum{x}(\varepsilon_x)y |+| \tnum{x}(\varepsilon_x)*y - x*y|\leq
\end{equation}
a po vytknutí $|\tnum{x}(\varepsilon_x)|$ z prvních dvou členů a $|y|$ z druhých dvou máme
\begin{equation}
\leq|\tnum{x}(\varepsilon_x)|*|\tnum{y}(\varepsilon_y) - y|+|y|*|\tnum{x}(\varepsilon_x) - x|\leq
\end{equation}
a po dvojím použití vztahu $|\tnum{x}(\varepsilon)-x|\leq\varepsilon$ získáváme
\begin{equation}
\begin{aligned}
\leq\frac{\varepsilon}{2}*|\tnum{x}(\varepsilon_x)|*\left|\frac{1}{\mathrm{max}\left\{(|\tnum{x}(\varepsilon)| + \varepsilon);1\right\}}\right|&+\\+\frac{\varepsilon}{2}*|y|*\left|\frac{1}{\mathrm{max}\left\{(|\tnum{y}(\varepsilon)| + \varepsilon);1\right\}}\right|&=
\end{aligned}
\end{equation}
a zjednodušíme-li zápis pomocí zlomku, dostáváme
\begin{equation}
= \frac{\varepsilon}{2}*\frac{\left|\tnum{x}(\varepsilon_x)\right|}{\mathrm{max}\left\{(|\tnum{x}(\varepsilon)| + \varepsilon);1\right\}}+\frac{\varepsilon}{2}*\frac{|y|}{\mathrm{max}\left\{(|\tnum{y}(\varepsilon)| + \varepsilon);1\right\}}\leq
\end{equation}
a díky dokázaným nerovnostem \eqref{ner:abst2} a \eqref{ner:abst3} pak platí
\begin{equation}
\leq\frac{\varepsilon}{2}+\frac{\varepsilon}{2}=\varepsilon.
\end{equation}}
\end{itemize}
\end{proof}
\end{theorem}

Právě dokázaná věta mluví o součinu dvou tnumů. Zobecnění na konečný počet tnumů by v tomto bodě šlo naprogramovat akumulací. To je ale neefektivní řešení, a proto by bylo dobré najít obecnou funkci. Následující vztah není dokázaný, vychází však z tvaru násobení pro dva tnumy.

\begin{hypothesis}[Součin tnumů]\label{vet:soucin_tnumu}
Nechť $\tnum{x_i}\in\Tnum{x_i}, x_i\in\mathbb{R}$ pro $i=0,1,\ldots,n$ a funkce $\tnum{}(\tnum{x_0},\tnum{x_1},\ldots,\tnum{x_n})$ má předpis
\begin{equation}
\tnum{}(\tnum{x_0},\tnum{x_1},\ldots,\tnum{x_n})(\varepsilon)=\prod_{i=0}^n\tnum{x_i}\left(\frac{\varepsilon}{(n+1)*\mathrm{max}\left\{\prod_{j=0, i\neq j}^n(|\tnum{x_j}(\varepsilon)|+\varepsilon);1\right\}}\right),
\end{equation}
pak $\tnum{}(\tnum{x_0},\tnum{x_1},\ldots,\tnum{x_n})\in\Tnum{\prod_{i=0}^nx_i}$.
\end{hypothesis}

Hypotéza mluví o nenulových číslech. Nesnižujeme ale obecnost, protože nula je agresivní prvek a výsledkem násobení čehokoli s nulou je nula, takže se ostatní numy ani nemusejí počítat a výsledek se může vrátit.

Samotná implementace pak využívá pomocnou mapovací funkci.

\begin{lispcode}{\texttt{create-list-for-multiplication}}{Pomocná fun\-kce pro násobení}
(\textcolor{funkcionalni}{defun} \textcolor{pojmenovan}{create-list-for-multiplication} (tnums eps)
  (\textcolor{vedlejsi}{let*} ((result nil) (len (\textcolor{funkcionalni}{list-length} tnums)) (el (\textcolor{matematicke}{/} eps len))
         (nums (\textcolor{funkcionalni}{mapcar} (\textcolor{funkcionalni}{lambda} (tnum) 
                         (\textcolor{moje}{tnum-to-num} tnum el)) tnums)))
    (\textcolor{funkcionalni}{dotimes} (i len result)
      (\textcolor{vedlejsi}{let} ((neps 1))
        (\textcolor{funkcionalni}{dotimes} (j len)
          (\textcolor{funkcionalni}{unless} (\textcolor{matematicke}{=} i j)
            (\textcolor{vedlejsi}{setf} neps (\textcolor{matematicke}{/} neps (\textcolor{matematicke}{+} (\textcolor{matematicke}{abs} (\textcolor{funkcionalni}{nth} j nums)) eps)))))
        (\textcolor{vedlejsi}{setf} result (cons
                      (\textcolor{funkcionalni}{if} (\textcolor{matematicke}{<=} neps 1) (\textcolor{funkcionalni}{nth} i nums)
                        (\textcolor{moje}{tnum-to-num} (\textcolor{funkcionalni}{nth} i tnums)
                                     (\textcolor{matematicke}{/} el (max neps 1))))
                      result))))))
\end{lispcode}

\begin{lispcode}{\texttt{tnum*}}{Funkce pro násobení tnumů}
(\textcolor{funkcionalni}{defun} \textcolor{pojmenovan}{tnum*} (&rest tnums)
  (\textcolor{funkcionalni}{if} (\textcolor{funkcionalni}{null} tnums)
      (\textcolor{moje}{num-to-tnum} 1)
    (\textcolor{funkcionalni}{lambda} (eps)
      (\textcolor{funkcionalni}{apply} \textquotesingle\textcolor{moje}{*} (\textcolor{moje}{create-list-for-multiplication} tnums eps)))))
\end{lispcode}

Operace \texttt{/} vyžaduje alespoň jeden argument, v případě zadání pouze tohoto vrací převrácený tnum, v případě více pak jejich podíl (zbývá doprogramovat podíl).

\begin{fact}[Podíl tnumů]
Nechť $\tnum{x_i}\in\Tnum{x_i},x_i\in\mathbb{R}\setminus\{0\}$ pro $i\in\{0,1,\ldots ,n\}$ a $x_{i}\in\mathbb{R}$ pro $i=-1$ a funkce $\tnum{}(\tnum{x_{-1}},\tnum{x_0},\ldots,\tnum{x_n})$ má předpis
\begin{equation}
\tnum{}(\tnum{x_{-1}},\tnum{x_0},\ldots,\tnum{x_n})=\tnum{x_{-1}/\prod_{i=0}^nx_i}=\tnum{x_{-1}*\left(/\prod_{i=0}^nx_i\right)},
\end{equation}
pak $\tnum{}(\tnum{x_{-1}},\tnum{x_0},\ldots,\tnum{x_n})\in\Tnum{x_{-1}/x_0/\ldots /x_n}$.
\end{fact}

\begin{lispcode}{\texttt{tnum/}}{Funkce pro dělení tnumů}
(\textcolor{funkcionalni}{defun} \textcolor{pojmenovan}{tnum/} (tnum1 &rest tnums)
  (\textcolor{funkcionalni}{if} (\textcolor{funkcionalni}{null} tnums)
      (\textcolor{moje}{/tnum} tnum1)
    (\textcolor{moje}{tnum*} tnum1 (\textcolor{moje}{/tnum} (\textcolor{funkcionalni}{apply} \textquotesingle\textcolor{moje}{tnum*} tnums)))))
\end{lispcode}

Pro tnumy $a, b\in\mathfrak{T}$ existuje tnum $a+b$ a $a*b$. Množina $\mathfrak{T}$ je tedy uzavřena na operace $\texttt{+}$ a $\texttt{*}$. Tnumy pak i z pohledu strukturální algebry dobře reprezentují rekurzivní čísla, ta jsou totiž číselným tělesem \cite{rice:kompr}.

\subsection{Mocninné operace}
K implementaci mocninných operací potřebujeme funkce, které přidáme až v další kapitole. Uveďme zde alespoň vztahy, podle kterých lze mocninné operace naprogramovat.

Obecná mocnina využívá přirozený logaritmus a exponenciálu
\begin{equation}\label{rov:obmoc}
a^b=e^{\mathrm{ln}(a)^b}=e^{(b*\mathrm{ln}(a))},a>0.
\end{equation}

Odmocninu lze vyjádřit pomocí převrácení hodnoty mocniny
\begin{equation}\label{rov:odmoc}
\sqrt[b]{a}=a^{(/b)},b\neq0.
\end{equation}