Číslo je matematický objekt, který na intuitivní úrovni všichni nějak chápeme. Mělo by to znamenat, že každý dokáže říci o nějakém objektu, jestli je to číslo, nebo nikoli a všichni se na tom shodneme. Je číslem $3$? Je číslem $2.999\ldots$? Je číslem $e$? Vágní pokus o definici by mohl vypadat tak jako na české wikipedii:

\begin{definition}[Číslo -- naivní \cite{wiki:cislo}]
Číslo je abstraktní entita užívaná pro vyjádření množství nebo pořadí.
\end{definition}

Uvedená definice přiřazuje číslům dvě funkce -- kardinální a ordinální. Jinak o povaze čísel neříká mnoho. Víme, že je to abstraktní entita -- to znamená, že ji nelze zapsat samu o sobě, ale pomocí nějakého symbolu. Symbol reprezentující číslo tři je $3$, $\frac{6}{2}$ nebo třeba i $2.999\ldots$. To ukazuje, že číslo není jednoznačně dáno svým zápisem, ale svým obsahem. To má společné s jinými matematickými objekty -- s množinami.

Množina je také dána svým obsahem (prvky), nikoli svým zápisem. Později budeme tuto jejich povahu nazývat principem extenzionality. Například $\{0, 1, 2, 3 \} = \{ n | n \in \mathbb{N} \land n \leq 3 \}$ je stejná množina s jinými zápisy. Je tedy něco jiného symbol s referencí na nějakou entitu a tato entita samotná. $3$ není to samé jako $2.999\ldots$, ale číslo $3$ je to samé jako číslo $2.999\ldots$. Alenka takhle zjistila rozdíl mezi tím, jak se říká názvu písně, jaké je její jméno, jak se píseň jmenuje a co píseň opravdu je \cite{TtLG}. Pokud koncept čísla a symbolu čísla čtenář začne rozlišovat, už mu nebude divné, že se čísla značí písmenem, nebo že má číslo dokonce nějaký divoký zápis, například $e = \lim_{n \to \infty} \left(1 +\frac{1}{n} \right)^n = \sum_{n\in\mathbb{N}}\frac{1}{n!}$.

\subsection{Přirozená čísla}
Přirozená čísla jsou čísla, jejichž \textit{kardinalita} určuje nějaký počet nějakých nedělitelných částí nějakého celku. Historicky asi vznikla nejdříve, proto se nazývají přirozená (anglicky \textit{natural} -- přírodní). Jedná se o čísla nula, jedna, dva atd. Symboly těchto čísel jsou 0, 1, 2, atd. Množina přirozených čísel má vlastnost, že \textit{každé} číslo $n$ má následníka. Toho značíme $s(n)$ nebo také $n+1$ \cite{LambdaCalcul}.

Hodnota přirozeného čísla je tedy počet entit v nějakém souboru. Například počet krav ve stádu, počet prstů na ruce, nebo třešniček na dortu. \textit{Následník} takového čísla značí, kolik entit bude v souboru, pokud přidáme jednu krávu, přišijeme jeden prst nebo vypěstujeme další třešničku.

Jako v matematice skoro vše, jsou i přirozená čísla \textit{množiny}. Konkrétně v~tomto textu je zavedu v Zermelově-Fraenkelově axiomatici teorii množin. Je snadné nahlédnout, že stačí nějak zkonstruovat číslo nula a \textit{zobrazení} $s$ přiřazující každému číslu následníka. Poté budeme mít celou nekonečnou množinu přirozených čísel zkonstruovanou. Protože se jedná o výklad teorie množin, celý zbytek této podkapitoly je pouze velmi stručný výtah z \cite{TeMno}.

Definice přirozených čísel je induktivní a stojí na jednoduché myšlence Johna von Neumanna, že \uv{přirozené číslo je množina všech menších přirozených čísel}. Číslo nula je zde prázdná množina značená $0$, $\emptyset$ nebo $\{ \}$. A následník čísla je sjednocení čísla s množinou toto obsahující, čili $s(n) = n \cup \{n\}$.

Několik prvních přirozených čísel tedy vypadá následovně.
\begin{example}[První přirozená čísla coby množiny]
\begin{equation}
0 = \emptyset
\end{equation}
\begin{equation}
1 = s(0) = 0 \cup \{ 0 \} = \emptyset \cup \{ \emptyset \} = \{ \emptyset \}
\end{equation}
\begin{equation}
2 = s(1)= 1 \cup \{ 1 \} = \{ \emptyset \} \cup \{ \{ \emptyset \} \} = \{ \emptyset , \{ \emptyset \} \}
\end{equation}
\begin{equation}
3 = s(2)= 2 \cup \{ 2 \} = \{ \emptyset , \{ \emptyset \} \} \cup \{ \{ \emptyset , \{ \emptyset \} \} \} = \{ \emptyset , \{ \emptyset \} , \{ \emptyset , \{ \emptyset \} \} \} 
\end{equation}
\begin{equation}
\begin{split}
4 =s(3)= 3 \cup \{ 3 \} = \{ \emptyset , \{ \emptyset \} , \{ \emptyset , \{ \emptyset \} \} \}  \cup \{ \{ \emptyset , \{ \emptyset \} , \{ \emptyset , \{ \emptyset \} \} \}  \} = \\ = \{ \emptyset , \{ \emptyset \} , \{ \emptyset , \{ \emptyset \} \} , \{ \emptyset , \{ \emptyset \} , \{ \emptyset , \{ \emptyset \} \} \}  \}
\end{split}
\end{equation}
\begin{equation}
5 =s(4)= 4 \cup \{ 4 \}
\end{equation}
\end{example}

\begin{remark}[Provázání hodnot a podmnožin]\label{pozn:kard_podm}
Všimněme si, že kardinalita v tomto pojetí znamená počet podmnožin. Neboli číslo $n$ má $n$ podmnožin, čímž se provázaly dva zdánlivě cizí pojmy a sice \textit{podmnožinovost} a \textit{hodnota} čísla. V tomto kontextu pak není překvapivá podobnost srovnávacích symbolů $\subseteq$ a $\leq$.
\end{remark}

Zermelova-Fraenkelova teorie stojí na následujících pěti axiomech a jednom axiomovém schématu (také se říká 6 axiomů) a nic jiného (kromě jazyka predikátové logiky) už nepotřebuje.
\begin{itemize}
\item Axiom extenzionality: $(\forall u)(u \in x \leftrightarrow u \in y) \rightarrow x = y$
\item Axiom fundovanosti: $ (\forall a)(a \neq \emptyset \rightarrow (\exists x)(x \in a \land x \cap a = \emptyset))$
\item Axiom sumy: $(\forall a)(\forall b)(\exists z)(\forall x)(x \in z \leftrightarrow (x = a \lor x = b))$
\item Axiom potence: $(\forall a)(\exists z)(\forall x)(x \in z \leftrightarrow x \subset a)$
\item Axiom nekonečna: $(\exists z)(\emptyset \in z \land (\forall x)(x \in z \rightarrow x \cup \{ x \} \in z))$
\item Schéma axiomů nahrazení: Je-li $\Psi (u, v)$ formule, která neobsahuje volné proměnné $w$ a $z$, potom formule $$\begin{aligned}(\forall u)(\forall v)(\forall w)((\Psi (u,v) \land \Psi (u, w)) \rightarrow v = w) \rightarrow \\ \rightarrow (\forall a)(\exists z)(\forall v)(v \in z \leftrightarrow (\exists u)(u \in a \land \Psi (u,v))) \end{aligned}$$ je axiom nahrazení.
\end{itemize}

Z těchto axiomů je možné odvodit i (slabší) tvrzení, které se někdy přijímají jako axiomy a sice
\begin{itemize}
\item Axiom dvojice: $(\forall a)(\forall b)(\exists z)(\forall x)(x \in z \leftrightarrow (x = a \lor x = b))$
\item Schéma axiomů vydělení: Je-li $\varphi (x)$ formule, která neobsahuje volnou proměnnou $z$, potom formule $(\forall a)(\exists z)(\forall x)(x \in z \leftrightarrow  (x \in a \land \varphi (x)))$ je axiom vydělení.
\end{itemize}

Z axiomů ZF je pro naše zkoumání důležitý napřílkad axiom nekonečna -- existuje alespoň jedna (nekonečná) množina. Za pomoci dalších axiomů poté konstruujeme další prvky, jako třeba prázdnou množinu -- tu dostaneme pomocí axiomu vydělení s jakoukoli množinou $a$ a formulí $\varphi(x) = x \neq x$ -- prázdná množina je tedy $\emptyset = \{ x: x \in a \land x \neq x \}$. Dále získáváme, že sjednocení množin je také množina, podobně jejich průnik. Tyto výstupy zde nebudu vyvozovat, zájemce odkáži na \cite{TeMno} a konečně přikročím k definici množiny přirozených čísel.

\begin{definition}[Induktivní množina]
Množina $A$ je \textit{induktivní}, pokud platí $(\emptyset \in A) \land (\forall a \in A)((a \cup \{ a \}) \in A)$.
\end{definition}

\begin{definition}[Přirozená čísla]
\begin{equation}
\mathbb{N} = \bigcap \{A : A \text{~je induktivní} \}
\end{equation}
\end{definition}

Množina přirozených čísel je nejmenší induktivní množina a je podmnožinou každé induktivní množiny. Zajímavý je i důsledek, co znamená, že třída všech přirozených čísel (množin předchůdců) je množina -- $\mathbb{N}$ je nejmenší nekonečný \textit{kardinál} a také jediný \textit{spočetný} kardinál. Rozsah této práce ale nedovoluje se tomuto fenoménu věnovat více.

\begin{remark}[Značení přirozených čísel]
Množina $z$ v axiomu nekonečna je stejná množina, jako $\mathbb{N}$. V teorii množin se značí $\omega$. Při zkoumání kardinality je pak $\aleph_0$. Pokud budu operovat s množinou přirozených čísel bez nuly, připíšu jako horní index $+$ ($\mathbb{N^+} = \mathbb{N} - \{0\}$). V~literatuře, kde se nula do přirozených čísel nezahrnuje naše přirozená čísla značí s nulou jako dolním indexem ($\mathbb{N}_0 = \mathbb{N} \cup \{0 \}$).
\end{remark}

\begin{remark}[Nula je přirozená]
Z definice přirozených čísel podle ZF přímo vyplývá, že součástí přirozených čísel je i číslo nula. Intuitivně i stádo, ve kterém není žádná kráva je stále stádem. Ruka bez prstů je stále rukou, byť s nula prsty. Dort bez třešniček je sice smutný, ale i tak dort.

Jestli je nula přirozená nechám spíše matematickým filosofům a v této práci, pokud neřeknu jinak, budu počítat s nulou jako přirozeným číslem. Už nikdy by zde čtenář tedy neměl vidět symbol $\mathbb{N}_0$.
\end{remark}

\subsection{Vyšší obory čísel}
\label{kap:vyssi_obory_cisel}
V minulé podkapitole jsem rigorózně zavedl přirozená čísla. S dalšími obory už nebudu postupovat takto exaktně, u jedné množiny to -- myslím -- stačilo.

Vyšším oborem čísel jsou čísla celá, značená $\mathbb{Z}$ a jsou to všechna čísla, která mohou vzniknout libovolným odčítáním přirozených čísel. Jejich výčet je diskrétní: $\mathbb{Z} = \{0, 1, -1, 2, -2, \ldots\}$\cite{OEIS:integer}. Zde \uv{$-$} je znaménko odčítání (snížení hodnoty prvního argumentu o hodnotu druhého). Značení: místo $0-2$ se píše úžeji $-2$.

Další obor vyjadřuje poměr velikosti nějakého celku vůči jinému. Zde si opět vypomůžu obory, které už máme zadefinované a mohu představit čísla racionální, pro která se vžil zápis $\mathbb{Q}$ a takovým číslem je číslo $x$, pokud jde zapsat jako $y$/$z$, kde $y \in \mathbb{Z}$ a $z \in \left( \mathbb{Z} \setminus \{0\} \right)$ a \uv{$/$} je symbol operace dělení \cite{SPIVAK:calculus}. Taková čísla jsou například $\mathbb{Q} = \{0, 1, 1/2, -1, 1/3, -1/2, 2, 1/4, \ldots \}$, a jsou také spočetná.

Stále v našem číselném systému nemáme například délku úhlopříčky jednotkového čtverce -- takovéto číslo značíme $\sqrt{2}$. Nebo třeba poměr obvodu kružnice k jejímu průměru -- toto značíme $\pi$. Když do našeho systému doplníme všechna tato čísla, získáváme konečně tzv. číselnou osu (přímku) reprezentující čísla reálná, značená $\mathbb{R}$. Ta čísla, která jsme přidali a tudíž byla iracionální (nebyla racionální), označíme $\mathbb{I}$, $\mathbb{I} = \mathbb{R} \setminus \mathbb{Q}$ \cite{tabulky}. Ke každým dvěma reálným číslům $r$ a $\varepsilon$ lze najít racionální číslo $q$ tak, aby
\begin{equation}\label{rov:rac_u_real}
|r-q|\leq\varepsilon \text{~\cite{TeCis}}.
\end{equation}

V reálných číslech matematikové objevili ještě jemnější struktury, než jen dělení na obory, které jsem teď představil. První jmenované iracionální číslo ($\sqrt{2}$) je tzv. \textit{algebraické}, protože může vypadnout jako řešení z nějaké algebraické rovnice, např $x ^ 2 = 2$ nebo $(-x) ^ 2 = 2$, zatímco druhé jmenované -- tzv. \textit{Ludolfovo číslo} ($\pi$) takovouto vlastností nedisponuje, je \textit{nealgebraické}, též \textit{transcendentní} \cite{TeMno}. Dále v reálných číslech existují tzv. \textit{rekurzivní čísla} (\textit{computable reals} -- vizte kapitolu \ref{kap:computable-reals}), která se dají vyčíslit v konečném čase. Pro reálné číslo $r$ a dané $\varepsilon$ existuje vyčíslitelná funkce (pro konečný vstup někdy skončí), jejímž výsledkem je racionální číslo $q$ tak, že platí nerovnice \ref{rov:rac_u_real}. Ostatní čísla jsou nerekurzivní a protože Turingových strojů je spočetně mnoho, je nerekurzivních čísel nekonečně mnoho \cite{wiki:CompN}. Poznamenejme už nyní, že přechod od racionálních čísel k rekurzivním je velmi složitý a tato práce je hlavně o tomto přechodu. V jednom oboru jsou i složité věci vcelku jednoduché, zatímco v tom druhém jsou i jednoduché věci vceklu složité.

Pokud nebudeme uvažovat pouze jednu číselnou osu, ale číslo bude mít více složek, přesněji $2^n$ složek, pak připouštíme existenci ještě vyšších číselných oborů -- komplexních čísel ($\mathbb{C}, n = 1$), kvaternionů ($\mathbb{H}, n = 2$), oktonionů ($\mathbb{O}, n = 3$) -- těmto čtyřem oborům (společně s reálnými čísly) říkáme \textit{normované algebry s dělením} (\textit{normed division algebra}) \cite{DAaQT}. Ještě vyšší obory (sedeniony -- $\mathbb{S}, n = 4$ atd.) už jsou úplně mimo ambice tohoto úvodu. V této práci již nadále \uv{číslem} myslím číslo \textit{rekurzivní}. Jiným názvem této práce by tedy mohlo být \uv{Přesné výpočty s rekurzivními čísly}.

\subsection{Operace s čísly}
\label{operace_s_cisly}
Už při vymezování číselných oborů jsem zmínil 3 operace, které čísla většinou \uv{zmenšují}, bylo to \textit{odčítání} (rozdíl), \textit{dělení} (poměr) a \textit{odmocňování} (odmocnina). K nim patří ještě operace opačné, které čísla veskrze \uv{zvětšují} a ty po řadě nazýváme \textit{sčítání} (součet), \textit{násobení} (součin) a \textit{umocňování} (mocnina).

Binární operace se značí znaménkem mezi operandy. Například součet čísel $x$ a $y$ značíme $x+y$, součin $x*y$ atd. Základní značení operací je uvedeno v~tabulce \ref{tab:znamenka_operaci}. Binárním operacím, které mají jako první operand neutrální prvek budeme říkat unární. Je to číslo opačné (k číslu $x$ je opak číslo $0-x$ a značíme ho $-x$) a převrácené číslo (k číslu $x\neq0$ je jeho převrácením číslo $1/x$ a značíme ho $x^{-1}$).

Operacím $\langle +, -\rangle$ říkáme aditivní, $\langle +, /\rangle$ multiplikativní a $\langle\string^ , \surd\rangle$ mocninné. Zajímavé jsou vztahy mezi operacemi. Že se dvojice jmenují spolu naznačuje jejich příbuznost. Tyto vztahy bych nazval \uv{horizontální}. Mnohem zajímavější jsou ale vztahy \uv{vertikální}. Platí
\begin{equation}
x*n = \underbrace{x + x + \ldots + x}_{n\text{-krát}} \text{~a také~} x^n = \underbrace{x * x * \ldots * x}_{n\text{-krát}}.
\end{equation}
Máme tedy návod, jak teoreticky vytvořit více operací. Další operací je 
\begin{equation}
x\#n = \underbrace{x \string^ x \string^ \ldots \string^ x}_{n\text{-krát}}
\end{equation}
a nazýváme ji \textit{tetrace} (\textit{tetration}) \cite{Operations}.

Pro operace vyšší arity je pak zvykem používat prefixovou notaci jednoho velkého znaménka -- součet $a_0 + a_1 + a_2$ je pak zapsán jako $\scalerel*{+}{\sum}(a_0, a_1, a_2)$ nebo též $\sum_{i=0}^2(a_i)$. Součin $a_0 * a_1 * a_2$ pak $\scalerel*{*}{\sum}(a_0, a_1, a_2)$ nebo $\prod_{i=0}^2(a_i)$.

\begin{table}[H]
\begin{mdframed}[backgroundcolor=lightpink,innertopmargin=-2.5pt,innerbottommargin=2.5pt]
\centering
\caption{Symboly operací s čísly}
\begin{tabular}{|>{\columncolor[gray]{1}}r|>{\columncolor[gray]{1}}l|}
\hline
Operace & Značení \\ \hline \hline
sčítání & $x+y$ \\
odčítání & $x-y$ \\
násobení & $x*y$, $x\cdot y$ nebo $xy$\\
dělení & $x/y$ nebo $\frac{x}{y}$ \\
umocňování & $x^y$ nebo $x\string^ y$\\ 
odmocňování & $\sqrt[y]{x}$ \\\hline
\end{tabular}\label{tab:znamenka_operaci}

Tabulka zobrazuje zápis binárních operací s čísly $x$ a $y$.
\end{mdframed}
\end{table}

\subsection{Funkce čísel}
\label{funkce_cisel}
Další zajímavá manipulace s čísly jsou tzv. funkce, které si (ne)lze představit jako nekonečnou tabulku o dvou sloupcích -- výstupní a výstupní hodnoty (jako příklad pro funkci $sinus$ jsem vytvořil tabulku \ref{tab:sinus_input_output}). Příkladem takových funkcí jsou funkce \textit{goniometrické}, funkce \textit{exponenciální} či \textit{logaritmus}.

\begin{table}[H]
\begin{mdframed}[backgroundcolor=lightpink,innertopmargin=-2.5pt,innerbottommargin=2.5pt]
\centering
\caption{Nekonečná vstupně/výstupní tabulka funkce $sinus$}
\label{tab:sinus_input_output}
\begin{tabular}{|>{\columncolor[gray]{1}}c|>{\columncolor[gray]{1}}c|}
\hline
Vstup & Výstup\\ \hline \hline
$0$ & $0$ \\ 
$1$ & $0.8414709848078965\ldots$ \\
$2$ & $0.9092974268256817\ldots$ \\
$3$ & $0.1411200080598672\ldots$ \\
$4$ & $-0.7568024953079282\ldots$ \\
\multicolumn{2}{|>{\columncolor[gray]{1}}c|}{$\vdots$} \\ \hline
\end{tabular}

Tabulka zobrazuje v prvním sloupci vstup do funkce sinus a ve druhém její výstup. Tabulka je nekonečná ve vertikálním směru, v horizontálním jsou pouze dva sloupce -- pro argument a vrácenou hodnotu.
\end{mdframed}
\end{table}

Funkce jsou velmi užitečným a potřebným nástrojem takřka ve všech odvětvích matematiky a jsou často zdrojem \textit{iracionálních} (\textit{transcendentních}) čísel -- například takový $\mathrm{atan}(1)$ dá za výsledek čtvrtinu \textit{Ludolfova čísla}.

\begin{definition}[Uspořádaná $n$-tice \cite{TeMno}]
Jsou-li dány množiny $a_1,a_2,\ldots a_m$, uspořádanou $n$-tici množin $a_1,a_2,\ldots,a_n$ pro $n<m$ definujeme tak, že pro $n=1$ položíme
\begin{equation}
\langle a_1 \rangle = a_1
\end{equation}
a je-li již definována uspořádaná $n$-tice $\langle a_1, a_2,\ldots , a_n \rangle$, položíme
\begin{equation}
\langle a_1, a_2, \ldots a_{n+1} \rangle = \langle \langle a_1, a_2 \ldots a_n \rangle , a_{n+1} \rangle .
\end{equation}
\end{definition}

\begin{definition}[Kartézský součin \cite{EPJVMAI} \cite{RachALG1}]
Nechť $A_0, A_1 \ldots A_{n}$ jsou množiny. Symbolem $\bigtimes_{i=0}^{n} A_i$ či $A_0 \times A_1 \times \ldots \times A_{n}$ označujeme množinu všech uspořádaných $(n+1)$-tic tvaru $\langle a_0, a_1, \ldots a_{n} \rangle$, přičemž $(a_0 \in A_0)\land (a_1 \in A_1)\land\ldots\land (a_{n}\in A_{n})$, neboli
\begin{equation}
A_0 \times A_1 \times \ldots \times A_{n} = \{\langle a_0 , a_1, \ldots , a_{n}  \rangle | (\forall i \in \mathbb{N}, i \leq n)(a_i \in A_i)\}
\end{equation}
a tuto množinu nazýváme kartézským součinem množin $A_0, A_1, \ldots A_n$.
\end{definition}

\begin{remark}[Množinovost kartézského součinu]\label{rem:mnozinovost_KS}
Že je kartézský součin (KS) množina jsem zavedl definitoricky, jako to udělaly autorky v \cite{EPJVMAI}, ovšem možná to nemusí být úplně jasné. V \cite{TeMno} se kartézský součin zavádí jako třída (soubor množin, který množina být nemusí -- například třída všech množin množinou není) a poté se pomocí axiomu potence jeho množinovost dokazuje. Tato práce na tomto faktu nestojí a proto jsem si dovolil přijmout množinovost KS jako fakt, ač je -- podobně jako všech 13 axiomů reálných čísel včetně axiomu o supremu (poté tedy věty o supremu) \cite{DK:DPFJP} -- dokazatelná z ZF teorie množin.
\end{remark}

\begin{definition}[Relace \cite{EPJVMAI}]
Relace mezi množinami $A$ a $B$ je libovolná podmnožina $\mathcal{R}$ kartézského součinu $A\times B$. Je-li $A=B$, mluvíme o relaci na $A$. Náleží-li dvojice $\langle a, b \rangle$ relaci $\mathcal{R}$, t.j. $\langle a,b \rangle \in \mathcal{R}$, říkáme, že $a$ a $b$ jsou v relaci $\mathcal{R}$, a zapisujeme též $a\mathcal{R}b$.
\end{definition}

\begin{definition}[Zobrazení \cite{EPJVMAI}]
Relaci $f \subseteq A \times B$ nazveme \textit{zobrazením} množiny $A$ do množiny $B$, jestliže platí, že ke každému prvku $x \in A$ existuje právě jeden prvek $y \in B$ takový, že $\langle x, y \rangle \in f$.
\end{definition}

Je-li relace $f \subseteq A \times B$ zobrazení, pak skutečnost, že $\langle x, y \rangle \in f$ zapisujemme ve tvaru $y = f(x)$. Rovněž používáme zápis $f:A\rightarrow B$, což znamená, že $f$ je zobrazení $A$ do $B$. Dále $x$ nazýváme \textit{nezávisle proměnnou} a $y$ \textit{závisle proměnnou} \cite{DK:DPFJP}.

\begin{definition}[Reálná posloupnost \cite{EPJVMAI}]
Zobrazení množiny $\mathbb{N}$ do množiny $\mathbb{R}$ nazýváme \textit{reálná posloupnost}.
\end{definition}

Místo obecného značení $a:\mathbb{N}\rightarrow\mathbb{R}$ pro zobrazení resp. značení $a(n)$ pro obraz bodu $n$ se vžilo pro posloupnost značení $\{a_n\}_{n\in\mathbb{N}}$ nebo $\{a_n\}_{n=0}^{\infty}$. Obraz bodu $n$ se značí $a_n$ a říkáme mu také $n$-tý člen posloupnosti $a$ \cite{EPJVMAI}. 

\begin{definition}[Nekonečná číselná řada \cite{ZDVNNR}]
Nechť $\{a\}_{n\in\mathbb{N}}$ je reálná posloupnost. Symbol $\sum_{n\in\mathbb{N}}a_n$ nebo $a_0 + a_1 + a_2 + \ldots$ nazýváme nekonečnou číselnou řadou.
\end{definition}

\begin{definition}[Posloupnost částečných součtů \cite{ZDVNNR}]
Posloupnost $\{s_n^a\}$, kde $s^a_n=\sum_{i=0}^na_i$ nazýváme \textit{posloupnost částečných sou\-čtů} řady $\sum_{n\in\mathbb{N}}a_n$.
\end{definition}

\begin{definition}[Konvergence řady \cite{ZDVNNR}]
Existuje-li vlastní limita $\lim_{n\to\infty}s^a_n = s$, řekneme, že řada $\sum_{n\in\mathbb{N}}a_n$ konverguje a má součet $s$.
\end{definition}

\begin{definition}[Divergence řady \cite{ZDVNNR}]
Neexistuje-li vlastní limita $\lim_{n\to\infty}s^a_n$, řekneme, že řada $\sum_{n\in\mathbb{N}}$ diverguje. Pokud limita $\lim_{n\to\infty}s^a_n$ neexistuje, říkáme, že řada osciluje. Pokud je $\lim_{n\to\infty}s^a_n = \infty$, pak říkáme, že řada diverguje k $\infty$. Pokud je $\lim_{n\to\infty}s^a_n = -\infty$, pak říkáme, že řada diverguje k $-\infty$. 
\end{definition}

\begin{definition}[Zbytek po $n$-tém členu řady \cite{ZDVNNR}]
Nechť $\sum_{n\in\mathbb{N}}a_n$ je konvergentní řada. Její součet $s$ lze psát ve tvaru $s = s_n^a + R_n^a$, kde $s_n^a=\sum_{i=0}^na_i$ je $n$-tý částečný součet řady $\sum_{n\in\mathbb{N}}a_n$ a $R_n^a = \sum_{i=n+1}^{\infty}a_i$ se nazývá \textit{zbytek po $n$-tém členu} a znamená velikost chyby, které se doupouštíme, když místo celé řady posčítáme pouze prvních $n$ členů posloupnosti $\{a_n\}_{n\in\mathbb{N}}$.
\end{definition}

\begin{definition}[Reálná funkce reálné proměnné \cite{DK:DPFJP}]
Buď $M\subseteq R$. Zobrazení $f:M\rightarrow\mathbb{R}$ nazýváme \textit{reálnou funkcí reálné proměnné} nebo stručně \textit{funkcí} jedné proměnné.

Množina $M$ se nazývá \textit{definiční obor} funkce $f$ a značí se $D(f)$, množina $H(f) = \{f(x) | x \in M \}$ se nazývá \textit{obor hodnot} funkce $f$. 
\end{definition}

\begin{example}[Sinus jako zobrazení]
Funkce $y=sin(x)$ je definována pro všechna $x\in\mathbb{R}$ a její obor hodnot je interval $[-1,1]$, jedná se tedy o reálnou funkci reálné proměnné. Pokud budeme hledět jen na obrazy $sin(x), x\in\mathbb{N}$ (jako v tabulce \ref{tab:sinus_input_output}), jedná se o reálnou posloupnost.
\end{example}

Povšimněme si, že zobrazení $f:A\rightarrow B$ je definováno tak, že $A$ i $B$ jsou množiny. Jak ale víme z poznámky \ref{rem:mnozinovost_KS}, je množinou i kartézský součin množin. Tedy lze definovat též zobrazení $\mathbb{R}\times\mathbb{I}\rightarrow\mathbb{Q}\times\mathbb{Z}\times\mathbb{N}$ atp., aniž bychom museli definici zobrazení jakkoli upravovat.

Při bližším zkoumáním by se daly najít jisté podobnosti mezi operacemi a funkcemi. Pokud se odprostíme od infixové notace a místo $a + b$ napíšeme $+(\langle a, b\rangle)$, lze i operace vyjádřit jako funkce. Matematickou operaci tedy chápeme jako speciální případ funkce, tedy že $n$-ární reálná operace je funkce $\bigtimes_{i=1}^{n}\mathbb{R}\rightarrow \mathbb{R}$.

Ba co více. Pokud vezmeme funkci $\emptyset \rightarrow \mathbb{R}$, zjistíme, že se jedná o konstantu, protože podle definice zobrazení pokud jsou v relaci $\langle \emptyset, x\rangle$ a $\langle \emptyset, y\rangle$, pak $x = y$. Čili taková funkce vždy se zobrazuje na stejné číslo a proto se jedná konstantu. Takže funkcemi můžeme vymodelovat jakákoli čísla i manipulaci s nimi.

\begin{definition}[Mocninná řada \cite{ZDVNNR}]
Buď $\{a_n\}_{n\in\mathbb{N}}$ posloupnost reálných čísel, $x_0$ libovolné reálné číslo. \textit{Mocninnou řadou} se středem v bodě $x_0$ a koeficienty $a_n$ rozumíme řadu ve tvaru
\begin{equation}
a_0+a_1(x-x_0)+a_2(x-x_2)^2+\ldots+a_n(x-x_0)^n+\ldots=\sum_{n\in\mathbb{N}}a_n(x-x_0)^n.
\end{equation}
\end{definition}

\begin{definition}[Taylorova řada \cite{ZDVNNR}]\label{def:taymac_rada}
Nechť funkce $f$ má v bodě $x_0$ derivace všech řádů. Mocninnou řadu ve tvaru $\sum_{n\in\mathbb{N}}\frac{f^{(n)}(x_0)}{n!}(x-x_0)^n$ nazýváme \textit{Taylorovou řadou} funkce $f$ v bodě $x_0$. Je-li $x_0=0$, mluvíme o \textit{Maclaurinově řadě} funkce $f$ a je ve tvaru $\sum_{n\in\mathbb{N}}\frac{f^{(n)}(0)}{n!}x^n$. Zbytku Taylorovy řady říkáme Taylorův zbytek a značíme ho $R^{f,a}_n(x)$.
\end{definition}

\begin{definition}[Geometrická řada \cite{ZDVNNR}]
Řadu nazýváme geometrickou, pokud je ve tvaru
\begin{equation}
a + a*q + a*q^2 + a*q^3 +\ldots = \sum_{i\in\mathbb{N}}aq^i
\end{equation}
\end{definition}

\begin{fact}[Částečný součet geometrické řady \cite{ZDVNNR}]\label{vet:o_castecnem_souctu_geometricke_rady}
Pro geometrickou řadu ve tvaru $\sum_{i\in\mathbb{N}}aq^i$ a $|q|<1$ platí
\begin{equation}
s^a_n=\sum_{i=0}^na*q^i=a\frac{1-q^{n+1}}{1-q}
\end{equation}
Pro důkaz vizte podkapitolu \ref{duk:o_castecnem_souctu_geometricke_rady} v příloze \ref{pril:dukazy}.
\end{fact}

\begin{fact}[Geometrická řada]\label{vet:o_geometricke_rade}
Pro geometrickou řadu ve tvaru $\sum_{i\in\mathbb{N}}aq^i$ a $|q|<1$ platí
\begin{equation}
\sum_{i\in\mathbb{N}}aq^i = \frac{a}{1-q}
\end{equation}
Pro důkaz vizte podkapitolu \ref{duk:o_geometricke_rade} v příloze \ref{pril:dukazy}.
\end{fact}

\begin{fact}[Zbytek geometrické řady]\label{vet:o_zbytku_geometricke_rady}
Pro $n$-tý zbytek geometrické řady $\sum_{i\in\mathbb{N}}aq^i$ platí
\begin{equation}
R_n = \frac{aq^{n+1}}{1-q}
\end{equation}
Pro důkaz vizte podkapitolu \ref{duk:o_zbytku_geometricke_rady} v příloze \ref{pril:dukazy}.
\end{fact}