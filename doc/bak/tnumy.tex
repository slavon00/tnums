Svoje struktury jsem zvolil jako funkce dvou proměnných -- reprezentovaného čísla a přesnosti. Podíváme se, jak se dají matematicky definovat, po Lispovsku vymodelovat a na závěr dokážu několik tvrzení, aby čtenář pochopil, jak se s tnumy pracuje. Nebude chybět prvních několik kódů, i když nejvíce programování bude spíše ke konci této části.

\subsection{Vztah čísel a tnumů}
Funkce, kterými budu modelovat rekurzivní čísla a implementovat práci s nimi pomocí jazyka Lisp, budu nazývat \textit{True Numbers}, zkráceně \textit{tnums}. Vystihuje to jejich podstatu a cíl - budou ve výsledku opravdovější a přesnější, než ostatní čísla, která by měla mít nekonečný rozvoj, ale jsou uložena jako hodnoty – často jako nějaká plovoucí čísla (jakási Nil Numbers). Vzniknuvší knihovna se pak jmenuje \texttt{tnums}.

\begin{definition}[Tnum]\label{def:tnum}
Funkce $\tnum{x}:(0,1)\rightarrow\mathbb{Q}$, která pro všechna $\varepsilon\in (0,1)$ vrací hodnotu $\txe$ splňující nerovnost
\begin{equation}\label{rov:def:tnum}
|\txe - x |\leq \varepsilon
\end{equation}
se nazývá \textit{tnum} čísla $x$.

Množinu všech tnumů čísla $x$ značíme $\Tnum{x}=\{\tnum{x}|(\forall x \in\mathbb{R})(\forall\varepsilon\in (0,1)):|\txe-x|\leq\varepsilon\}$, množinu všech tnumů pak symbolem $\mathfrak{T}$.
\end{definition}

Tnum $\tnum{x}$ je struktura představující rekurzivní číslo $x$. Místo vztahu \ref{rov:def:tnum} lze psát $\txe\in[x-\varepsilon,x+\varepsilon]$. Při výpočtu hodnoty tnumu nejprve tento tnum vytvoříme (toto bude výpočetně rychlé) a jakmile bude vytvořený, necháme ho vyčíslit (zavolat s přesností) a toto může být na dlouho. Vyčíslení tedy odkládáme na nejpozdější možnou dobu. Mluvíme o líném vyhodnocování.

Tnumy přesných čísel lze vyčíslit s dokonalou přesností. Využiji co nejvíce z~přesnosti, kterou nabízí Lisp a ten přesně reprezentuje všechna racionální čísla. To je ve shodě s představou o vyčíslení rekurzivního čísla. Pomocí $\tnum{r}(\varepsilon)$ tedy získáváme $q$ z nerovnice \ref{rov:rac_u_real}. Číslu, jak ho chápe Lisp říkám nadále \textit{num} (number).

\begin{lemma}[O numu jako tnumu]\label{lem:num-to-tnum}
Pro všechna $x\in\mathbb{R}$ a všechna $\varepsilon \in (0,1)$ platí: číslo $\txe$ lze nahradit číslem $x$.
\begin{proof}
Z nerovnosti \ref{rov:def:tnum} získáváme $|\txe - x |\leq \varepsilon$. Po dosazení $\txe := x$ pak $|x - x | = 0 \leq \varepsilon$, což platí pro všechna myslitelná $x$ i $\varepsilon$.
\end{proof}
\end{lemma}

Nejpřesnější reprezentace čísla reprezentovaného tnumem reprezentujícím čí\-slo je toto číslo samotné. Proto je tedy vhodné co nejvíce takových čísel přenechat na reprezentaci Lispu a počítat jen s těmi, které nezvládne. Protože Lisp pracuje i se zlomky (typ \texttt{ratio}), nejvyšší obor čísel, který umí vracet s nulovou odchylkou jsou racionální čísla.

\begin{lispcode}{\texttt{num-to-tnum}}{Funkce převádějící číslo z interní reprezentace Lispu na tnum}
(\textcolor{funkcionalni}{defun} \textcolor{pojmenovan}{num-to-tnum} (num)
  (\textcolor{vedlejsi}{let} ((rat_num (\textcolor{matematicke}{rationalize} num)))
    (\textcolor{funkcionalni}{lambda} (eps) (\textcolor{vedlejsi}{declare} (\textcolor{vedlejsi}{ignore} eps))
      rat_num)))
\end{lispcode}

Převod opačným směrem je přímočarý. Chceme-li číslo $x$ s~přesností $\varepsilon$, stačí zavolat $\txe$. Přesnost musí být z $(0, 1)$, jiné číslo interpretujeme jako $10^{-|\varepsilon|}$.

\begin{lemma}[O převodu tnumu na num]
Pokud existuje funkce $\tnum{x}\in\Tnum{x}$, pak po zavolání s argumentem $\varepsilon$ vrací hodnotu $\txe$ splňující $(|\txe - x |\leq \varepsilon)$.
\begin{proof}
Plyne přímo z definice \ref{def:tnum}.
\end{proof}
\end{lemma}

\begin{lispcode}{\texttt{rat-expt}}{Funkce pro racionální umocňování}
(\textcolor{funkcionalni}{defun} \textcolor{pojmenovan}{rat-expt} (num exp)
  (\textcolor{matematicke}{rationalize} (\textcolor{matematicke}{expt} num exp)))
\end{lispcode}

\begin{lispcode}{\texttt{tnum-to-num}}{Funkce převádějící tnum na číslo}
(\textcolor{funkcionalni}{defun} \textcolor{pojmenovan}{tnum-to-num} (tnum eps)
  (\textcolor{funkcionalni}{when} (\textcolor{funkcionalni}{or} (\textcolor{matematicke}{>=} 0 eps) (\textcolor{matematicke}{<=} 1 eps))
    (\textcolor{vedlejsi}{setf} eps (\textcolor{moje}{rat-expt} 10 (\textcolor{matematicke}{-} (\textcolor{matematicke}{abs} eps)))))
  (\textcolor{funkcionalni}{funcall} tnum (\textcolor{matematicke}{rationalize} eps)))
\end{lispcode}

Zatímco tedy pro převod z čísla na tnum jsme toto mohli udělat pro všechna čísla, opačným směrem toto funguje pouze za předpokladu, že daný tnum existuje. V našem systému teď máme jen tnumy pro racionální čísla a umíme je převádět tam a zpět. V dalším textu tedy půjde hlavně o to zaplnit tuto mezeru a přinést existenci co nejvíce tnumů.

\subsection{Ludolfovo číslo}
Prvním iracionálním číslem, které do knihovny přidáme je číslo Ludolfovo.

\begin{definition}[Ludolfovo číslo \cite{piratio}]
Ludolfovým číslem myslíme poměr obvodu kružnice k jejímu průměru.
\end{definition}

Ludolfovo číslo je asi nejslavnější transcendentní konstanta a proto není divu, že pro její vyčíslení existuje bezpočet vzorců. Asi nejpřímější je Leibnizův vzorec, který vypočítává čtvrtinu Ludolfova čísla a plyne z Taylorovy řady funkce arctan v bodě 1. Pokud Ludolfovo číslo značím $\pi$, pak ho lze zapsat jako $\pi=4\sum_{n\in\mathbb{N}}\frac{(-1)^n}{2n+1}$ \cite{approxpi}, tato řada ale konverguje velmi pomalu. Já proto použiji aproximaci jinou. Tento vzorec se jmenuje BBP podle svých tvůrců (Bailey, Borwein, Plouffe) a je zapsán ve formě řady.

\begin{fact}[Ludolfovo číslo jako řada \cite{BBP}]
Nechť $\pi$ značí Ludolfovo číslo. Pak jej lze zapsat jako
\begin{equation}\label{rov:pi-rada}
\pi=\sum_{i\in\mathbb{N}}\frac{1}{16^i}\left(\frac{4}{8i+1}-\frac{2}{8i+4}-\frac{1}{8i+5}-\frac{1}{8i+6}\right).
\end{equation}
\end{fact}

Mám tedy řadu, která generuje konstantu, kterou chci přidat do \texttt{tnums}. Výraz $\left(\frac{4}{8i+1}-\frac{2}{8i+4}-\frac{1}{8i+5}-\frac{1}{8i+6}\right)$ je pro $i>0$ menší než jedna, proto se každý nenultý člen může zhora omezit $\frac{1}{16^i}$ a to je geometrická posloupnost, jejíž zbytek je dle faktu \ref{vet:o_zbytku_geometricke_rady} roven $\frac{1}{16^{i+1}}*\frac{16}{15}$, což je $\frac{1}{16^i*15}$. Platí tedy

\begin{equation}
\left|\pi - \sum_{i=0}^n\frac{1}{16^i}\left(\frac{4}{8i+1}-\frac{2}{8i+4}-\frac{1}{8i+5}-\frac{1}{8i+6}\right) \right| \leq \frac{1}{16^n*15}.
\end{equation}

\begin{consequence}[Tnum Ludolfova čísla]
Nechť $\tnum{}$ je funkce s předpisem $\tnum{}(\varepsilon)=$\uv{najdi nějaké $n$ tak, aby platilo $/(16^n15)\leq\varepsilon$ a poté vrať $n$-tý částečný součet řady ze vztahu \ref{rov:pi-rada}}, pak $\tnum{}\in\Tnum{\pi}$.
\end{consequence}

Kód vypadá trochu složitěji, ale není to nic jiného, než co bylo právě popsáno. Nižší čitelnost je zde vykoupena vyšší efektivitou a protože je vyčíslování $\pi$ jedna z nejdůležitějších funkcionalit, rozhodl jsem se ji zavést takto efektivně, ač na úkor čitelnosti.

\begin{lispcode}{\texttt{tnum-pi}}{Funkce na vytvoření tnumu Ludolfova čísla}
(\textcolor{funkcionalni}{defun} \textcolor{pojmenovan}{tnum-pi} ()
  (\textcolor{funkcionalni}{lambda} (eps)
    (\textcolor{vedlejsi}{let} ((/16pown 0) (result 0) (above 1))
      (\textcolor{funkcionalni}{loop} \textcolor{obarvi}{for} n \textcolor{obarvi}{from} 0
            \textcolor{obarvi}{until} (\textcolor{matematicke}{<=} above eps)
            \textcolor{obarvi}{do} (\textcolor{funkcionalni}{progn} 
                 (\textcolor{vedlejsi}{setf} /16pown (\textcolor{moje}{rat-expt} 16 (\textcolor{matematicke}{-} n)))
                 (\textcolor{vedlejsi}{incf} result
                       (\textcolor{matematicke}{*} /16pown
                          (\textcolor{matematicke}{-} (\textcolor{matematicke}{/} 4 (\textcolor{matematicke}{+} (\textcolor{matematicke}{*} 8 n) 1))
                             (\textcolor{matematicke}{/} 2 (\textcolor{matematicke}{+} (\textcolor{matematicke}{*} 8 n) 4))
                             (\textcolor{matematicke}{/} 1 (\textcolor{matematicke}{+} (\textcolor{matematicke}{*} 8 n) 5))
                             (\textcolor{matematicke}{/} 1 (\textcolor{matematicke}{+} (\textcolor{matematicke}{*} 8 n) 6)))))
                 (\textcolor{vedlejsi}{setf} above (\textcolor{matematicke}{/} /16pown 15)))
            \textcolor{obarvi}{finally} (\textcolor{funkcionalni}{return} result)))))
\end{lispcode}

\subsection{Přenásobování numem}
Posledním dílkem, který přidám v této kapitole je přenásobování tnumu konstantou. Když už máme všechna racionální čísla a Ludolfovo číslo, zvládneme pak i například $2\pi$ nebo $\frac{\pi}{-2}$.

\begin{theorem}[O přenásobení tnumu racionální konstantou]
Necht $c\in\mathbb{Q}$, $\tnum{x}\in\Tnum{x},x\in\mathbb{R}$ a funkce $\tnum{}(\tnum{x},c)$ má předpis
\begin{equation}
\tnum{}(\tnum{x},c)(\varepsilon)=\begin{cases}c*\tnum{x}\left(\frac{\varepsilon}{|c|}\right) & \text{pro~}c\not = 0,\\0&\text{jinak,}\end{cases}
\end{equation}
pak $\tnum{}(\tnum{x},c)\in\Tnum{x*c}$.
\begin{proof}
Pokud přenásobíme hodnotu tnumu nulou, je výsledkem nula, protože je to agresivní prvek vůči násobení. Znění věty pro nenulovou konstantu dokážeme tak, že z předpokladu $|\txe -x|\leq\varepsilon$ odvodíme $|c*\tnum{x}(\frac{\varepsilon}{|c|})-c*x|\leq\varepsilon$. Protože pracujeme s nerovnicemi, budeme v důkazu postupovat dvěmi větvemi -- pro $c$ kladné a záporné.

Z definice tnumu předpokládáme
\begin{equation}
|\txe-x|\leq\varepsilon,
\end{equation}
po přenásobení kladným $c>0$ dostáváme
\begin{equation}
c*|\txe-x|\leq c*\varepsilon,
\end{equation}
protože je ale $c$ kladné, můžu jím absolutní hodnotu roznásobit
\begin{equation}
|c*\txe-c*x|\leq c*\varepsilon,
\end{equation}
a protože na pravé straně potřebuji přesnost $\varepsilon$, v argumentu ji podělím $c$ a pak
\begin{equation}
\left|c*\tnum{x}\left(\frac{\varepsilon}{c}\right)-c*x\right|\leq \varepsilon.
\end{equation}
Pro zápornou konstantu je běh důkazu podobný a protože jako argument tnumů bereme kladné číslo, přibývá v děliteli v argumentu tnumu ještě absolutní hodnota. Dohromady pak získáváme
\begin{equation}
\left|c*\txe^{x}\left(\frac{\varepsilon}{|c|}\right)-c*x\right|\leq\varepsilon,
\end{equation}
což jsme chtěli ukázat.
\end{proof}
\end{theorem}

\begin{lispcode}{\texttt{tnum*num}}{Funkce přenásobující tnum racionální konstantou}
(\textcolor{funkcionalni}{defun} \textcolor{pojmenovan}{tnum*num} (tnum num)
  (\textcolor{vedlejsi}{let} ((rat_num (\textcolor{matematicke}{rationalize} num)))
    (\textcolor{funkcionalni}{lambda} (eps)
      (\textcolor{funkcionalni}{if} (\textcolor{funkcionalni}{zerop} num)
          (\textcolor{moje}{num-to-tnum} 0)
        (\textcolor{matematicke}{*} (\textcolor{moje}{tnum-to-num} tnum (\textcolor{matematicke}{/} eps (\textcolor{matematicke}{abs} rat_num))) rat_num)))))  
\end{lispcode}

\begin{consequence}[Opačný tnum]\label{dusl:negace_tnumu}
Nechť $\tnum{x}\in\Tnum{x}, x\in\mathbb{R}$ a funkce $\tnum{}(\tnum{x})$ má předpis
\begin{equation}
\tnum{}(\tnum{x})(\varepsilon)=-\txe,
\end{equation}
pak $\tnum{}(\tnum{x})\in\Tnum{-x}$.
\begin{proof}
Protože $-x = (-1)x$ a $|-1|=1$, pak podle přechozí věty dostáváme $-\txe=(-1)\txe=(-1)\tnum{x}(\frac{\varepsilon}{1})=(-1)\tnum{x}(\frac{\varepsilon}{|-1|})=\tnum{(-1)x}(\varepsilon)=\tnum{-x}(\varepsilon)\in\Tnum{-x}$.
\end{proof}
\end{consequence}

\begin{lispcode}{\texttt{-tnum}}{Funkce pro opačný tnum}
(\textcolor{funkcionalni}{defun} \textcolor{pojmenovan}{-tnum} (tnum)
  (\textcolor{moje}{tnum*num} tnum -1))
\end{lispcode}

