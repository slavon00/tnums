Struktury jsou dány jako funkce jedné proměnné -- přesnosti. Je potřeba je matematicky nadefinovat a poté v Lispu vymodelovat.

\subsection{Vztah čísel a tnumů}
Funkce, kterými budu modelovat rekurzivní čísla a které budu poté v Lispu implementovat nazývám \textit{True Numbers}, zkráceně \textit{tnums}. Jsou totiž opravdovější než čísla, která jsou uložena jako hodnoty, čímž ztrácí na přesnosti. Vzniknuvší knihovna se pak jmenuje \texttt{tnums}.

\begin{definition}[Tnum]\label{def:tnum}
Funkce $\tnum{x}:(0,1)\rightarrow\mathbb{Q}$, která pro všechna $\varepsilon\in (0,1)$ vrací hodnotu $\txe$ splňující nerovnost
\begin{equation}\label{rov:def:tnum}
|\txe - x |\leq \varepsilon
\end{equation}
se nazývá \textit{tnum} \normalfont{[ti:n{\textturnv}m]} čísla $x$.

Množinu všech tnumů čísla $x$ značíme $\Tnum{x},(\forall x \in\mathbb{R})(\Tnum{x}=\{\tnum{x}|(\forall\varepsilon\in (0,1)):|\txe-x|\leq\varepsilon\})$. Množinu všech tnumů pak značíme symbolem $\mathfrak{T}$.
\end{definition}

Tnum $\tnum{x}$ je struktura představující rekurzivní číslo $x$. Při výpočtu jeho hodnoty nejprve tento tnum vytvoříme (výpočetně rychlé), a pak ho necháme vyčíslit, tj. zavolat s přesností (výpočetně pomalé). Vyčíslení tedy odkládáme na nejpozdější možnou dobu, mluvíme o líném vyhodnocování.

Tnumy přesných čísel lze vyčíslit s dokonalou přesností. Lisp přesně reprezentuje všechna racionální čísla. To je ve shodě s principem vyčíslení rekurzivního čísla. Pomocí $\tnum{r}(\varepsilon)$ tedy získáváme $q$ z nerovnice \eqref{rov:rac_u_real} a též $enum(s_r,\varepsilon)$ z nerovnice \eqref{ner:struktura}. Číslo, jak ho chápe Lisp dále označuji \textit{num}.

\begin{lemma}[O numu jako tnumu]\label{lem:num-to-tnum}
Pro všechna $x\in\mathbb{R}$ a všechna $\varepsilon \in (0,1)$ platí: číslo $\txe$ lze nahradit číslem $x$.
\begin{proof}
Z nerovnosti \eqref{rov:def:tnum} získáváme $|\txe - x |\leq \varepsilon$. Po dosazení $\txe := x$ pak $|x - x | = 0 \leq \varepsilon$, což platí pro všechna $x\in\mathbb{R}$ i $\varepsilon\in{(0,1)}$.
\end{proof}
\end{lemma}

Nejpřesnější reprezentace čísla je toto číslo samotné. Lisp pracuje i se zlomky (datový typ \texttt{ratio}), proto mohou být přesná všechna racionální čísla.

\begin{lispcode}{\texttt{num-to-tnum}}{Funkce převádějící num na tnum}
(\textcolor{funkcionalni}{defun} \textcolor{pojmenovan}{num-to-tnum} (num)
  (\textcolor{vedlejsi}{let} ((rat_num (\textcolor{matematicke}{rationalize} num)))
    (\textcolor{funkcionalni}{lambda} (eps) (\textcolor{vedlejsi}{declare} (\textcolor{vedlejsi}{ignore} eps))
      rat_num)))
\end{lispcode}

Převod opačným směrem je přímočarý. Chceme-li číslo $x\in\mathbb{R}$ s~přesností $\varepsilon\in{(0,1)}$, stačí zavolat $\txe$. Mimo přípustný interval $\varepsilon$ chápeme jako $10^{-|\varepsilon|}$.

\begin{lemma}[O převodu tnumu na num]
Pokud existuje funkce $\tnum{x}\in\Tnum{x}$, pak po zavolání s argumentem $\varepsilon$ vrací hodnotu $\txe$ splňující $(|\txe - x |\leq \varepsilon)$.
\begin{proof}
Plyne přímo z definice \ref{def:tnum}.
\end{proof}
\end{lemma}

\begin{lispcode}{\texttt{rat-expt}}{Funkce pro racionální umocňování}
(\textcolor{funkcionalni}{defun} \textcolor{pojmenovan}{rat-expt} (num exp)
  (\textcolor{matematicke}{rationalize} (\textcolor{matematicke}{expt} num exp)))
\end{lispcode}

\begin{lispcode}{\texttt{tnum-to-num}}{Funkce převádějící tnum na num}
(\textcolor{funkcionalni}{defun} \textcolor{pojmenovan}{tnum-to-num} (tnum eps)
  (\textcolor{funkcionalni}{when} (\textcolor{funkcionalni}{or} (\textcolor{matematicke}{>=} 0 eps) (\textcolor{matematicke}{<=} 1 eps))
    (\textcolor{vedlejsi}{setf} eps (\textcolor{moje}{rat-expt} 10 (\textcolor{matematicke}{-} (\textcolor{matematicke}{abs} eps)))))
  (\textcolor{funkcionalni}{funcall} tnum (\textcolor{matematicke}{rationalize} eps)))
\end{lispcode}

Z čísla na tnum je převod jednoduchý, opačným směrem lze převádět jen, pokud tnum existuje. V knihovně teď jsou jen tnumy racionálních čísel a aparát pro převody mezi $\mathfrak{T}$ a $\mathbb{Q}$. Teď půjde o tvorbu co nejvíce tnumů iracionálních čísel.

\subsection{Ludolfovo číslo}
První iracionální konstantou, která do knihovny přibude je Ludolfovo číslo.

\begin{definition}[Ludolfovo číslo \cite{piratio}]
Ludolfovo číslo $\pi$ je dáno jako poměr obvodu kružnice k jejímu průměru.
\end{definition}

Ludolfovo číslo je nejslavnější transcendentní konstanta, pro jejíž vyčíslení existuje přemnoho vzorců, například vzorec Leibnizův: $\pi=4\sum_{n\in\mathbb{N}}\frac{(-1)^n}{2n+1}$ \cite{approxpi}.

Rychleji konverguje řada v BBP (tvůrci Bailey, Borwein, Plouffe) vzorci.

\begin{fact}[Kohoutkový BBP vzorec \cite{BBP}]
\begin{equation}\label{rov:pi-rada}
\pi=\sum_{i\in\mathbb{N}}\frac{1}{16^i}\left(\frac{4}{8i+1}-\frac{2}{8i+4}-\frac{1}{8i+5}-\frac{1}{8i+6}\right).
\end{equation}
\end{fact}

Máme řadu čísla, jež chceme přidat do \texttt{tnums}. Výraz v závorce je pro $i>0$\\ menší než jedna. Proto lze každý $i$-tý člen, kde $i>0$, zhora omezit $16^{-i}$. Omezující členy tvoří geometrickou řadu, jejíž zbytek je dle faktu \ref{vet:o_zbytku_geometricke_rady} roven $\frac{1}{16^i*15}$. Platí\\
\begin{equation}
\left|\pi - \sum_{i=0}^n\frac{1}{16^i}\left(\frac{4}{8i+1}-\frac{2}{8i+4}-\frac{1}{8i+5}-\frac{1}{8i+6}\right) \right| \leq \frac{1}{16^n*15}.
\end{equation}

\begin{consequence}[Tnum Ludolfova čísla]
Nechť $\tnum{}()$ je funkce s předpisem 
\begin{equation}
\tnum{}()(\varepsilon)=\left[\begin{array}{l}
\mathrm{1.~}\text{Najdi~nejmenší~}n\in\mathbb{N}^+\text{~tak,~aby~}/(16^n15)\leq\varepsilon;\\
\mathrm{2.~}\text{Vrať~}\sum_{i=0}^n\frac{1}{16^i}\left(\frac{4}{8i+1}-\frac{2}{8i+4}-\frac{1}{8i+5}-\frac{1}{8i+6}\right),
\end{array}\right.
\end{equation} pak $\tnum{}\in\Tnum{\pi}$.
\end{consequence}
\begin{lispcode}{\texttt{tnum-pi}}{Funkce pro tnum Ludolfova čísla}
(\textcolor{funkcionalni}{defun} \textcolor{pojmenovan}{tnum-pi} ()
  (\textcolor{funkcionalni}{lambda} (eps)
    (\textcolor{vedlejsi}{let} ((result 0))
      (\textcolor{funkcionalni}{loop} \textcolor{obarvi}{for} i \textcolor{obarvi}{from} 0
            \textcolor{obarvi}{for} /16powi = (\textcolor{matematicke}{expt} 16 (\textcolor{matematicke}{-} i)) \textcolor{obarvi}{and} 8i = (\textcolor{matematicke}{*} 8 i)
            \textcolor{obarvi}{do} (\textcolor{vedlejsi}{incf} result 
                     (\textcolor{matematicke}{*} /16powi
                        (\textcolor{matematicke}{-} (\textcolor{matematicke}{/} 4 (\textcolor{matematicke}{+} 8i 1))
                           (\textcolor{matematicke}{/} 2 (\textcolor{matematicke}{+} 8i 4))
                           (\textcolor{matematicke}{/} (\textcolor{matematicke}{+} 8i 5))
                           (\textcolor{matematicke}{/} (\textcolor{matematicke}{+} 8i 6)))))
            \textcolor{obarvi}{until} (\textcolor{matematicke}{<=} (\textcolor{matematicke}{/} /16powi 15) eps)
            \textcolor{obarvi}{finally} (\textcolor{funkcionalni}{return} result)))))
\end{lispcode}

\subsection{Přenásobování numem}

Racionální násobky tnumů umožní vyčíslení například čísla $2\pi{/3}$.

\begin{theorem}[O přenásobení tnumu racionální konstantou]
Nechť $c\in\mathbb{Q}$, $\tnum{x}\in\Tnum{x},x\in\mathbb{R}$ a funkce $\tnum{}(\tnum{x},c)$ má předpis
\begin{equation}
\tnum{}(\tnum{x},c)(\varepsilon)=\begin{cases}c*\tnum{x}\left(\frac{\varepsilon}{|c|}\right) & \text{pro~}c\neq{0},\\0&\text{pro~}c={0},\end{cases}
\end{equation}
pak $\tnum{}(\tnum{x},c)\in\Tnum{x*c}$.
\begin{proof}
Pokud přenásobíme jakékoli číslo nulou, je výsledkem nula (agresivní prvek vůči násobení). Znění věty pro nenulovou konstantu dokážeme tak, že z předpokladu $|\txe -x|\leq\varepsilon$ odvodíme $|c*\tnum{x}(\frac{\varepsilon}{|c|})-c*x|\leq\varepsilon$. Protože pracujeme s nerovnicemi, budeme postupovat dvěmi větvemi -- pro $c$ kladné a záporné.

Z definice tnumu předpokládáme
\begin{equation}
|\txe-x|\leq\varepsilon,
\end{equation}
po přenásobení kladným $c>0$ dostáváme
\begin{equation}
c*|\txe-x|\leq c*\varepsilon,
\end{equation}
protože je ale $c$ kladné, lze jím absolutní hodnotu roznásobit
\begin{equation}
|c*\txe-c*x|\leq c*\varepsilon,
\end{equation}
a nyní na pravé straně potřebujeme dostat přesnost $\varepsilon$. Protože dle definice platí $|\tnum{y}(\varepsilon)-y|\leq\varepsilon$, platí jistě i $|\tnum{y}(\frac{\varepsilon}{c})-y|\leq\frac{\varepsilon}{c}$, pak ale musí platit i
\begin{equation}
\left|c*\tnum{x}\left(\frac{\varepsilon}{c}\right)-c*x\right|\leq \varepsilon.
\end{equation}
Pro zápornou konstantu je důkaz podobný, a protože jako argument tnumů bereme kladné číslo, přibývá v děliteli v argumentu tnumu ještě absolutní hodnota. Dohromady pak získáváme
\begin{equation}
\left|c*\tnum{x}\left(\frac{\varepsilon}{|c|}\right)-c*x\right|\leq\varepsilon,
\end{equation}
což jsme chtěli odvodit.
\end{proof}
\end{theorem}

\begin{lispcode}{\texttt{tnum*num}}{Funkce přenásobující tnum racionální konstantou}
(\textcolor{funkcionalni}{defun} \textcolor{pojmenovan}{tnum*num} (tnum num)
  (\textcolor{vedlejsi}{let} ((rat_num (\textcolor{matematicke}{rationalize} num)))
    (\textcolor{funkcionalni}{lambda} (eps)
      (\textcolor{funkcionalni}{if} (\textcolor{funkcionalni}{zerop} num)
          (\textcolor{moje}{num-to-tnum} 0)
        (\textcolor{matematicke}{*} (\textcolor{moje}{tnum-to-num} tnum (\textcolor{matematicke}{/} eps (\textcolor{matematicke}{abs} rat_num))) rat_num))))) 
\end{lispcode}

\begin{consequence}[Opačný tnum]\label{dusl:negace_tnumu}
Nechť $\tnum{x}\in\Tnum{x}, x\in\mathbb{R}$ a funkce $\tnum{}(\tnum{x})$ má předpis
\begin{equation}
\tnum{}(\tnum{x})(\varepsilon)=-\txe,
\end{equation}
pak $\tnum{}(\tnum{x})\in\Tnum{-x}$.
\begin{proof}
Protože $-x = (-1)x$ a $|-1|=1$, pak z přechozí věty dostáváme $-\txe=(-1)\txe=(-1)\tnum{x}(\frac{\varepsilon}{1})=(-1)\tnum{x}(\frac{\varepsilon}{|-1|})=\tnum{(-1)x}(\varepsilon)=\tnum{-x}(\varepsilon)\in\Tnum{-x}$.
\end{proof}
\end{consequence}

\begin{lispcode}{\texttt{-tnum}}{Funkce pro opačný tnum}
(\textcolor{funkcionalni}{defun} \textcolor{pojmenovan}{-tnum} (tnum)
  (\textcolor{funkcionalni}{lambda} (eps)
    (\textcolor{matematicke}{-} (\textcolor{moje}{tnum-to-num} tnum eps))))
\end{lispcode}

