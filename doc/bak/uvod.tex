Tento dokument čtenáři přibližuje myšlenku, že funkcionální programování umí důstojně zrcadlit matematický svět. Ukážu to na příkladu výpočtů reálných čísel. Jako funkcionální jazyk byl zvolem Common Lisp (CL, Lisp) pro jeho syntaktickou jednoduchost, dynamičnost a lenost. Každý Lispový kód následuje vždy za matematickým výrazem a ve stejném znění ho převádí. Lisp je pro nás tedy jen nástrojem, jak přivést matematiku k životu a hlavní těžiště poznání je vzniknuvší matematická teorie, která není specifická pro jednotlivý programovací jazyk, ale když ji někdo přepíše do jiného (doporučuji funkcionálního) jazyka, měl by dostat stejný systém. 

Nejprve nás čeká teoretický úvod ohledně toho, co vlastně čísla jsou a jak se rozvrstvují podle vlastností na obory. Také se podíváme, jak se s čísly operuje a podíváme se na pojem zobrazení a jeho dva důležité typy. Poté přejdeme k těžišti této práce a zamyslíme se, co znamená přesná reprezentace čísla pro různé jeho podoby. Seznámíme se s několika existujícími knihovnami.

Ve druhé části na matematickém základě postavíme knihovnu \texttt{tnums} přesně vyčíslující reálná čísla. Nejprve si od Lispu vypůjčíme jeho racionální čísla a přidáme Ludolfovo číslo. Poté začneme čísla kombinovat operacemi a měnit jejich funkcemi, přibude Eulerovo číslo. Jak jsem již napsal výše, kód v této části bude jen syntaktickým přepisem předcházejícího matematického výrazu. To  mimo jiné znamená, že práci může číst i neprogramátor, přeskočí jen kódy a přesto mu bude práce dávat smysl.

V poslední části potom přidáme nějakou praktickou zkušenost a vymezíme pojem uživatelské funkce. Podíváme se, jak takové funkce lze přidávat a zjistíme, že některé mimoděk vznikly už při programování. Také doplníme poslední konstantu a to Zlatý řez. V poslední kapitole se podíváme na nápad napsat přesnou kalkulačku, poté jak urychlit práci knihovny \texttt{tnums} a nakonec na její bolavá místa.