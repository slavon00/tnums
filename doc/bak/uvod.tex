Funkcionální programování umožňuje důsledně popsat matematický svět, což v této práci bude ukázáno na příkladu výpočtů reálných čísel. Jako funkcionální jazyk byl zvolen Common Lisp (Lisp) pro jeho syntaktickou jednoduchost, dynamičnost a lenost. Každý Lispový kód následuje vždy za matematickým výrazem a ve stejném znění ho převádí. Lisp představuje nástroj, pomocí něhož realizujeme matematické výpočty, hlavním těžištěm poznání je ale vybudovaná matematická teorie, která není specifická pro konkrétní programovací jazyk.

V první části je teoreticky popsáno, co čísla jsou a jak se rozvrstvují podle vlastností na obory. Také je zde ukázáno, jak se s čísly operuje a je představen pojem zobrazení a jeho dva důležité typy -- funkce a posloupnost. Je zde zavedena stěžejní představa, co znamená přesná reprezentace čísla pro jeho různé podoby a představeno několik existujících knihoven.

Ve druhé části práce je popsána konstrukce knihovny \texttt{tnums} přesně vyčíslující reálná čísla. Je využita existence racionálních čísel v Lispu a přidáno Ludolfovo číslo. Zavedená čísla jsou poté kombinována operacemi a měněna svými funkcemi, v návaznosti na to je zavedeno Eulerovo číslo.

V závěrečné části je uvedeno praktické užití řešení a vymezen pojem uživatelské funkce. Je zde ukázáno, jak lze takové funkce přidávat, a je zde demonstrováno, že některé vznikly mimoděk už při programování. Také doplníme poslední konstantu, a to Zlatý řez. V poslední kapitole je představena možnost napsání přesné kalkulačky, nápady na urychlení práce knihovny \texttt{tnums} a nakonec její nedostatky.

\section*{Značení}
\begin{tabular}{l|l}
$\square$ & konec důkazu \\
$\blacksquare$ & konec poznámky \\
modré pozadí & obrázek \\
nachové pozadí & tabulka \\
$(a,b)$& otevřený interval mezi $a$ a $b$ \\
$[a,b]$ & uzavřený interval mezi $a$ a $b$ \\
$\rightarrow$ & logická implikace nebo \uv{do} \\
$\leftrightarrow$ & logická ekvivalence \\
$\Leftrightarrow$ & \uv{právě tehdy, když} \\
$\cup$& množinové sjednocení \\
$\cap$ & množinový průnik \\
$|a|$ & absolutní hodnota čísla $a$ \\
:= & přiřazení \\
$\frac{d}{dx}f(x)$ & derivace funkce $f$ v bodě $x$, je-li jasná proměnná, pak jen $f^{'}(x)$.
\end{tabular}