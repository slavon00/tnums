\documentclass[
master=false,
field=inf,
encoding=utf8,
language=czech,
printversion=true,]{kidiplom}

\definecolor{funkcionalni}{RGB}{0,0,100}
\definecolor{pojmenovan}{RGB}{0,100,0}
\definecolor{vedlejsi}{RGB}{100,0,0}
\definecolor{matematicke}{RGB}{100,0,50}
\definecolor{moje}{RGB}{0,50,50}
\definecolor{lightlightgray}{gray}{.875}
\definecolor{lightpink}{RGB}{255,233,233}
\definecolor{obarvi}{RGB}{59,55,20}
\definecolor{lightlightblue}{RGB}{245,251,255}
\newcounter{lispcodecnt}

\setcounter{section}{-1}

\declaretheoremstyle[
headfont=\normalfont\bfseries,
notebraces={(}{)},
headpunct={:}]{mystyle}

\declaretheorem[name=Lispový kód, style=mystyle]{lispcodethm}

\newenvironment{lispcode}[2]
{\def\firstarg{#1}
\def\secondarg{#2}
\VerbatimEnvironment
\begin{mdframed}[backgroundcolor=lightlightgray]
\begin{Verbatim}[commandchars=\\\{\}, numbers=left]}
{\end{Verbatim}
\end{mdframed}
%\vspace{-.92\baselineskip}
\begin{center}
\noindent\begin{minipage}[c]{.9\linewidth}
\begin{lispcodethm}[\firstarg]
\textit{\secondarg}
\end{lispcodethm}
\end{minipage}
\end{center}}

\declaretheorem[name=Lispový test, style=mystyle]{lispcodetest}

\newenvironment{myfigure}[1]{\begin{figure}[#1]\begin{mdframed}[backgroundcolor=lightlightblue,innertopmargin=-2.5pt,innerbottommargin=2.5pt]}{\end{mdframed}\end{figure}}

\newenvironment{myremark}[1]{\begin{remark}[#1]}{\hfill$\blacksquare$\end{remark}}
\newenvironment{myremarkbez}[1]{\begin{remark}[#1]}{\end{remark}}

\newenvironment{lisptest}[2]
{\def\firstarg{#1}
\def\secondarg{#2}
\VerbatimEnvironment
\begin{mdframed}[backgroundcolor=lightlightgray]
\begin{Verbatim}[commandchars=\\\{\}, numbers=left]}
{\end{Verbatim}
\end{mdframed}
%\vspace{-.92\baselineskip}
\begin{center}
\noindent\begin{minipage}[c]{.9\linewidth}
\begin{lispcodetest}[\firstarg]\textit{\secondarg}
\end{lispcodetest}
\end{minipage}
\end{center}}

\title{Přesné výpočty s reálnými čísly}
\title[english]{Precise computation of real numbers}

\author{Ondřej Slavík}
\supervisor{doc. RNDr. Michal Krupka, Ph.D.}

\annotation{Fenomén vyčíslitelnosti reálných čísel provází každého informatika, který se snaží používat počítač k počítání. Jakmile se totiž musíme spolehnout na výpočty s čísly uloženými jako hodnoty, narážíme na limity přesnosti a rozsahu takto reprezentovaných čísel. Řešením není zpřesňování pomocí vyšší dotace paměťového prostoru (např. binary32 $\rightarrow$ binary64) a související změna architektury systému, nýbrž fundamentální změna v přístupu k vyčíslení reálných čísel. Tato práce dává návod, jak takovýto přístup přijmout, a přináší knihovnu, která umožňuje základní výpočty a vyčíslení reálných čísel.}

\annotation[english]{Every computer scientist who tries to use a computer to compute encounters the phenomenon of real numbers' computability. Once we have to rely on calculations with numbers stored as values, we come across limits of precision and range of thus represented numbers. The solution is not to refine using a higher memory space allocation (eg binary32 $\rightarrow$ binary64) and the related change in system architecture, but fundamental change in the approach to computation of real numbers. This work gives direction of how to adopt such an approach, and brings a library that allows the basic calculations and enumerations of real numbers.}

\keywords{reálná čísla, funkce, Lisp, líné vyhodnocování, libovolná přesnost, rekurzivní čísla}
\keywords[english]{real numbers, functions, Lisp, lazy evaluation, arbitrary precision, recursive numbers}

\thanks{Mockrát děkuji doc. RNDr. Michalu Krupkovi, Ph.D. za vedení práce a rodině za podporu.}

\makeatletter
\AtBeginDocument{%
\@ifpackageloaded{amsthm}%
 {%
  \renewrobustcmd\mdf@patchamsthm{%
   \chardef\kludge@catcode@hyphen=\catcode`\-
   \catcode`\-=12
   \let\mdf@deferred@thm@head\deferred@thm@head
   \pretocmd{\deferred@thm@head}{\@inlabelfalse}%
      {\mdf@PackageInfo{mdframed detected package amsthm ^^J%
                        changed the theorem header of amsthm\MessageBreak}%
      }{%
       \mdf@PackageError{mdframed detected package amsthm ^^J%
                         changed the theorem header of amsthm
                         failed\MessageBreak}%
       }%
   \catcode`\-=\kludge@catcode@hyphen
     }%
 }{}%
}
\makeatother

\newcommand{\mypart}{\newpage\part}
\newcommand{\tnum}[1]{\mathfrak{t}^{#1}}
\newcommand{\Tnum}[1]{\mathcal{T}^{#1}}
\newcommand{\txe}{\tnum{x}(\varepsilon)}

\begin{document}
\setcounter{tocdepth}{3}
\maketitle
\section{Úvod}
Funkcionální programování umožňuje důsledně popsat matematický svět, což v této práci bude ukázáno na příkladu výpočtů reálných čísel. Jako funkcionální jazyk byl zvolen Common Lisp (Lisp) pro jeho syntaktickou jednoduchost, dynamičnost a lenost. Každý Lispový kód následuje vždy za matematickým výrazem a ve stejném znění ho převádí. Lisp představuje nástroj, pomocí něhož realizujeme matematické výpočty, hlavním těžištěm poznání je ale vybudovaná matematická teorie, která není specifická pro konkrétní programovací jazyk.

V první části je teoreticky popsáno, co čísla jsou a jak se rozvrstvují podle vlastností na obory. Také je zde ukázáno, jak se s čísly operuje a je představen pojem zobrazení a jeho dva důležité typy -- funkce a posloupnost. Je zde zavedena stěžejní představa, co znamená přesná reprezentace čísla pro jeho různé podoby a představeno několik existujících knihoven.

Ve druhé části práce je popsána konstrukce knihovny \texttt{tnums} přesně vyčíslující reálná čísla. Je využita existence racionálních čísel v Lispu a přidáno Ludolfovo číslo. Zavedená čísla jsou poté kombinována operacemi a měněna svými funkcemi, v návaznosti na to je zavedeno Eulerovo číslo.

V závěrečné části je uvedeno praktické užití řešení a vymezen pojem uživatelské funkce. Je zde ukázáno, jak lze takové funkce přidávat, a je zde demonstrováno, že některé vznikly mimoděk už při programování. Také doplníme poslední konstantu, a to Zlatý řez. V poslední kapitole je představena možnost napsání přesné kalkulačky, nápady na urychlení práce knihovny \texttt{tnums} a nakonec její nedostatky.

\section*{Značení}
\begin{tabular}{l|l}
$\square$ & konec důkazu \\
$\blacksquare$ & konec poznámky \\
modré pozadí & obrázek \\
nachové pozadí & tabulka \\
$(a,b)$& otevřený interval mezi $a$ a $b$ \\
$[a,b]$ & uzavřený interval mezi $a$ a $b$ \\
$\rightarrow$ & logická implikace nebo \uv{do} \\
$\leftrightarrow$ & logická ekvivalence \\
$\Leftrightarrow$ & \uv{právě tehdy, když} \\
$\cup$& množinové sjednocení \\
$\cap$ & množinový průnik \\
$|a|$ & absolutní hodnota čísla $a$ \\
:= & přiřazení \\
$\frac{d}{dx}f(x)$ & derivace funkce $f$ v bodě $x$, je-li jasná proměnná, pak jen $f^{'}(x)$.
\end{tabular}
\mypart{Teorie}
V teoretické části je představena axiomatická teorie množin a naznačeno, jak se z ní vytváří čísla. Je demonstrováno, že čísla jsou množinami. Poté jsou stručně představeny matematické operace a matematické funkce. Dále vyplyne, že jakékoli číslo lze vyjádřit jako funkci a že funkce je také množina. Závěr teoretické části je zaměřen na problém uložení čísla v paměti počítače, která je fyzicky konečná. Nejprve je rozebrán v čistě teoretické rovině a poté je diskutováno řešení v podobě reálně existujících knihoven.
\section{Čísla}\label{kap:cisla}
\input{cisla.tex}
\clearpage
\section{Čísla v počítači}
\input{cisla_v_pocitaci.tex}
\mypart{Implementace}
V implementační části aplikujeme teorii a dáme vzniknout knihovně \texttt{tnums}. Všechna čísla i manipulaci s nimi lze vyjádřit jako funkce, přesně toto naprogramujeme. Situace nás vede k užití funkcionálního programovacího jazyka, zvolen byl Lisp. Budou ukázány převody, konstanty, operace a funkce.
\section{Tnumy}
Svoje struktury jsem zvolil jako funkce dvou proměnných -- reprezentovaného čísla a přesnosti. Podíváme se, jak se dají matematicky definovat, po Lispovsku vymodelovat a na závěr dokážu několik tvrzení, aby čtenář pochopil, jak se s tnumy pracuje. Nebude chybět prvních několik kódů, i když nejvíce programování bude spíše ke konci této části.

\subsection{Vztah čísel a tnumů}
Funkce, kterými budu modelovat rekurzivní čísla a implementovat práci s nimi pomocí jazyka Lisp, budu nazývat \textit{True Numbers}, zkráceně \textit{tnums}. Vystihuje to jejich podstatu a cíl - budou ve výsledku opravdovější a přesnější, než ostatní čísla, která by měla mít nekonečný rozvoj, ale jsou uložena jako hodnoty – často jako nějaká plovoucí čísla (jakási Nil Numbers). Vzniknuvší knihovna se pak jmenuje \texttt{tnums}.

\begin{definition}[Tnum]\label{def:tnum}
Funkce $\tnum{x}:(0,1)\rightarrow\mathbb{Q}$, která pro všechna $\varepsilon\in (0,1)$ vrací hodnotu $\txe$ splňující nerovnost
\begin{equation}\label{rov:def:tnum}
|\txe - x |\leq \varepsilon
\end{equation}
se nazývá \textit{tnum} čísla $x$.

Množinu všech tnumů čísla $x$ značíme $\Tnum{x}=\{\tnum{x}|(\forall x \in\mathbb{R})(\forall\varepsilon\in (0,1)):|\txe-x|\leq\varepsilon\}$, množinu všech tnumů pak symbolem $\mathfrak{T}$.
\end{definition}

Tnum $\tnum{x}$ je struktura představující rekurzivní číslo $x$. Místo vztahu \ref{rov:def:tnum} lze psát $\txe\in[x-\varepsilon,x+\varepsilon]$. Při výpočtu hodnoty tnumu nejprve tento tnum vytvoříme (toto bude výpočetně rychlé) a jakmile bude vytvořený, necháme ho vyčíslit (zavolat s přesností) a toto může být na dlouho. Vyčíslení tedy odkládáme na nejpozdější možnou dobu. Mluvíme o líném vyhodnocování.

Tnumy přesných čísel lze vyčíslit s dokonalou přesností. Využiji co nejvíce z~přesnosti, kterou nabízí Lisp a ten přesně reprezentuje všechna racionální čísla. To je ve shodě s představou o vyčíslení rekurzivního čísla. Pomocí $\tnum{r}(\varepsilon)$ tedy získáváme $q$ z nerovnice \ref{rov:rac_u_real}. Číslu, jak ho chápe Lisp říkám nadále \textit{num} (number).

\begin{lemma}[O numu jako tnumu]\label{lem:num-to-tnum}
Pro všechna $x\in\mathbb{R}$ a všechna $\varepsilon \in (0,1)$ platí: číslo $\txe$ lze nahradit číslem $x$.
\begin{proof}
Z nerovnosti \ref{rov:def:tnum} získáváme $|\txe - x |\leq \varepsilon$. Po dosazení $\txe := x$ pak $|x - x | = 0 \leq \varepsilon$, což platí pro všechna myslitelná $x$ i $\varepsilon$.
\end{proof}
\end{lemma}

Nejpřesnější reprezentace čísla reprezentovaného tnumem reprezentujícím čí\-slo je toto číslo samotné. Proto je tedy vhodné co nejvíce takových čísel přenechat na reprezentaci Lispu a počítat jen s těmi, které nezvládne. Protože Lisp pracuje i se zlomky (typ \texttt{ratio}), nejvyšší obor čísel, který umí vracet s nulovou odchylkou jsou racionální čísla.

\begin{lispcode}{\texttt{num-to-tnum}}{Funkce převádějící číslo z interní reprezentace Lispu na tnum}
(\textcolor{funkcionalni}{defun} \textcolor{pojmenovan}{num-to-tnum} (num)
  (\textcolor{vedlejsi}{let} ((rat_num (\textcolor{matematicke}{rationalize} num)))
    (\textcolor{funkcionalni}{lambda} (eps) (\textcolor{vedlejsi}{declare} (\textcolor{vedlejsi}{ignore} eps))
      rat_num)))
\end{lispcode}

Převod opačným směrem je přímočarý. Chceme-li číslo $x$ s~přesností $\varepsilon$, stačí zavolat $\txe$. Přesnost musí být z $(0, 1)$, jiné číslo interpretujeme jako $10^{-|\varepsilon|}$.

\begin{lemma}[O převodu tnumu na num]
Pokud existuje funkce $\tnum{x}\in\Tnum{x}$, pak po zavolání s argumentem $\varepsilon$ vrací hodnotu $\txe$ splňující $(|\txe - x |\leq \varepsilon)$.
\begin{proof}
Plyne přímo z definice \ref{def:tnum}.
\end{proof}
\end{lemma}

\begin{lispcode}{\texttt{rat-expt}}{Funkce pro racionální umocňování}
(\textcolor{funkcionalni}{defun} \textcolor{pojmenovan}{rat-expt} (num exp)
  (\textcolor{matematicke}{rationalize} (\textcolor{matematicke}{expt} num exp)))
\end{lispcode}

\begin{lispcode}{\texttt{tnum-to-num}}{Funkce převádějící tnum na číslo}
(\textcolor{funkcionalni}{defun} \textcolor{pojmenovan}{tnum-to-num} (tnum eps)
  (\textcolor{funkcionalni}{when} (\textcolor{funkcionalni}{or} (\textcolor{matematicke}{>=} 0 eps) (\textcolor{matematicke}{<=} 1 eps))
    (\textcolor{vedlejsi}{setf} eps (\textcolor{moje}{rat-expt} 10 (\textcolor{matematicke}{-} (\textcolor{matematicke}{abs} eps)))))
  (\textcolor{funkcionalni}{funcall} tnum (\textcolor{matematicke}{rationalize} eps)))
\end{lispcode}

Zatímco tedy pro převod z čísla na tnum jsme toto mohli udělat pro všechna čísla, opačným směrem toto funguje pouze za předpokladu, že daný tnum existuje. V našem systému teď máme jen tnumy pro racionální čísla a umíme je převádět tam a zpět. V dalším textu tedy půjde hlavně o to zaplnit tuto mezeru a přinést existenci co nejvíce tnumů.

\subsection{Ludolfovo číslo}
Prvním iracionálním číslem, které do knihovny přidáme je číslo Ludolfovo.

\begin{definition}[Ludolfovo číslo \cite{piratio}]
Ludolfovým číslem myslíme poměr obvodu kružnice k jejímu průměru.
\end{definition}

Ludolfovo číslo je asi nejslavnější transcendentní konstanta a proto není divu, že pro její vyčíslení existuje bezpočet vzorců. Asi nejpřímější je Leibnizův vzorec, který vypočítává čtvrtinu Ludolfova čísla a plyne z Taylorovy řady funkce arctan v bodě 1. Pokud Ludolfovo číslo značím $\pi$, pak ho lze zapsat jako $\pi=4\sum_{n\in\mathbb{N}}\frac{(-1)^n}{2n+1}$ \cite{approxpi}, tato řada ale konverguje velmi pomalu. Já proto použiji aproximaci jinou. Tento vzorec se jmenuje BBP podle svých tvůrců (Bailey, Borwein, Plouffe) a je zapsán ve formě řady.

\begin{fact}[Ludolfovo číslo jako řada \cite{BBP}]
Nechť $\pi$ značí Ludolfovo číslo. Pak jej lze zapsat jako
\begin{equation}\label{rov:pi-rada}
\pi=\sum_{i\in\mathbb{N}}\frac{1}{16^i}\left(\frac{4}{8i+1}-\frac{2}{8i+4}-\frac{1}{8i+5}-\frac{1}{8i+6}\right).
\end{equation}
\end{fact}

Mám tedy řadu, která generuje konstantu, kterou chci přidat do \texttt{tnums}. Výraz $\left(\frac{4}{8i+1}-\frac{2}{8i+4}-\frac{1}{8i+5}-\frac{1}{8i+6}\right)$ je pro $i>0$ menší než jedna, proto se každý nenultý člen může zhora omezit $\frac{1}{16^i}$ a to je geometrická posloupnost, jejíž zbytek je dle faktu \ref{vet:o_zbytku_geometricke_rady} roven $\frac{1}{16^{i+1}}*\frac{16}{15}$, což je $\frac{1}{16^i*15}$. Platí tedy

\begin{equation}
\left|\pi - \sum_{i=0}^n\frac{1}{16^i}\left(\frac{4}{8i+1}-\frac{2}{8i+4}-\frac{1}{8i+5}-\frac{1}{8i+6}\right) \right| \leq \frac{1}{16^n*15}.
\end{equation}

\begin{consequence}[Tnum Ludolfova čísla]
Nechť $\tnum{}$ je funkce s předpisem $\tnum{}(\varepsilon)=$\uv{najdi nějaké $n$ tak, aby platilo $/(16^n15)\leq\varepsilon$ a poté vrať $n$-tý částečný součet řady ze vztahu \ref{rov:pi-rada}}, pak $\tnum{}\in\Tnum{\pi}$.
\end{consequence}

Kód vypadá trochu složitěji, ale není to nic jiného, než co bylo právě popsáno. Nižší čitelnost je zde vykoupena vyšší efektivitou a protože je vyčíslování $\pi$ jedna z nejdůležitějších funkcionalit, rozhodl jsem se ji zavést takto efektivně, ač na úkor čitelnosti.

\begin{lispcode}{\texttt{tnum-pi}}{Funkce na vytvoření tnumu Ludolfova čísla}
(\textcolor{funkcionalni}{defun} \textcolor{pojmenovan}{tnum-pi} ()
  (\textcolor{funkcionalni}{lambda} (eps)
    (\textcolor{vedlejsi}{let} ((/16pown 0) (result 0) (above 1))
      (\textcolor{funkcionalni}{loop} \textcolor{obarvi}{for} n \textcolor{obarvi}{from} 0
        \textcolor{obarvi}{until} (\textcolor{matematicke}{<=} above eps)
        \textcolor{obarvi}{do} (\textcolor{funkcionalni}{progn} 
          (\textcolor{vedlejsi}{setf} /16pown (\textcolor{moje}{rat-expt} 16 (\textcolor{matematicke}{-} n)))
          (\textcolor{vedlejsi}{incf} result
            (\textcolor{matematicke}{*} /16pown
              (\textcolor{matematicke}{-} (\textcolor{matematicke}{/} 4 (\textcolor{matematicke}{+} (\textcolor{matematicke}{*} 8 n) 1))
                (\textcolor{matematicke}{/} 2 (\textcolor{matematicke}{+} (\textcolor{matematicke}{*} 8 n) 4))
                (\textcolor{matematicke}{/} 1 (\textcolor{matematicke}{+} (\textcolor{matematicke}{*} 8 n) 5))
                (\textcolor{matematicke}{/} 1 (\textcolor{matematicke}{+} (\textcolor{matematicke}{*} 8 n) 6)))))
          (\textcolor{vedlejsi}{setf} above (\textcolor{matematicke}{/} /16pown 15)))
        \textcolor{obarvi}{finally} (\textcolor{funkcionalni}{return} result)))))
\end{lispcode}

\subsection{Přenásobování numem}
Posledním dílkem, který přidám v této kapitole je přenásobování tnumu konstantou. Když už máme všechna racionální čísla a Ludolfovo číslo, zvládneme pak i například $2\pi$ nebo $\frac{\pi}{-2}$.

\begin{theorem}[O přenásobení tnumu racionální konstantou]
Necht $c\in\mathbb{Q}$, $\tnum{x}\in\Tnum{x},x\in\mathbb{R}$ a funkce $\tnum{}$ má předpis
\begin{equation}
\tnum{}(\varepsilon)=\begin{cases}c*\tnum{x}\left(\frac{\varepsilon}{|c|}\right) & \text{pro~}c\not = 0,\\0&\text{jinak,}\end{cases}
\end{equation}
pak $\tnum{}\in\Tnum{x*c}$.
\begin{proof}
Pokud přenásobíme hodnotu tnumu nulou, je výsledkem nula, protože je to agresivní prvek vůči násobení. Znění věty pro nenulovou konstantu dokážeme tak, že z předpokladu $|\txe -x|\leq\varepsilon$ odvodíme $|c*\tnum{x}(\frac{\varepsilon}{|c|})-c*x|\leq\varepsilon$. Protože pracujeme s nerovnicemi, budeme v důkazu postupovat dvěmi větvemi -- pro $c$ kladné a záporné.

Z definice tnumu předpokládáme
\begin{equation}
|\txe-x|\leq\varepsilon,
\end{equation}
po přenásobení kladným $c>0$ dostáváme
\begin{equation}
c*|\txe-x|\leq c*\varepsilon,
\end{equation}
protože je ale $c$ kladné, můžu jím absolutní hodnotu roznásobit
\begin{equation}
|c*\txe-c*x|\leq c*\varepsilon,
\end{equation}
a protože na pravé straně potřebuji přesnost $\varepsilon$, v argumentu ji podělím $c$ a pak
\begin{equation}
\left|c*\tnum{x}\left(\frac{\varepsilon}{c}\right)-c*x\right|\leq \varepsilon.
\end{equation}
Pro zápornou konstantu je běh důkazu podobný a protože jako argument tnumů bereme kladné číslo, přibývá v děliteli v argumentu tnumu ještě absolutní hodnota. Dohromady pak získáváme
\begin{equation}
\left|c*\txe^{x}\left(\frac{\varepsilon}{|c|}\right)-c*x\right|\leq\varepsilon,
\end{equation}
což jsme chtěli ukázat.
\end{proof}
\end{theorem}

\begin{lispcode}{\texttt{tnum*num}}{Funkce přenásobující tnum racionální konstantou}
(\textcolor{funkcionalni}{defun} \textcolor{pojmenovan}{tnum*num} (tnum num)
  (\textcolor{vedlejsi}{let} ((rat_num (\textcolor{matematicke}{rationalize} num)))
    (\textcolor{funkcionalni}{lambda} (eps)
      (\textcolor{funkcionalni}{if} (\textcolor{funkcionalni}{zerop} num)
        (\textcolor{moje}{num-to-tnum} 0)
        (\textcolor{matematicke}{*} (\textcolor{moje}{tnum-to-num} tnum (\textcolor{matematicke}{/} eps (\textcolor{matematicke}{abs} rat_num))) rat_num)))))
\end{lispcode}

\begin{consequence}[Opačný tnum]\label{dusl:negace_tnumu}
Nechť $\tnum{x}\in\Tnum{x}, x\in\mathbb{R}$ a funkce $\tnum{}$ má předpis
\begin{equation}
\tnum{}(\varepsilon)=-\txe,
\end{equation}
pak $\tnum{}\in\Tnum{-x}$.
\begin{proof}
Protože $-x = (-1)x$ a $|-1|=1$, pak podle přechozí věty dostáváme $-\txe=(-1)\txe=(-1)\tnum{x}(\frac{\varepsilon}{1})=(-1)\tnum{x}(\frac{\varepsilon}{|-1|})=\tnum{(-1)x}(\varepsilon)=\tnum{-x}(\varepsilon)\in\Tnum{-x}$.
\end{proof}
\end{consequence}

\begin{lispcode}{\texttt{-tnum}}{Funkce pro opačný tnum}
(\textcolor{funkcionalni}{defun} \textcolor{pojmenovan}{-tnum} (tnum)
  (\textcolor{moje}{tnum*num} tnum -1))
\end{lispcode}


\section{Operace tnumů}
Matematické operace se neimplementují všechny stejně složitě. Zatímco aditivní vyřešíme relativně rychle, multiplikativní budou o úroveň těžší, tak mocninné v této kapitole ani nezvládneme a budeme na ně muset počkat až na konec další kapitoly. Začneme tedy operacemi sčítání a odčítání, pak se přesuneme k~násobení a dělení.

\subsection{Aditivní operace}
Součet je operace neomezeného počtu argumentů, pro žádný vrací nulu (neutrální prvek aditivní grupy \cite{RachALG1}), pro jeden vrací tento a pro více se pak jedná o postupné zvětšování výsledku o hodnoty argumentů. Rozdíl potom vyžaduje alespoň jeden argument, v případě zadání pouze tohoto se vrací tnum k němu opačný, v případě více pak součet prvního a opačného tnumu součtu ostatních.

\begin{theorem}[O součtu tnumů]
Mějme čísla $x_0, x_1, \ldots, x_n$ a jejich tnumy. Pak platí
\begin{equation}
\mathcal{T}^{\sum_{i=0}^nx_i}(\varepsilon)=\sum_{i=0}^n\mathcal{T}^{x_i}\left(\frac{\varepsilon}{n+1}\right)
\end{equation}

\begin{proof}
Předpokládejme $\mathcal{T}^{x_i}(\varepsilon) + \varepsilon \geq x_i$ a $\mathcal{T}^{x_i}(\varepsilon) - \varepsilon \leq x_i$ pro $i \in \{0, 1, \ldots , n\}$ a ukažme $\sum_{i=0}^n\mathcal{T}^{x_i}\left(\frac{\varepsilon}{n+1}\right) + \varepsilon \geq \sum_{i=0}^nx_i$ a $\sum_{i=0}^n\mathcal{T}^{x_i}\left(\frac{\varepsilon}{n+1}\right) - \varepsilon \leq \sum_{i=0}^nx_i$.

Z definice tnumu předpokládáme
\begin{equation}
\begin{aligned}
\mathcal{T}^{x_0}\left(\frac{\varepsilon}{n+1}\right)+\frac{\varepsilon}{n+1}\geq x_0 &\land \mathcal{T}^{x_0}\left(\frac{\varepsilon}{n+1}\right)-\frac{\varepsilon}{n+1}\leq x_0,\\
\mathcal{T}^{x_1}\left(\frac{\varepsilon}{n+1}\right)+\frac{\varepsilon}{n+1}\geq x_1 &\land \mathcal{T}^{x_1}\left(\frac{\varepsilon}{n+1}\right)-\frac{\varepsilon}{n+1}\leq x_1,\\
\vdots \\
\mathcal{T}^{x_n}\left(\frac{\varepsilon}{n+1}\right)+\frac{\varepsilon}{n+1}\geq x_n &\land \mathcal{T}^{x_n}\left(\frac{\varepsilon}{n+1}\right)-\frac{\varepsilon}{n+1}\leq x_n.\\
\end{aligned}
\end{equation}
Sečtěme teď všechny výrazy a získáváme
\begin{equation}
\sum_{i=0}^n\mathcal{T}^{x_i}\left(\frac{\varepsilon}{n+1}\right) + \sum_{i=0}^n\frac{\varepsilon}{n+1} \geq \sum_{i=0}^nx_i \land \sum_{i=0}^n\mathcal{T}^{x_i}\left(\frac{\varepsilon}{n+1}\right) - \sum_{i=0}^n\frac{\varepsilon}{n+1} \leq \sum_{i=0}^nx_i,
\end{equation}
dále $\sum_{i=0}^n\frac{\varepsilon}{n+1} = \varepsilon$, takže
\begin{equation}
\sum_{i=0}^n\mathcal{T}^{x_i}\left(\frac{\varepsilon}{n+1}\right) + \varepsilon \geq \sum_{i=0}^nx_i \land \sum_{i=0}^n\mathcal{T}^{x_i}\left(\frac{\varepsilon}{n+1}\right) - \varepsilon \leq \sum_{i=0}^nx_i,
\end{equation}
výraz $\sum_{i=0}^n\mathcal{T}^{x_i}\left(\frac{\varepsilon}{n+1}\right)$ je tedy ekvivalentní s $\mathcal{T}^{\sum_{i=0}^nx_i}(\varepsilon)$.
\end{proof}
\end{theorem}

V našem chápání nevracíme přímo čísla, ale tnumy, takže pro prázdný argument místo nuly její tnum -- $\mathcal{T}^0$.

\begin{lispcode}{\texttt{tnum+}}{Funkce na součet tnumů}
(\textcolor{funkcionalni}{defun} \textcolor{pojmenovan}{tnum+} (&rest tnums)
  (\textcolor{funkcionalni}{if} (\textcolor{funkcionalni}{null} tnums)
    (\textcolor{moje}{num-to-tnum} 0)
    (\textcolor{funkcionalni}{lambda} (eps)
      (\textcolor{vedlejsi}{let} ((new-eps (\textcolor{matematicke}{/} eps (\textcolor{funkcionalni}{list-length} tnums))))
        (\textcolor{funkcionalni}{apply} \textquotesingle\textcolor{moje}{+} 
          (\textcolor{funkcionalni}{mapcar} (\textcolor{funkcionalni}{lambda} (tnum) (\textcolor{moje}{tnum-to-num} tnum new-eps))
            tnums))))))
\end{lispcode}

Odčítání využívá právě dokázané věty a také důsledku \ref{dusl:negace_tnumu}.
\begin{fact}[Rozdíl tnumů]
Mějme čísla $x_{-1}, x_0, \ldots, x_n$, a jejich tnumy. Pak platí
\begin{equation}
\mathcal{T}^{x_{-1}-x_0-\ldots-x_n}(\varepsilon)=\mathcal{T}^{x_{-1}-\sum_{i=0}^nx_n}(\varepsilon)=\mathcal{T}^{x_{-1}+\left(-\sum_{i=0}^nx_n\right)}(\varepsilon)
\end{equation}
\end{fact}

Kód odpovídá Lispovskému \texttt{-}, tedy pro jeden argument vrací opačný tnum onoho a pro více argumentů vrací jejich rozdíl.

\begin{lispcode}{\texttt{tnum-}}{Funkce pro rozdíl tnumů, případně opačný tnum}
(\textcolor{funkcionalni}{defun} \textcolor{pojmenovan}{tnum-} (tnum1 &rest tnums)
  (\textcolor{funkcionalni}{if} (\textcolor{funkcionalni}{null} tnums)
    (\textcolor{moje}{-tnum} tnum1)
    (\textcolor{moje}{tnum+} tnum1 (\textcolor{moje}{-tnum} (\textcolor{funkcionalni}{apply} \textquotesingle\textcolor{moje}{tnum+} tnums)))))
\end{lispcode}

\subsection{Multiplikativní operace}
Jako první multiplikativní operaci představím převrácení hodnoty tnumu. Je to podobná operace jako opačný tnum, jen jde o jinou inverzi.

Protože převrácení pracuje pouze s nenulovými čísly, přidáme do našeho aparátu ještě práci s tnumy nenulových čísel.

\begin{definition}[Nenulový tnum]
Tnum $\mathcal{T}$, který nikdy nenabývá nulové hodnoty, neboli $(\forall \varepsilon \in (0,1))(\mathcal{T}(\varepsilon) \neq 0)$ budeme nazývat \textit{nenulový tnum} a budeme ho značit $\mathcal{T}_\emptyset$.
\end{definition}

Čísla, která nenabývají nuly tedy budeme moci reprezentovat nenulovými tnumy.

\begin{definition}[Bezpečné epsilon]\label{def:bezpecne_epsilon}
Pro libovolné $\varepsilon$ a tnum $\mathcal{T}^x$, kde $x\neq 0$ uvažujeme \uv{funkci} $\varepsilon_\emptyset : \mathfrak{T}\times(0,1)\to(0,1)$ tak, aby
\begin{enumerate}
\item{$0<\varepsilon_\emptyset(\mathcal{T}^x, \varepsilon)\leq\varepsilon$},
\item{$\mathcal{T}^x(\varepsilon_\emptyset(\mathcal{T}^x, \varepsilon)) \neq 0$} a
\item{$|\mathcal{T}^x(\varepsilon_\emptyset(\mathcal{T}^x, \varepsilon))| > \varepsilon_\emptyset(\mathcal{T}^x,\varepsilon)$}.
\end{enumerate}
a její hodnotu nazýváme bezpečným epsilonem.
\end{definition}

\begin{lemma}[O nenulovém tnumu nenulového čísla]\label{vet:nenul}
Tnum nenulového čísla lze vyčíslit nenulově, čili $(\forall x \neq 0)((\exists \mathcal{T}^x) \to (\exists \mathcal{T}^x_\emptyset))$. 
\begin{proof}
Vezměme za $\mathcal{T}^x_\emptyset$ tnum, který místo $\varepsilon$ dosadí $\varepsilon_\emptyset$, neboli $\mathcal{T}^x_\emptyset(\varepsilon) := \mathcal{T}^x(\varepsilon_\emptyset(\mathcal{T}^x, \varepsilon))$. Pak díky podmínce $1$ v definici \ref{def:bezpecne_epsilon} se přesnost nemůže zhoršit a tudíž vyčíslení proběhne v pořádku. Dále díky bodu $2$ ve stejné definici bude vyčíslení nenulové, takže se jedná o nenulový tnum.
\end{proof} 
\end{lemma}

Funkce pro vracení bezpečného epsilonu musí kvůli kontrole nenulovosti vypočítat i num zadaného tnumu a musí také vracet nové epsilon. Aby se tnum nevyčísloval vícekrát, když už jeho hodnotu známe, vrací funkce i tento num.

\begin{lispcode}{\texttt{get-nonzero-num+eps}}{Funkce pro nalezení přesnosti, při které nebude po aplikaci tnumu nulový výsledek a následné vrácení výsledku i epsilonu}
(\textcolor{funkcionalni}{defun} \textcolor{pojmenovan}{get-nonzero-num+eps} (tnum eps)
  (\textcolor{vedlejsi}{let} ((num (\textcolor{moje}{tnum-to-num} tnum eps)))
    (\textcolor{funkcionalni}{if} (\textcolor{funkcionalni}{and} (\textcolor{funkcionalni}{zerop} num) (\textcolor{matematicke}{<=} (\textcolor{matematicke}{abs} num) eps))
      (\textcolor{moje}{get-nonzero-num+eps} tnum (\textcolor{matematicke}{/} eps 10))
      (\textcolor{matematicke}{values} num eps))))
\end{lispcode}

\begin{theorem}[O převráceném tnumu]\label{hyp:prevraceni_tnumu}
Mějme $x \neq 0$ a jeho tnum. Pak platí
\begin{equation}
\mathcal{T}^{x^{-1}}(\varepsilon)=\left[\mathcal{T}^x(\varepsilon_\emptyset(\mathcal{T}^x, (\varepsilon*|\mathcal{T}^x(\varepsilon_\emptyset(\mathcal{T}^x, \varepsilon))|*(|\mathcal{T}^x(\varepsilon_\emptyset(\mathcal{T}^x, \varepsilon))|-\varepsilon_\emptyset(\mathcal{T}^x, \varepsilon)))))\right]^{-1}
\end{equation}
\begin{proof}

Podle rovnice \ref{rov:def:tnum} platí 
\begin{equation}
\left|\mathcal{T}^x(\varepsilon)-x\right|\leq \varepsilon,
\end{equation}
což lze díky lemmatu \ref{vet:nenul} a předpokladu nenulovosti $x$ přepsat na
\begin{equation}
\left|\mathcal{T}^x(\varepsilon_\emptyset(\mathcal{T}^x, \varepsilon))-x\right|\leq \varepsilon,
\end{equation}
díky absolutní hodnotě pak platí
\begin{equation}
\left| x-\mathcal{T}^x(\varepsilon_\emptyset(\mathcal{T}^x, \varepsilon))\right|\leq \varepsilon.
\end{equation}
Nerovnici vydělíme kladným číslem $|\mathcal{T}^x(\varepsilon_\emptyset(\mathcal{T}^x, \varepsilon))*x|$
\begin{equation}
\frac{\left| x-\mathcal{T}^x(\varepsilon_\emptyset(\mathcal{T}^x, \varepsilon))\right|}{|\mathcal{T}^x(\varepsilon_\emptyset(\mathcal{T}^x, \varepsilon))*x|}\leq \frac{\varepsilon}{|\mathcal{T}^x(\varepsilon_\emptyset(\mathcal{T}^x, \varepsilon))*x|}
\end{equation}
a protože $|a|*|b|=|a*b|$, po dvojí aplikaci platí
\begin{equation}
\left|\frac{ x-\mathcal{T}^x(\varepsilon_\emptyset(\mathcal{T}^x, \varepsilon))}{\mathcal{T}^x(\varepsilon_\emptyset(\mathcal{T}^x, \varepsilon))*x}\right|\leq \frac{\varepsilon}{|\mathcal{T}^x(\varepsilon_\emptyset(\mathcal{T}^x, \varepsilon))|*|x|}
\end{equation}
a po roztržení levého výrazu na rozdílné jmenovatele dostáváme
\begin{equation}
\left|\frac{1}{\mathcal{T}^x(\varepsilon_\emptyset(\mathcal{T}^x, \varepsilon))}-\frac{1}{x}\right|\leq \frac{\varepsilon}{|\mathcal{T}^x(\varepsilon_\emptyset(\mathcal{T}^x, \varepsilon))|*|x|}.
\end{equation}
Dále díky předpokladu $3$ z definice \ref{def:bezpecne_epsilon} $|x|\geq|\mathcal{T}^x(\varepsilon_\emptyset(\mathcal{T}^x, \varepsilon))|-\varepsilon_\emptyset(\mathcal{T}^x, \varepsilon)$ a proto
\begin{equation}
\left|\frac{1}{\mathcal{T}^x(\varepsilon_\emptyset(\mathcal{T}^x, \varepsilon))}-\frac{1}{x}\right|\leq \frac{\varepsilon}{|\mathcal{T}^x(\varepsilon_\emptyset(\mathcal{T}^x, \varepsilon))|*(|\mathcal{T}^x(\varepsilon_\emptyset(\mathcal{T}^x, \varepsilon))|-\varepsilon_\emptyset(\mathcal{T}^x, \varepsilon))},
\end{equation}
takže po úpravě přesnosti dostáváme
\begin{equation}
\left|\frac{1}{\mathcal{T}^x(\varepsilon_\emptyset(\mathcal{T}^x, (\varepsilon*|\mathcal{T}^x(\varepsilon_\emptyset(\mathcal{T}^x, \varepsilon))|*(|\mathcal{T}^x(\varepsilon_\emptyset(\mathcal{T}^x, \varepsilon))|-\varepsilon_\emptyset(\mathcal{T}^x, \varepsilon)))))}-\frac{1}{x}\right|\leq \varepsilon.
\end{equation}
\end{proof}
\end{theorem}

\begin{lispcode}{\texttt{/tnum}}{Funkce pro převracení hodnoty tnumu. Pokud je nové epsilon větší nebo stejné, vrací se již vypočtený tnum.}
(\textcolor{funkcionalni}{defun} \textcolor{pojmenovan}{/tnum} (tnum)
  (\textcolor{funkcionalni}{lambda} (eps)
    (\textcolor{matematicke}{multiple-value-bind} (num eps0)
      (\textcolor{moje}{get-nonzero-num+eps} tnum eps)
      (\textcolor{vedlejsi}{let*} ((absnum (\textcolor{matematicke}{abs} num))
          (neweps (\textcolor{matematicke}{*} eps absnum (\textcolor{matematicke}{-} absnum eps0))))
        (\textcolor{matematicke}{/} (\textcolor{funkcionalni}{if} (\textcolor{matematicke}{>=} neweps eps) num
          (\textcolor{moje}{get-nonzero-num+eps} tnum neweps)))))))
\end{lispcode}

Násobení bere libovolně mnoho argumentů. Pro žádný vrátí jedničku (jednotka v multiplikativní grupě \cite{RachALG1}), pro jeden vrátí tento a pro více pak jejich součin.

\begin{theorem}[O součinu dvou tnumů]\label{vet:soucin_dvou_tnumu}
Pro nenulová čísla $x$, $y$ a jejich tnumy platí
\begin{equation}
\mathcal{T}^{x*y}(\varepsilon)=\mathcal{T}^x\left(\frac{\varepsilon}{2(|\mathcal{T}^y(\varepsilon)| + \varepsilon + 1)}\right)*\mathcal{T}^y\left(\frac{\varepsilon}{2(|\mathcal{T}^x(\varepsilon)| + \varepsilon + 1)}\right)
\end{equation}
\begin{proof}
Nejprve si dokážeme dvě nerovnice, které posléze použijeme v těle důkazu.
První nerovnicí je 
\begin{equation}\label{duk:soucin_prvni}
\frac{|y|}{|\mathcal{T}^y(\varepsilon)| + \varepsilon + 1}\leq 1.
\end{equation}
To je ekvivalentní s
\begin{equation}
|y|\leq |\mathcal{T}^y(\varepsilon)|+\varepsilon+1
\end{equation}
a protože $\mathcal{T}^y(\varepsilon)+\varepsilon\geq y$, platí
\begin{equation}
|y|\leq |y|+1,
\end{equation}
což je jistě pravda. 

Druhá nerovnice je
\begin{equation}\label{duk:soucin_druha}
\frac{|\mathcal{T}^x(\frac{\varepsilon}{2(|\mathcal{T}^y(\varepsilon)|+\varepsilon+1|)})|}{|\mathcal{T}^x(\varepsilon)|+\varepsilon+1}\leq 1.
\end{equation}
Ekvivalentní zápis je
\begin{equation}
|\mathcal{T}^x(\frac{\varepsilon}{2(|\mathcal{T}^y(\varepsilon)|+\varepsilon+1|)})|\leq |\mathcal{T}^x(\varepsilon)|+\varepsilon+1
\end{equation}
a protože $\mathcal{T}^x(\varepsilon)+\varepsilon\geq x$, platí
\begin{equation}
|\mathcal{T}^x(\frac{\varepsilon}{2(|\mathcal{T}^y(\varepsilon)|+\varepsilon+1|)})|\leq |x|+1
\end{equation}
a stejným postupem rozepíšeme levou stranu, takže dostáváme
\begin{equation}
|x|+|\frac{\varepsilon}{2(|\mathcal{T}^x(\varepsilon)|+\varepsilon+1)}|\leq |x|+1
\end{equation}
a po odečtení $|x|$ zbyde
\begin{equation}
|\frac{\varepsilon}{2(|\mathcal{T}^x(\varepsilon)|+\varepsilon+1)}|\leq 1
\end{equation}
což ale platí, protože
\begin{equation}
|\mathcal{T}^x(\varepsilon)+\varepsilon\geq 0.
\end{equation}

Nyní konečně přejděme k důkazu věty. Aby věta platila, musíme dokázat
\begin{equation}
\left| \mathcal{T}^x\left(\frac{\varepsilon}{2(|\mathcal{T}^y(\varepsilon)|+\varepsilon +1)}\right) *\mathcal{T}^y\left(\frac{\varepsilon}{2(|\mathcal{T}^x(\varepsilon)|+\varepsilon +1)}\right) -xy \right|\leq\varepsilon.
\end{equation}
Rozepíšeme proto levou stranu
\begin{equation}
\left| \mathcal{T}^x\left(\frac{\varepsilon}{2(|\mathcal{T}^y(\varepsilon)|+\varepsilon +1)}\right) *\mathcal{T}^y\left(\frac{\varepsilon}{2(|\mathcal{T}^x(\varepsilon)|+\varepsilon +1)}\right) -xy \right| =
\end{equation}
přičtením a odečtením členu $\mathcal{T}^x\left(\frac{\varepsilon}{2(|\mathcal{T}^y(\varepsilon)|+\varepsilon +1)} \right)y$ dostáváme
\begin{equation}
\begin{split}=\bigl\vert \mathcal{T}^x\left(\frac{\varepsilon}{2(|\mathcal{T}^y(\varepsilon)|+\varepsilon +1)}\right) *\mathcal{T}^y\left(\frac{\varepsilon}{2(|\mathcal{T}^x(\varepsilon)|+\varepsilon +1)}\right) -&
\\-\mathcal{T}^x\left(\frac{\varepsilon}{2(|\mathcal{T}^y(\varepsilon)|+\varepsilon +1)} \right)y + \mathcal{T}^x\left(\frac{\varepsilon}{2(|\mathcal{T}^y(\varepsilon)|+\varepsilon +1)} \right)y -xy \bigr\vert =&
\end{split}
\end{equation}
a po vytknutí $|\mathcal{T}^x\left(\frac{\varepsilon}{2(|\mathcal{T}^y(\varepsilon)|+\varepsilon +1)}\right)|$ z prvních dvou členů a $|y|$ z druhých dvou máme
\begin{equation}
\begin{split}
=|\mathcal{T}^x\left(\frac{\varepsilon}{2(|\mathcal{T}^y(\varepsilon)|+\varepsilon +1)}\right)||\mathcal{T}^y\left(\frac{\varepsilon}{2(|\mathcal{T}^x(\varepsilon)|+\varepsilon +1)}\right)-y|+\\+|y||\mathcal{T}^x\left(\frac{\varepsilon}{2(|\mathcal{T}^y(\varepsilon)|+\varepsilon +1)}\right)-x|\leq
\end{split}
\end{equation}
a po dvojím použití pravidla $|\mathcal{T}^x(\varepsilon)-x|\leq\varepsilon$ získáváme
\begin{equation}
\begin{split}
\leq |\mathcal{T}^x\left(\frac{\varepsilon}{2(|\mathcal{T}^y(\varepsilon)|+\varepsilon +1)}\right)||\frac{\varepsilon}{2}||\frac{1}{(|\mathcal{T}^x(\varepsilon)|+\varepsilon +1)}|+\\+|y||\frac{\varepsilon}{2}||\frac{1}{(|\mathcal{T}^y(\varepsilon)|+\varepsilon +1)}|=
\end{split}
\end{equation}
a zjednodušíme-li pomocí zlomku, dostáváme
\begin{equation}
\begin{split}
= |\frac{\varepsilon}{2}||\frac{|\mathcal{T}^x\left(\frac{\varepsilon}{2(|\mathcal{T}^y(\varepsilon)|+\varepsilon +1)}\right)|}{(|\mathcal{T}^x(\varepsilon)|+\varepsilon +1)}|+\\+|\frac{\varepsilon}{2}||\frac{|y|}{(|\mathcal{T}^y(\varepsilon)|+\varepsilon +1)}|\leq
\end{split}
\end{equation}
a díky dokázaným nerovnostem \ref{duk:soucin_prvni} a \ref{duk:soucin_druha} pak platí
\begin{equation}
\leq\frac{\varepsilon}{2}+\frac{\varepsilon}{2}=\varepsilon.
\end{equation}
\end{proof}
\end{theorem}

Právě dokázaná věta mluví o součinu dvou tnumů. Zobecnění na konečný počet tnumů už uvedu bez důkazu.

\begin{fact}[Součin tnumů]\label{vet:soucin_tnumu}
Pro nenulová čísla $\{x_i\}_{i=0}^{n}$ a jejich tnumy platí
\begin{equation}
\mathcal{T}^{\prod_{i=0}^nx_i}(\varepsilon)=\prod_{i=0}^n\mathcal{T}^{x_i}\left(\frac{\varepsilon}{(n+1)*\prod_{j=0, i\neq j}^n(|\mathcal{T}^{x_j}(\varepsilon)|+\varepsilon +1)}\right).
\end{equation}
\end{fact}

Věta mluví o nenulových číslech. Nesnižujeme ale obecnost, protože nula je agresivní prvek a výsledkem násobení čehokoli s nulou je nula, takže se ostatní numy ani nemusejí počítat a výsledek se může vrátit.

Samotná implementace pak využívá pomocnou mapovací funkci, která vypadá trochu složitěji, ale velmi zlepšila porozumění funkci \texttt{tnum*}, o kterou nám teď jde především. Nejprve tedy pomocná

\begin{lispcode}{\texttt{create-list-for-multiplication}}{Pomocná fun\-kce pro násobení}
(\textcolor{funkcionalni}{defun} \textcolor{pojmenovan}{create-list-for-multiplication} (tnums eps)
  (\textcolor{vedlejsi}{let} ((result nil)
      (nums
        (\textcolor{funkcionalni}{mapcar} (\textcolor{funkcionalni}{lambda} (tnum) (\textcolor{moje}{tnum-to-num} tnum eps)) tnums)))
    (\textcolor{funkcionalni}{dotimes} (i (\textcolor{funkcionalni}{list-length} tnums) result)
      (\textcolor{vedlejsi}{let} ((actual-eps (\textcolor{matematicke}{/} eps (\textcolor{funkcionalni}{list-length} tnums))))
        (\textcolor{funkcionalni}{dotimes} (j (\textcolor{funkcionalni}{list-length} tnums))
          (\textcolor{funkcionalni}{unless} (\textcolor{matematicke}{=} i j)
            (\textcolor{vedlejsi}{setf} actual-eps (\textcolor{matematicke}{/} actual-eps
              (\textcolor{matematicke}{+} (\textcolor{funkcionalni}{nth} j nums) eps 1)))))
        (\textcolor{vedlejsi}{setf} result (cons
          (\textcolor{moje}{tnum-to-num} (\textcolor{funkcionalni}{nth} i tnums) actual-eps) result))))))
\end{lispcode}

a konečně už slíbená hlavní funkce. Místo jedničky vrací odpovídající $\mathcal{T}^1$.

\begin{lispcode}{\texttt{tnum*}}{Funkce pro násobení tnumů}
(\textcolor{funkcionalni}{defun} \textcolor{pojmenovan}{tnum*} (&rest tnums)
  (\textcolor{funkcionalni}{if} (\textcolor{funkcionalni}{null} tnums)
    (\textcolor{moje}{num-to-tnum} 1)
    (\textcolor{funkcionalni}{lambda} (eps)
      (\textcolor{funkcionalni}{apply} \textquotesingle\textcolor{moje}{*} (\textcolor{moje}{create-list-for-multiplication} tnums eps)))))
\end{lispcode}

Druhou obecnou multiplikativní funkcí je dělení.

\begin{fact}[Podíl tnumů]
Mějme čísla $x_{-1}, x_0, \ldots, x_n$ a jejich tnumy. Pak platí
\begin{equation}
\mathcal{T}^{x_{-1}/x_0/\ldots /x_n}(\varepsilon)=\mathcal{T}^{x_{-1}/\prod_{i=0}^nx_n}(\varepsilon)=\mathcal{T}^{x_{-1}*(\prod_{i=0}^nx_n)^{-1}}(\varepsilon)
\end{equation}
\end{fact}

Jedná se o tnumovský protějšek funkce \texttt{/}. Pro jeden argument vrací jeho převrácení, pro více pak jejich postupný podíl.

\begin{lispcode}{\texttt{tnum/}}{Funkce pro dělení tnumů}
(\textcolor{funkcionalni}{defun} \textcolor{pojmenovan}{tnum/} (tnum1 &rest tnums)
  (\textcolor{funkcionalni}{if} (\textcolor{funkcionalni}{null} tnums)
    (\textcolor{moje}{/tnum} tnum1)
    (\textcolor{moje}{tnum*} tnum1 (\textcolor{moje}{/tnum} (\textcolor{funkcionalni}{apply} \textquotesingle\textcolor{moje}{tnum*} tnums)))))
\end{lispcode}

Pro tnumy $a, b\in\mathfrak{T}$ existuje tnum $a+b$ a $a*b$. Tyto operace jsou tedy uzavřené. Tnum i v tomto smyslu dobře reprezentuje rekurzivní číslo, rekurzivní čísla totiž tvoří číselné těleso \cite{rice:kompr}.

\subsection{Mocninné operace}
Jak už sem psal v úvodu k této kapitole, na mocnění a odmocňování ještě nemáme v našem systému dostatečný aparát. K implementaci mocninných operací totiž potřebujeme funkce přirozeného logaritmu a exponenciály, které přidáme až v další kapitole.

\begin{lemma}[O mocnině tnumu]\label{vet:mocnina_tnumu}
Mějme $a>0$ a $b\in\mathbb{R}$, pak jejich mocninu $a^b$ lze vyjádřit jako $e^{(b*\mathrm{ln}(a))}$.
\begin{proof}
Kladné číslo $a$ lze vyjádřit jako $e^{(\mathrm{ln}(a))}$, $a^b$ je pak $e^{({\mathrm{ln}(a)})^b}$, což je pak $e^{(b*\mathrm{ln}(a))}$.
\end{proof}
\end{lemma}

Odmocninu potom přijmeme jako mocninu obrácené hodnoty.

\begin{fact}[Odmocnina jako mocnina \cite{tabulky}]\label{fac:odmocnina_tnumu}
Mějme $a, b\in\mathbb{R^+}$. Pak platí
\begin{equation}
\sqrt[a]{b}=b^{(a^{-1})}
\end{equation}
\end{fact}

Operace dokončíme na konci následující kapitoly.


\section{Funkce tnumů}
Knihovna \texttt{tnums} v této chvíli umí přidávat racionální čísla, Ludolfovo číslo a provádět mezi nimi multiplikativní a aditivní operace. Bylo by vhodné teď přidat další rekurzivní čísla. Jejich dobrým zdrojem, jak jsem napsal již v~podkapitole \ref{funkce_cisel}, jsou matematické funkce. V této kapitole se podíváme na funkci exponenciální, šest funkcí goniometrických a přirozený logaritmus. Na konci potom dodělám matematické operace a tím bude knihovna v použitelné verzi hotová.

\subsection{Aproximace funkcí}
Představa funkce tnumu je, že bude opět vracet tnum. Chci totiž opět libovolnou přesnost a také umožnit zřetězování funkcí. Hledám pak formu aproximace, která bude umožňovat libovolně škálovat, jak blízko ke kýženému číslu se výpočet ukončí. Dobrým nástrojem k tomu jsou Taylorovy polynomy. Ty se snaží hledat hodnotu $T(x)$ tak, aby byla co nejblíže hledané hodnotě $f(x)$ tak, že z~nějakého bodu, kterému budeme říkat počátek, se co nejlépe snaží nepodobit průběh funkce, kterou aproximují. Pro funkci $f$ budu Taylorův polynom stupně $n$ se středem v $a$ značit $T^{f,a}_n$. Pro práci s Taylorovými polynomy potřebujeme ještě naprogramovat faktoriál přirozeného čísla. To bývá typická úloha na rekurzi -- té se ale vyhýbáme, protože pro velké vstupy může přetékat zásobník. Iterativní verze by tímto neduhem neměla trpět.

\begin{lispcode}{\texttt{factorial}}{Funkce pro výpočet faktoriálu přirozeného čísla}
(\textcolor{funkcionalni}{defun} \textcolor{pojmenovan}{factorial} (n)
  (\textcolor{vedlejsi}{let} ((result 1))
    (\textcolor{funkcionalni}{loop} \textcolor{obarvi}{for} i \textcolor{obarvi}{from} n \textcolor{obarvi}{downto} 1
      \textcolor{obarvi}{do} (\textcolor{vedlejsi}{setf} result (\textcolor{matematicke}{*} result i)))
    result))
\end{lispcode}

Podívejme se teď na to, jak se prakticky dá počítat aproximace funkce v bodě $x$. Nejjednodušší je vzít funkci $T^{f,a}_0(x) = f(a)$. Je to jednoduchá aproximace, která na velmi blízkém okolí bodu $a$ může fungovat i velmi uspokojivě. Lepší nápadem je vzít přímku, která se bude dotýkat grafu funkce $f$ v bodě $a$. Předpis takovéto bude $T^{f,a}_1(x) = f(a)+f'(a)(x-a)$. To už je lepší aproximace, protože nebere v úvahu jen hodnotu funkce $f$ v bodě $a$ ale i její první derivaci, takže víme více o směru, kam se možná bude pohybovat. Ještě lepším nápadem pak je vzít parabolu přimknutou k grafu funkce $f$ jako $T^{f,a}_2(x) = f(a)+f'(a)(x-a)+\frac{f''(a)}{2}(x-a)^2$. Teď už zohledňujeme funkční hodnotu, směr křivky i konvexnost. Ještě lepším nápadem je použít $T^{f,a}_3(x):y=f(a)+f'(a)(x-a)+\frac{f''(a)}{2}(x-a)^2+\frac{f'''(a)}{6}(x-a)^3\ldots$\cite{MTTP}.

\begin{remind}[Taylorova a Maclaurinova řada]
Kdybychom takto postupovali donekonečna (v limitním smyslu), dostali bychom Taylorovu řadu z definice \ref{def:taymac_rada}. Pro $a=0$ pak Taylorovu řadu nazýváme řadou Maclaurinovou.
\end{remind}

Lze odvodit, že pokud Taylorovy zbytky konvergují k nule, lze Taylorovou řadou $T_\infty^{f,a}$ nahradit funkci $f$ \cite{ZDVNNR}. Nám ale nestačí pouhá konvergence zbytků, ale chtěli bychom jejich velikost nějak omezovat. Nejprve si zkusme nějakou formou zbytky vyjádřit.

\begin{fact}[Taylorova věta \cite{TMA:Calculus}]
Nechť $f$ má spojité derivace až do řádu $n+1$ na nějakém intervalu obsahujícím $a$. Pak pro každé $x$ z tohoto intervalu máme Taylorův vzorec
\begin{equation}
f(x) = T_n^{f,a}(x) + R_n^{f,a}(x)\text{, kde}
\end{equation}
\begin{equation}\label{TP}
T_n^{f,a}(x) = \sum_{i=0}^{n}\frac{f{(i)}(a)}{i!}(x-a)^i,
\end{equation}
\begin{equation}\label{integralnitvar}
R_n^{f,a}(x)=\int_a^x\frac{(x-t)^n}{n!}f^{(n+1)}(t)dt.
\end{equation}
Navíc existuje číslo $\xi$, z intervalu s krajními body $x$ a $a$ takové, že
\begin{equation}\label{lagrangeuvtvar}
R_n^{f,a}(x)=\frac{f^{(n+1)}(\xi)}{(n+1)!}(x-a)^{n+1}.
\end{equation}
Pro důkaz vizte kapitolu 7.5 v \cite{TMA:Calculus}.
\end{fact}

Součet v rovnici \ref{TP} nazýváme Taylorův polynom funkce $f$ stupně $n$ v bodě $a$, $R_n^{f,a}(x)$ nazýváme $n$-tým Taylorovým zbytkem. Vyjádření \ref{integralnitvar} pak říkáme \textit{integrální tvar} zbytku a \ref{lagrangeuvtvar} je Lagrangeův tvar zbytku \cite{MTTP}.

Když už máme vyjádřeny zbytky, můžeme se pokusit je zhora omezovat, stejně jako tomu bylo u geometrické řady. Ve skutečnosti nám na celou kapitolu vystačí pouze tyto dva mechanismy, tedy \textit{Taylorův zbytek} a \textit{Zbytek geometrické řady}.

\subsection{Exponenciála}
Exponenciála je funkce s předpisem $\mathrm{exp}(x) = e^x$ kde $e$ je tzv. \textit{Eulerovo číslo} definované $e=\lim_{n\to\infty}\left(1+\frac{1}{n}\right)^n$ \cite{EPJVMAI}. To je transcendentní konstanta a je to též základ přirozeného logaritmu. Exponenciále se proto také dá říkat přirozená mocnina. Ještě podotkněme, že $\frac{d}{dx}\mathrm{exp}(x)=\mathrm{exp}(x)$.

\begin{remark}[Značení exponenciály]
Mimo informatickou oblast jsem si nikde nevšiml, že by se exponenciála čísla $x$ značila jinak než $e^x$, mé značení $\mathrm{exp}(x)$ tedy možná působí neadekvátně. V dalším textu ale používám i pouze funkci ($\mathrm{exp}$), nikoli její hodnotu ($\mathrm{exp}(x)$) a zápis $e^x$ umožňuje jen toto druhé použití. Proto se omlouvám matematickému čtenáři za neintuitivní značení, ale je zde důvodné. Navíc lépe vyjadřuje, že je exponenciála funkcí.
\end{remark}

\subsubsection{Exponenciála čísla}

\begin{fact}[Exponenciála jako Maclaurinova řada \cite{ZDVNNR}]\label{vet:exp_jako_rada}
Funkci $\mathrm{exp}(x)$ lze vyjádřit jako Maclaurinovu řadu ve tvaru
\begin{equation}
\mathrm{exp}(x) = \underset{i \in \mathbb{N}}{\sum} \frac{x^i}{i!} = \frac{1}{1} + \frac{x}{1} + \frac{x^2}{2!} + \frac{x^3}{3!} + \ldots
\end{equation}
Pro důkaz vizte podkapitolu \ref{duk:exp_jako_rada} v příloze \ref{pril:dukazy}.
\end{fact}

Podívejme se nyní na zbytek této řady. Když rozepíšeme Lagrangeův tvar, získáváme pro nějaké $\xi\in(0,x)$
\begin{equation}
R_n^{exp, 0}(x) = \frac{e^\xi}{(n+1)!}x^{n+1}.
\end{equation}

Podotkněme, že exponenciála je rostoucí a že $e<2.72$ a proto
\begin{equation}
(\forall\xi\in (0,x))(\mathrm{exp}(\xi)<\mathrm{exp}(x) < 2.72^x).
\end{equation}

Když vše poskládáme dohromady, získáváme aproximaci exponenciály na shora omezenou přesnost, pro $x\in\mathbb{R}$ platí
\begin{fact}[Omezení Taylorova zbytku exponenciály]
\begin{equation}
|R_n^{exp, 0}(x)| = \left|\mathrm{exp}(x)- \sum_{i=0}^n \frac{x^i}{i!}\right| \leq \left| \frac{2.72^x}{(n+1)!}x^{n+1} \right|.
\end{equation}
\end{fact}

\begin{consequence}[O tnumu exponenciály numu]
Pro všechna $\varepsilon$ existuje $n\in\mathbb{N}^+$ tak, aby $\left|\frac{2.72^x}{(n+1)!}x^{n+1}\right| \leq \varepsilon$ a pak
\begin{equation}
\mathcal{T}^{\mathrm{exp}(x)}(\varepsilon)=\sum_{i=0}^n \frac{x^i}{i!}.
\end{equation}
\begin{proof}
Existence čísla $n$ je zřejmá z definice limity posloupnosti a z toho, že limita podílu polynomu a faktoriálu je rovna nule.

Dále protože $\mathrm{exp}(x) = T^{exp, 0}_n(x)+R^{exp, 0}_n(x)$, lze psát
\begin{equation}
T^{exp, 0}_n(x) \in [\mathrm{exp}(x) - |R^{exp, 0}_n(x)|, \mathrm{exp}(x) + |R^{exp, 0}_n(x)|],
\end{equation}
přičemž dle předchozího faktu platí
\begin{equation}
T^{exp, 0}_n(x) \in [\mathrm{exp}(x) - \left| \frac{2.72^x}{(n+1)!}x^{n+1} \right|, \mathrm{exp}(x) + \left| \frac{2.72^x}{(n+1)!}x^{n+1} \right|]
\end{equation}
a z předpokladu pak
\begin{equation}
T^{exp, 0}_n(x) \in [\mathrm{exp}(x) - \varepsilon, \mathrm{exp}(x) + \varepsilon]
\end{equation}
a tedy
\begin{equation}\label{eq:tat}
\mathcal{T}^{\mathrm{exp}(x)}(\varepsilon) = T^{exp, 0}_n(x).
\end{equation}
\end{proof}
\end{consequence}

Při implementaci stačí jen iterovat přes $n$, dokud nebude právě odvozené omezení zbytku menší než kýžená přesnost. Stejný přístup jsme viděli již u Ludolfova čísla.

\begin{lispcode}{\texttt{num-exp}}{Funkce pro výpočet exponenciály čísla na danou přesnost}
(\textcolor{funkcionalni}{defun} \textcolor{pojmenovan}{num-exp} (num eps)
  (\textcolor{vedlejsi}{let} ((above (rat-expt 272/100 num)) (n 0)
      (nfact 1) (xpown 1) (result 1))
    (\textcolor{funkcionalni}{loop} 
      \textcolor{obarvi}{until} (\textcolor{matematicke}{<=} (\textcolor{matematicke}{/} (\textcolor{matematicke}{*} above xpown) nfact) eps)
      \textcolor{obarvi}{do} (progn
        (\textcolor{vedlejsi}{incf} n)
        (\textcolor{vedlejsi}{setf} nfact (\textcolor{moje}{factorial} n)
          xpown (\textcolor{matematicke}{expt} num n))
        (\textcolor{vedlejsi}{incf} result (\textcolor{matematicke}{/} xpown nfact)))
      \textcolor{obarvi}{finally} (\textcolor{funkcionalni}{return} result))))
\end{lispcode}

\begin{remark}[Eulerovo číslo jako exponenciála]
Protože triviálně platí $e = e^1$, můžu do knihovny přidat i samotné Eulerovo číslo jako jednoduchou uživatelskou funkci.
\begin{lispcode}{\texttt{tnum-e}}{Funkce pro $\mathcal{T}^e$}
(\textcolor{funkcionalni}{defun} \textcolor{pojmenovan}{tnum-e} ()
  (\textcolor{funkcionalni}{lambda} (eps)
    (\textcolor{moje}{num-exp} 1 eps)))
\end{lispcode}
\end{remark}

Už tedy umíme exponenciálu čísla na danou přesnost. Teď jsme tedy ve stádiu, kdy lze pro $q\in\mathbb{Q}$ napsat
\begin{equation}
\mathcal{T}^{\mathrm{exp}(q)}=\texttt{(num-exp } q \texttt{)}.
\end{equation}
To není malý výsledek, bohužel nás ale sotva uspokojí. Nyní ještě musíme rozšířit funkcionalitu na všechna reálná čísla, která mají tnum. Opět jde o linku rozdílu mezi racionálními a rekurzivními čísly, která prochází celou prací.

\subsubsection{Exponenciála tnumu}
Víme, že $\mathcal{T}^x(\varepsilon) \in [x-\varepsilon,x+\varepsilon]$, podívejme se, jak se chová přesnost čísla po projití exponenciální funkcí.

\begin{myfigure}{H}
\caption{Obraz přesnosti po průchodu exponenciálou}
\includegraphics[width=\linewidth]{graphics/exp1.pdf}\label{fig:exp1}
Horní interval je vyšší než spodní.

\includegraphics[width=\linewidth]{graphics/exp4.pdf}\label{fig:exp4}
Interval $[x-\varepsilon,x+\varepsilon]$ se nezobrazí na $[e^x-\varepsilon,e^x+\varepsilon]$. Tím pádem $\mathcal{T}^{\mathrm{exp}(x)}(\varepsilon) \neq \mathcal{T}^{\mathrm{exp}(\mathcal{T}^x(\varepsilon))}(\varepsilon)$.
\end{myfigure}

Vidíme, že $|(x-\varepsilon)-x|=|(x+\varepsilon)-x|$, ale $|\mathrm{exp}(x-\varepsilon)-\mathrm{exp}(x)|\neq|\mathrm{exp}(x+\varepsilon)-\mathrm{exp}(x)|$, tedy že přesnost se průchodem nelineární funkcí deformuje a proto se interval $[\mathrm{exp}(x-\varepsilon),\mathrm{exp}(x+\varepsilon)]$ neshoduje s intervalem $[\mathrm{exp}(x)-\varepsilon,\mathrm{exp}(x)+\varepsilon]$. Důsledkem pak je, že nelze rozšířit funkce racionálních čísel na tnumy ve smyslu $\mathcal{T}^{\mathrm{exp}(x)}(\varepsilon) := \mathcal{T}^{\mathrm{exp}(\mathcal{T}^x(\varepsilon)}(\varepsilon)$, ale budeme to muset udělat šetrněji.

Pátráme po metodě, která nám řekne, jak přesné má být číslo na vstupu do funkce, aby jeho obraz byl v zadané přesnosti.

\begin{myfigure}{H}
\caption{Vzor přesnosti před průchodem exponenciálou}
\includegraphics[width=\linewidth]{graphics/exp2.pdf}\label{fig:exp2}
Levý interval je širší než pravý.

\includegraphics[width=\linewidth]{graphics/exp3.pdf}\label{fig:exp3}
Hledáme, jaké okolí bodu $x$ se zobrazí na $\varepsilon$-okolí bodu $e^x$.
\end{myfigure}

Když si to vyneseme do rovnice, bude vypadat
\begin{equation}\label{invpresexp}
\mathrm{exp}(x)+\varepsilon=\mathrm{exp}(x+w),
\end{equation}
kde hledaná neznámá je $w$.

\begin{myfigure}{H}
\caption{Zobrazení neznámé $w$}
\includegraphics[width=.5\linewidth]{graphics/exp5.pdf}\label{fig:exp5}
\includegraphics[width=.5\linewidth]{graphics/exp6.pdf}\label{fig:exp6}
Neznámou $w$ lze zobrazit jako jednu z odvěsen, druhá je $\varepsilon$. Fialová čára není přepona, ale exponenciála.
\end{myfigure}

Podívejme se nyní, jaký vztah je mezi $w$ a $\varepsilon$. Pro úhel $\alpha$ při vrcholu $W$ platí, že $\mathrm{ctan}(\alpha) = \frac{\varepsilon}{w}$. Tedy
\begin{equation}
w = \frac{\varepsilon}{\mathrm{ctan}(\alpha)} = \frac{\varepsilon}{\mathrm{tan}(\frac{\pi}{2}-\alpha)} = \frac{\varepsilon}{exp(x+w)}.
\end{equation}

Vychází rekurzivní vztah pro $w$, po jeho dosazení do vztahu \ref{invpresexp} dostáváme
\begin{equation}\label{rekpresexp}
\mathrm{exp}(x)+\varepsilon=\mathrm{exp}\left(x+\frac{\varepsilon}{\mathrm{exp}\left(x+\frac{\varepsilon}{\mathrm{exp}(x+\ldots)}\right)}\right).
\end{equation}

Podobným způsobem lze odvodit i vztah pro opačný kraj okolí a to
\begin{equation}\label{rekpresexp2}
\mathrm{exp}(x)-\varepsilon=\mathrm{exp}\left(x-\frac{\varepsilon}{\mathrm{exp}\left(x-\frac{\varepsilon}{\mathrm{exp}(x-\ldots)}\right)}\right),
\end{equation}

dohromady to pak po propojení s notací tnumů dává vztah

\begin{equation}\label{rekpresexp3}
\mathcal{T}^{\mathrm{exp}(x)}(\varepsilon)\in\left[
\mathrm{exp}\left(x-\frac{\varepsilon}{\mathrm{exp}\left(x-\frac{\varepsilon}{\mathrm{exp}(x-\ldots)}\right)}\right), \mathrm{exp}\left(x+\frac{\varepsilon}{\mathrm{exp}\left(x+\frac{\varepsilon}{\mathrm{exp}(x+\ldots)}\right)}\right)\right].
\end{equation}

Nyní by mělo být jasné, že při implementaci budeme hledat pevný bod a proto si zavedeme tzv. \textit{precizní operátor}.

\begin{definition}[Precizní operátor]
Definujme následující posloupnost:
\begin{equation}
\left[ \mathcal{T}^x\right]_0^{f,\varepsilon}=\mathcal{T}^{f(\mathcal{T}^x(\varepsilon))}(\varepsilon),
\end{equation}
\begin{equation}
\left[ \mathcal{T}^x\right]_{n+1}^{f,\varepsilon}=\mathcal{T}^{f\left(\mathcal{T}^x\left(\frac{\varepsilon}{\left|\left[ \mathcal{T}^x\right]_n^{f,\varepsilon}\right|+\varepsilon}\right)\right)}(\varepsilon)
\end{equation}
a pokud existuje $m$ tak, že
\begin{equation}
\left[\mathcal{T}^x\right]_m^{f,\varepsilon} = \left[\mathcal{T}^x\right]_{m+1}^{f,\varepsilon}\text{, pak klademe}\left[\mathcal{T}^x \right]_\infty^{f,\varepsilon} := \left[\mathcal{T}^x\right]_m^{f,\varepsilon}.
\end{equation}
\end{definition}

\begin{remark}[Vše je operátor]
Jako operátor se v matematice běžně označuje funkce, která jako vstup nebere číslo, ale nějakou jinou funkci. V tomto pohledu jsou všechny funkce, které jsme naprogramovali pro tnumy ve skutečnosti operátory, nikoli funkcemi. My ale funkcemi simulujeme čísla a proto dodržíme zavedenou terminologii a operátorem budeme nazývat pouze Precizní operátor, ikdyž operátorem je v podstatě vše.
\end{remark}

\begin{lispcode}{\texttt{precise-operator}}{Funkce pro rozšíření funkcí z racionálních čísel na tnumy}
(\textcolor{funkcionalni}{defun} \textcolor{pojmenovan}{precise-operator} (tnum eps f)
  (\textcolor{vedlejsi}{let*} ((num (\textcolor{moje}{tnum-to-num} tnum eps))
         (fnum (\textcolor{funkcionalni}{funcall} f num eps))
         (new 1))
    (\textcolor{funkcionalni}{loop} 
      \textcolor{obarvi}{until} (\textcolor{matematicke}{=} num new)
      \textcolor{obarvi}{do} (\textcolor{vedlejsi}{setf} new (\textcolor{funkcionalni}{if} (\textcolor{matematicke}{>} (\textcolor{matematicke}{abs} fnum) 1)
            (\textcolor{moje}{tnum-to-num} tnum (\textcolor{matematicke}{/} eps (\textcolor{matematicke}{+} (\textcolor{matematicke}{abs} fnum) eps)))
            num)
        num new
        fnum (\textcolor{funkcionalni}{funcall} f num eps))
      \textcolor{obarvi}{finally} (\textcolor{funkcionalni}{return} fnum))))
\end{lispcode}

\begin{consequence}[O exponenciále tnumu]\label{dusl:expotnumu}
Pro tnum $\mathcal{T}^x$ lze najít tnum $\mathcal{T}^{e^x}$ a má tvar
\begin{equation}
\mathcal{T}^{exp(x)}(\varepsilon)=\left[\mathcal{T}^x\right]_\infty^{exp,\varepsilon}.
\end{equation}
\begin{proof}
Vychází přímo ze vztahu \ref{rekpresexp3}.
\end{proof}
\end{consequence}

Je jasné, že operátor musí brát jako argumenty tnum, přesnost a funkci. Exponenciálu tnumu s výše uvedeným již naprogramujeme velmi přehledně.

\begin{lispcode}{\texttt{tnum-exp}}{Funkce pro výpočet exponenciály tnumu}
(\textcolor{funkcionalni}{defun} \textcolor{pojmenovan}{tnum-exp} (tnum)
  (\textcolor{funkcionalni}{lambda} (eps)
    (\textcolor{moje}{precise-operator} tnum eps \textquotesingle\textcolor{moje}{num-exp})))
\end{lispcode}

\subsection{Goniometrické}
Goniometrické funkce jsou opět reálné funkce reálné proměnné. Po zkušenosti s implementací exponenciály už nás kód nepřekvapí. První dvě naprogramujeme nízkoúrovňově, zbylé čtyři pak využijí již existujících. Poznamenejme, že $\frac{d}{dx}\mathrm{sin}(x) = \mathrm{cos}(x)$, $\frac{d}{dx}\mathrm{cos}(x) = -\mathrm{sin}(x)$ a že $H(\mathrm{sin}) = [-1,1] = H(\mathrm{cos})$.

\subsubsection{Sinus}
\begin{fact}[Sinus jako Maclaurinova řada \cite{ZDVNNR}]\label{vet:sin_jako_rada}
Funkci $\mathrm{sin}(x)$ lze vyjádřit jako Maclaurinovu řadu ve tvaru
\begin{equation}
\mathrm{sin}(x) =\sum_{i \in \mathbb{N}} (-1)^i \frac{x^{2i+1}}{(2i+1)!} =\frac{x}{1} - \frac{x^3}{3!} + \frac{x^5}{5!} - \frac{x^7}{7!} + \ldots
\end{equation}
Pro důkaz vizte podkapitolu \ref{duk:sin_jako_rada} v příloze \ref{pril:dukazy}.
\end{fact}

Dále protože jsou funkční hodnoty všech možných derivací v intervalu $[-1,1]$, lze Lagrangeův tvar zbytku vyjádřit znaménka a pak díky Taylorově větě platí
\begin{equation}
\left|\mathcal{R}^{sin, 0}_n(x)\right|\leq\left|\frac{x^{2n+1+1}}{(2n+1+1)!}\right|=\left|\frac{x^{2n+2}}{(2n+2)!}\right|.
\end{equation}

\begin{consequence}[Sinus numu]
Pro všechna $\varepsilon$ existuje $n\in\mathbb{N}^+$ tak, aby $\left|\frac{x^{2n+2}}{(2n+3)!}\right| \leq \varepsilon$ a pak
\begin{equation}
\mathcal{T}^{\mathrm{sin}(x)}(\varepsilon)=\sum_{i=0}^n (-1)^i \frac{x^{2i+1}}{(2i+1)!}
\end{equation}
\begin{proof}
Běží podobně jako u exponenciály. Jde opět o exponenciálu nad faktoriálem, proto je jasná limita i existence $n$. Dále z omezení $R^{\mathrm{sin}, 0}_n$ lze odvodit $T^{\mathrm{sin}, 0}_n\in[\mathrm{sin}(x)-|R^{\mathrm{sin}, 0}_n|,\mathrm{sin}(x)+|R^{\mathrm{sin}, 0}_n|]$ a pak $T^{\mathrm{sin}, 0}_n\in[\mathrm{sin}(x)-|\frac{x^{2n+2}}{(2n+2)!}|,\mathrm{sin}(x)+|\frac{x^{2n+2}}{(2n+2)!}|]$ a tudíž $T^{\mathrm{sin}, 0}_n\in[\mathrm{sin}(x)-\varepsilon,\mathrm{sin}(x)+\varepsilon]$, z čehož pak $\mathcal{T}^{\mathrm{sin}(x)}(\varepsilon)=T^{\mathrm{sin},0}_n(x)$.
\end{proof}
\end{consequence}

\begin{lispcode}{\texttt{num-sin}}{Funkce pro sinus čísla}
(\textcolor{funkcionalni}{defun} \textcolor{pojmenovan}{num-sin} (x eps)
  (\textcolor{vedlejsi}{let} ((n 0) (result 0) (2n+1 1))
    (\textcolor{funkcionalni}{loop} 
      \textcolor{obarvi}{until} (\textcolor{matematicke}{<=} (\textcolor{matematicke}{abs} (\textcolor{matematicke}{/} (\textcolor{moje}{rat-expt} x (\textcolor{matematicke}{1+} 2n+1))
                        (\textcolor{moje}{factorial} (\textcolor{matematicke}{1+} 2n+1))))
                eps)
      \textcolor{obarvi}{do} (\textcolor{funkcionalni}{progn} 
          (\textcolor{vedlejsi}{incf} result
            (\textcolor{matematicke}{/} (\textcolor{moje}{rat-expt} x 2n+1)
               (\textcolor{moje}{factorial} 2n+1)
               (\textcolor{matematicke}{expt} -1 n)))
          (\textcolor{vedlejsi}{incf} n)
          (\textcolor{vedlejsi}{setf} 2n+1 (\textcolor{matematicke}{1+} (\textcolor{matematicke}{*} 2 n))))
      \textcolor{obarvi}{finally} (\textcolor{funkcionalni}{return} result))))
\end{lispcode}

Pro rozšíření možných vstupů mimo racionálních čísel též na všechna reálná čísla, která mají tnumy použijeme opět Precizní operátor. Zformulujme si tedy nyní obecnější podobu důsledku \ref{dusl:expotnumu}.

\begin{hypothesis}[O funkci tnumu]\label{hyp:ofcitnumu}
Pro funkce $f$ a $\mathcal{T}^x$, $x\in D(f)$ platí
\begin{equation}
\mathcal{T}^{f(x)}(\varepsilon)=[\mathcal{T}^x]^{f, \varepsilon}_\infty.
\end{equation}
\end{hypothesis}

\begin{consequence}[Sinus tnumu]
\begin{equation}
\mathcal{T}^{\mathrm{sin}(x)}(\varepsilon)=[\mathcal{T}^x]^{\mathrm{sin}, \varepsilon}_\infty
\end{equation}
\end{consequence}

\begin{lispcode}{\texttt{tnum-sin}}{Funkce pro sinus tnumu}
(\textcolor{funkcionalni}{defun} \textcolor{pojmenovan}{tnum-sin} (tnum)
  (\textcolor{funkcionalni}{lambda} (eps)
    (\textcolor{moje}{precise-operator} tnum eps \textquotesingle\textcolor{moje}{num-sin})))
\end{lispcode}

\subsubsection{Kosinus}
\begin{fact}[Kosinus jako Maclaurinova řada \cite{ZDVNNR}]\label{vet:cos_jako_rada}
Funkci $\mathrm{cos}(x)$ lze vyjádřit jako Maclaurinovu řadu ve tvaru
\begin{equation}
\mathrm{cos}(x) = \sum_{i\in\mathbb{N}}(-1)^i \frac{x^{2i}}{(2i)!} = \frac{1}{1} - \frac{x^2}{2!} + \frac{x^4}{4!} - \frac{x^6}{6!} + \ldots
\end{equation}
Pro důkaz vizte podkapitolu \ref{duk:cos_jako_rada} v příloze \ref{pril:dukazy}.
\end{fact}

Z Taylorovy věty získáváme omezení Taylorova zbytku
\begin{equation}
|R^{cos}_n(x)|\leq\left|\frac{x^{2n+1}}{(2n+1)!}\right|
\end{equation}
a proto opět hledáme takové $n$, že když pro jakékoli $\varepsilon$ je $|\frac{x^{2n+1}}{(2n+1)!}|\leq\varepsilon$, pak 
\begin{consequence}[Kosinus numu]
\begin{equation}
\mathcal{T}^{\mathrm{cos}(x)}(\varepsilon)=\sum_{i=0}^n(-1)^i \frac{x^{2i}}{(2i)!}.
\end{equation}
\end{consequence}

\begin{lispcode}{\texttt{num-cos}}{Funkce pro výpočet kosinu čísla}
(\textcolor{funkcionalni}{defun} \textcolor{pojmenovan}{num-cos} (x eps)
  (\textcolor{vedlejsi}{let} ((n 0) (result 0) (2n 0))
    (\textcolor{funkcionalni}{loop} 
      \textcolor{obarvi}{until} (\textcolor{matematicke}{<} (\textcolor{matematicke}{abs} (\textcolor{matematicke}{/} (\textcolor{moje}{rat-expt} x (\textcolor{matematicke}{1+} 2n))
                        (\textcolor{moje}{factorial} (\textcolor{matematicke}{1+} 2n))))
                eps)
      \textcolor{obarvi}{do} (\textcolor{funkcionalni}{progn} 
          (\textcolor{vedlejsi}{incf} result
            (\textcolor{matematicke}{/} (\textcolor{moje}{rat-expt} x 2n)
               (\textcolor{moje}{factorial} 2n)
               (\textcolor{matematicke}{expt} -1 n)))
          (\textcolor{vedlejsi}{incf} n)
          (\textcolor{vedlejsi}{setf} 2n (\textcolor{matematicke}{*} 2 n)))
      \textcolor{obarvi}{finally} (\textcolor{funkcionalni}{return} result))))
\end{lispcode}

A nakonec právě naprogramovanou funkci využijeme ke kosinování jakékoli proměnné s tnumem. Nepřekvapí použití Precizního operátoru.

\begin{lemma}[O kosinu tnumu]
Pro funkci $\mathcal{T}^x$, $x\in\mathbb{R}$ existuje $\mathcal{T}^{\mathrm{cos}(x)}$ a je ve tvaru
\begin{equation}
\mathcal{T}^{\mathrm{cos}(x)}(\varepsilon)=[\mathcal{T}^x]_\infty^{\mathrm{cos}, \varepsilon}.
\end{equation}
\begin{proof}
Protože $D(\mathrm{cos})=\mathbb{R}$, plyne přímo z hypotézy \ref{hyp:ofcitnumu}.
\end{proof}
\end{lemma}

\begin{lispcode}{\texttt{tnum-cos}}{Funkce pro výpočet kosinu tnumu}
(\textcolor{funkcionalni}{defun} \textcolor{pojmenovan}{tnum-cos} (tnum)
  (\textcolor{funkcionalni}{lambda} (eps)
    (\textcolor{moje}{precise-operator} tnum eps \textquotesingle\textcolor{moje}{num-cos})))
\end{lispcode}

Zbylé goniometrické funkce už naprogramujeme uživatelsky.

\subsubsection{Další goniometrické funkce}
Další goniometrickou funkcí je tangens. Dá se vyjádřit pomocí sinu a kosinu, díky čemuž ho nemusím vyjadřovat jako řadu, i když pro všechny goniometrické funkce řady existují. Nejsou ale konvergentní na celé reálné ose, takže se pro naši knihovnu nehodí.
\begin{fact}[Tangens jako poměr sinu a kosinu \cite{tabulky}]
\begin{equation}
\mathrm{tg}(x)=\frac{\mathrm{sin}(x)}{\mathrm{cos}(x)}
\end{equation}
\end{fact}

\begin{convention}[O vypuštění některých důsledků]
Nyní by měl následovat důsledek že $\mathcal{T}^{\mathrm{tan}(x)}=\mathcal{T}^{\frac{\mathcal{T}^{\mathrm{sin}(x)}}{\mathcal{T}^{\mathrm{cos}(x)}}}$, což je ale myslím jasné a proto zde ani u dalších zřejmých přepsání vzorečků do jazyka tnumů tyto důsledky neuvádím.
\end{convention}

\begin{lispcode}{\texttt{tnum-tan}}{Funkce pro výpočet tangentu tnumu}
(\textcolor{funkcionalni}{defun} \textcolor{pojmenovan}{tnum-tan} (tnum)
  (\textcolor{moje}{tnum/} (\textcolor{moje}{tnum-sin} tnum) (\textcolor{moje}{tnum-cos} tnum)))
\end{lispcode}

Zbylé funkce jsou obrácenou hodnotou již napsaných.

\begin{fact}[Kosekans jako obrácená hodnota sinu \cite{tabulky}]
  \begin{equation}
    \mathrm{csc}(x)=\mathrm{sin}^{-1}(x)
  \end{equation}
\end{fact}
\begin{lispcode}{\texttt{tnum-csc}}{Funkce pro výpočet kosekantu tnumu}
(\textcolor{funkcionalni}{defun} \textcolor{pojmenovan}{tnum-csc} (tnum)
  (\textcolor{moje}{/tnum} (\textcolor{moje}{tnum-sin} tnum)))
\end{lispcode}

\begin{fact}[Sekans jako obrácená hodnota kosinu \cite{tabulky}]
  \begin{equation}
    \mathrm{sec}(x)=\mathrm{cos}^{-1}(x)
  \end{equation}
\end{fact}
\begin{lispcode}{\texttt{tnum-sec}}{Funkce pro výpočet sekantu tnumu}
(\textcolor{funkcionalni}{defun} \textcolor{pojmenovan}{tnum-sec} (tnum)
  (\textcolor{moje}{/tnum} (\textcolor{moje}{tnum-cos} tnum)))
\end{lispcode}

\begin{fact}[Kotangens jako obrácená hodnota tangentu \cite{tabulky}]
  \begin{equation}
    \mathrm{cotg}(x)=tan^{-1}(x)=\frac{\mathrm{cos}(x)}{\mathrm{sin}(x)}
  \end{equation}
\end{fact}
\begin{lispcode}{\texttt{tnum-ctan}}{Funkce pro výpočet kotangentu tnumu}
(\textcolor{funkcionalni}{defun} \textcolor{pojmenovan}{tnum-ctan} (tnum)
  (\textcolor{moje}{tnum/} (\textcolor{moje}{tnum-cos} tnum) (\textcolor{moje}{tnum-sin} tnum)))
\end{lispcode}

\subsection{Logaritmus}
Logaritmus je inverzní funkce k exponenciále. Je opět vyjadřitelná řadou.
\begin{fact}[Logaritmus jako řada \cite{HoMF}]
Pro $x\in\mathbb{R}^+$ platí
  \begin{equation}
    \mathrm{ln}(x)=2\sum_{i\in\mathbb{N}}\frac{1}{2i+1}\left(\frac{x-1}{x+1}\right)^{2i+1}
  \end{equation}
\end{fact}

Člen $\frac{1}{2i+1}(\frac{x-1}{x+1})^{2i+1}$ je menší než $(\frac{x-1}{x+1})^{2i+1}$ a  tento je menší než $(\frac{x-1}{x+1})^{2i}$. Toto je geometrická posloupnost, jejíž $n$-tý zbytek je roven $\frac{(\frac{x-1}{x+1})^{2n+2}}{1-\frac{x-1}{x+1}}$ podle faktu \ref{vet:o_zbytku_geometricke_rady}.
\begin{lispcode}{\texttt{num-ln}}{Funkce pro logaritmus čísla}
(\textcolor{funkcionalni}{defun} \textcolor{pojmenovan}{num-ln} (x eps)
  (\textcolor{vedlejsi}{setf} eps (\textcolor{matematicke}{/} eps 2))
  (\textcolor{vedlejsi}{let} ((n 0) (result 0) (q (\textcolor{matematicke}{/} (\textcolor{matematicke}{1-} x) (\textcolor{matematicke}{1+} x))))
    (\textcolor{funkcionalni}{loop} 
      \textcolor{obarvi}{until} (\textcolor{matematicke}{<=} (\textcolor{matematicke}{/} (\textcolor{matematicke}{expt} q (\textcolor{matematicke}{*} 2 (\textcolor{matematicke}{1+} n))) (\textcolor{matematicke}{-} 1 q))
                eps)
      \textcolor{obarvi}{do} (\textcolor{funkcionalni}{progn} 
          (\textcolor{vedlejsi}{incf} result
            (\textcolor{matematicke}{/} (\textcolor{matematicke}{expt} q (\textcolor{matematicke}{1+} (\textcolor{matematicke}{*} 2 n))) (\textcolor{matematicke}{1+} (\textcolor{matematicke}{*} 2 n))))
          (\textcolor{vedlejsi}{incf} n))
      \textcolor{obarvi}{finally} (\textcolor{funkcionalni}{return} (\textcolor{matematicke}{*} 2 result)))))
\end{lispcode}

Tím bychom měli přirozený logaritmus pro čísla. Precizní operátor překvapivě funguje i zde, takže rozšíření na tnumy je již dílem okamžiku.

\begin{lispcode}{\texttt{tnum-ln}}{Funkce pro logaritmus tnumu}
(\textcolor{funkcionalni}{defun} \textcolor{pojmenovan}{tnum-ln} (tnum)
  (\textcolor{funkcionalni}{lambda} (eps)
    (\textcolor{moje}{precise-operator} tnum eps \textquotesingle\textcolor{moje}{num-ln})))
\end{lispcode}

\begin{remark}[Dokončení systému]
Pomocí logaritmu v kombinaci s exponenciálou lze přinést i mocninné operace a ucelím tak základní funkcionalitu knihovny \texttt{tnums}, kterou jsem si předsevzal. Mocninu udělám jako v lemmatu \ref{vet:mocnina_tnumu}.
\begin{lispcode}{\texttt{tnum-expt}}{Funkce pro umonování tnumů}
(\textcolor{funkcionalni}{defun} \textcolor{pojmenovan}{tnum-expt} (tnum1 tnum2)
  (\textcolor{moje}{tnum-exp} (\textcolor{moje}{tnum*} tnum2 (\textcolor{moje}{tnum-ln} tnum1))))
\end{lispcode}

Odmocninu pak píši podle faktu \ref{fac:odmocnina_tnumu}. Oproti mocnině jsou prohozené argumenty, ale tnum-tou odmocninu tnumu chápu tak, že odmocnitel je jako první a odmocněnec jako druhý, říká se \uv{třetí odmocnina z osmi}.

\begin{lispcode}{\texttt{tnum-root}}{Funkce pro odmocňování tnumu}
(\textcolor{funkcionalni}{defun} \textcolor{pojmenovan}{tnum-root} (tnum1 tnum2)
  (\textcolor{moje}{tnum-expt} tnum2 (\textcolor{moje}{/tnum} tnum1)))
\end{lispcode}
\end{remark}

Takto je tedy dokončena základní funkcionalita knihovny \texttt{tnums} a v další části se podíváme na její používání, perspektivu a doprogramujeme nějaké uživatelské funkce.
\mypart{Rozhraní}\label{cast:rozhrani}
V poslední části pojednáme o uživatelském pohledu na knihovnu \texttt{tnums} -- používání (převody, konstanty, operace, funkce), rychlost a uživatelskou rozšiřitelnost.
\section{Uživatelské funkce}
Při tvorbě knihovny pracující s tnumy jsem už některé funkce, které bych prohlásil za uživatelské napsal. Uživatelskou funkcí myslím takovou funkci, která nezná vnitřní implementaci tnumů (jako funkcí přesnosti) a volá jen funkce rozhraní zobrazené v Tabulce \ref{tab:funkce_rozhrani}. V této kapitole některé další uživatelské funkce dopíšeme, aby bylo vidět, jak se knihovna \texttt{tnums} používá. Poté, co si projdeme proces instalace se podíváme na jednoduché převody a vyčíslování konstant, poté přidáme operace a na závěr funkce. Ukážeme, že síla knihovny spočívá v jednoduchém vytváření nových tnumů a ve velmi silně oddělené vnitřní implementaci od vnějšího chování.

\subsection{Instalace}
Knihovna \texttt{tnums} je k dostání na githubu na odkaze \url{https://github.com/slavon00/tnums} nebo pro čtenáře tištěné verze na přiloženém fyzickém disku. Je to Lispová knihovna a od uživatele předpokládá základy práce s Lispem a REPLem.

Vše, co jsme doposud naprogramovali najdeme v souboru \texttt{src/tnums.lisp} a všechny funkce, které přidáme v této kapitole pak v \texttt{src/user-functions.lisp}. Testy, které budu ukazovat jsou v souboru \texttt{src/tests.lisp}. Vše je možné jednoduše načíst jen evaluací souboru \texttt{load.lisp}, jak ukazuje Obrázek \ref{obr:loading}. Pro kompletní obsah adresáře vizte přílohu \ref{pril:adresar}.

Knihovna se v budoucnosti může měnit, takže tyto informace mohou zastarat. Pro zpětnou kompatibilitu ale knihovna na githubu bude vždy obsahovat soubor \texttt{README.md} nebo ekvivalentní, aby mohla instalace proběhnout bez problémů.

\begin{myfigure}{H}
\caption{Načtení knihovny \texttt{tnum} do \texttt{SBCL}}
\includegraphics[width=\linewidth]{./graphics/loading.png}\label{obr:loading}
Aby se nemusely ručně načítat všechny soubory, lze knihovnu \texttt{tnums} též načíst jen evaluací souboru \texttt{load.lisp}. 
\end{myfigure}

\subsection{Převody a konstanty}
Téměř všechny naprogramované funkce berou jako vstup tnumy. To jsou abstraktní struktury, které interně reprezentujeme jako funkce malých čísel. Pokud chceme vytvořit tnum z nějakého čísla, které již máme nějak uložené v Lispu, slouží k tomu funkce \texttt{num-to-tnum}.

\begin{lisptest}{\texttt{num-to-tnum}}{Představení funkce pro převod numu na tnum}
* (num-to-tnum 42.123)
#<FUNCTION (LAMBDA (EPS) :IN NUM-TO-TNUM) {10020A056B}>
\end{lisptest}

Pro převod opačným směrem máme inverzní funkci \texttt{tnum-to-num}, která kro\-mě tnumu bere i druhý argument představující přesnost, se kterou chceme daný tnum vyčíslit.

\begin{lisptest}{\texttt{tnum-to-num}}{Představení funkce na převod tnumu na num}
* (tnum-to-num * 0.1)
34246/813
\end{lisptest}

Jak vidno, num vrácený funkcí \texttt{tnum-to-num} je racionální číslo.

\begin{lisptest}{Typ výstupu je číslo}{Ověření typu vraceného numu}
* (type-of *)
RATIO
* (float **)
42.123
\end{lisptest}

Protože knihovna vrací čísla jak je chápe Lisp, jsou výsledky vyčíslení plně kompatibilní s ostatními funkcemi pro čísla. Naprogramujme nyní funkci, která bude tnumy převádět na textové řetězce. Takto získáme dlouhé rozvoje v čitelné podobě, ne jen jako lidskému oku nic neříkající velké zlomky.

\begin{lispcode}{\texttt{tnum-to-string}}{Funkce na převod tnumu na textový řetězec}
(\textcolor{funkcionalni}{defun} \textcolor{pojmenovan}{tnum-to-string} (tnum count)
  (\textcolor{vedlejsi}{let} ((num (\textcolor{moje}{tnum-to-num} tnum (\textcolor{matematicke}{1+} count))) (output ""))
    (\textcolor{funkcionalni}{when} (\textcolor{matematicke}{<} num 0) (\textcolor{vedlejsi}{setf} output "-" num (\textcolor{matematicke}{-} num)))
    (\textcolor{matematicke}{multiple-value-bind} (digit rem)
        (\textcolor{matematicke}{floor} num)
      (\textcolor{vedlejsi}{setf} output (\textcolor{matematicke}{concatenate} \textquotesingle\textcolor{moje}{string} output 
                                (\textcolor{funkcionalni}{write-to-string} digit) ".")
            num (\textcolor{matematicke}{*} 10 rem)))
    (\textcolor{funkcionalni}{dotimes} (i count (\textcolor{matematicke}{concatenate} \textquotesingle\textcolor{moje}{string} output "..."))
      (\textcolor{matematicke}{multiple-value-bind} (digit rem)
          (\textcolor{matematicke}{floor} num)
        (\textcolor{vedlejsi}{setf} output (\textcolor{matematicke}{concatenate} \textquotesingle\textcolor{moje}{string} output 
                                  (\textcolor{funkcionalni}{write-to-string} digit))
              num (\textcolor{matematicke}{*} 10 rem))))))
\end{lispcode}
Funkce bere jako vstup tnum a přirozené číslo značící počet desetinných míst. Otestujeme ji na výpisu Ludolfova čísla.

\begin{lisptest}{\texttt{tnum-string} a \texttt{tnum-pi}}{Vyčíslení Ludolfova čísla na 50 desetinných míst}
* (tnum-to-string (tnum-pi) 50)
"3.14159265358979323846264338327950288419716939937510..."
\end{lisptest}

Vzhledem k rozšíření množiny přípustných hodnot funkce \texttt{tnum-to-num} je možné pohodlně přepínat mezi návratovou hodnotou jako číslem a textovým řetězcem. Ukážeme si to na vyčíslení Eulerova čísla.

\begin{lisptest}{\texttt{tnum-string} a \texttt{tnum-e}}{Vyčíslení Eulerova čísla na 20 desetinných míst a jeho vrácení jako čísla a jako stringu}
* (tnum-to-num (tnum-e) 20)
611070150698522592097/224800145555521536000
* (tnum-to-string (tnum-e) 20)
"2.71828182845904523536..."
\end{lisptest}

\subsection{Operace}
Operace tnumu je funkce $\bigtimes_{i=0}^{n-1}\mathfrak{T}\rightarrow\mathfrak{T}$, kde $n\in\mathbb{N}$ nazýváme aritou. Operace \texttt{tnum+} a \texttt{tnum*} mohou mít libovolný počet argumentů, jsou tedy $(n \in \mathbb{N})$-ární, operace \texttt{-tnum} a \texttt{/tnum} jsou striktně unární, operace \texttt{tnum-} a \texttt{tnum/} potřebují alespoň jeden argument, jsou $(n \in \mathbb{N}^+)$-ární a operace \texttt{tnum-expt} a \texttt{tnum-root} jsou binární.

V lispu jsou dvě hezké funkce na inkrementaci a dekrementaci čísla. To samé nyní přidáme pro tnumy.

\begin{lispcode}{\texttt{tnum-1+}}{Funkce pro inkrementaci tnumu o jedničku}
(\textcolor{funkcionalni}{defun} \textcolor{pojmenovan}{tnum-1-} (tnum)
  (\textcolor{moje}{tnum-} tnum (\textcolor{moje}{num-to-tnum} 1)))
\end{lispcode}

Funkce pro dekrementaci se také dá napsat pohodlně uživatelsky.

\begin{lispcode}{\texttt{tnum-1-}}{Funkce pro dekrementaci tnumu o jedničku}
(\textcolor{funkcionalni}{defun} \textcolor{pojmenovan}{tnum-1-} (tnum)
  (\textcolor{moje}{tnum-} tnum (\textcolor{moje}{num-to-tnum} 1)))
\end{lispcode}

Také by šla napsat nejpoužívanější odmocnina a sice druhá.

\begin{lispcode}{\texttt{tnum-sqrt}}{Funkce pro druhou odmocninu tnumu}
(\textcolor{funkcionalni}{defun} \textcolor{pojmenovan}{tnum-sqrt} (tnum)
  (\textcolor{moje}{tnum-root} (\textcolor{moje}{num-to-tnum} 2) tnum))
\end{lispcode}

Výše naprogramované použijeme k zavedení další konstanty, zlatého řezu.

\begin{definition}[Zlatý řez \cite{GR}]
Zlatý řez představuje kladné řešení rovnice $x^2-x-1=0$, je tedy roven hodnotě
\begin{equation}
\varphi=\frac{1+\sqrt{5}}{2}
\end{equation}
\end{definition}

Jedná se o další iracionální konstantu, tentokrát algebraickou, protože je řešením algebraické rovnice a její tnum je jen syntaktický přepis uvedené definice.

\begin{lispcode}{\texttt{tnum-phi}}{Funkce vracející tnum představující zlatý řez}
(\textcolor{funkcionalni}{defun} \textcolor{pojmenovan}{tnum-phi} ()
  (\textcolor{moje}{tnum/} (\textcolor{moje}{tnum-1+} (\textcolor{moje}{tnum-sqrt} (\textcolor{moje}{num-to-tnum} 5))) (\textcolor{moje}{num-to-tnum} 2)))
\end{lispcode}

A ještě test.

\begin{lisptest}{\texttt{tnum-phi}}{Představení funkce pro zlatý řez}
* (tnum-to-string (tnum-phi) 50)
"1.61803398874989484820458683436563811772030917980576..."
\end{lisptest}

\subsection{Funkce}
Funkce tnumů jsou všechny unární. Jedná se o přirozený logaritmus, goniometrické funkce a přirozenou exponenciálu. Omezení definičního oboru jsou stejná jako jsme zvyklí, naprogramované funkce tedy nejsou o nic \uv{slabší}.

Exponenciálu jsme využili už při psaní obecné mocniny. V podobném duchu nyní zavedeme obecný logaritmus. Vyjdeme z faktu, že lze převádět mezi různými základy.

\begin{fact}[Obecný logaritmus jako podíl přirozených  \cite{tabulky}]
Pro $a>1$ a $x\in\mathbb{R}^+$ platí
\begin{equation}
\log_a(x)=\frac{ln(x)}{ln(a)}
\end{equation}
\end{fact}

Nová funkce bude brát dva argumenty a proto nejde tak úplně o funkci tnumu, jak ji v této práci chápeme, ale spíše o matematickou operaci, nicméně na eleganci zápisu to nic neubírá.

\begin{lispcode}{\texttt{tnum-log}}{Funkce pro výpočet obecného logaritmu}
(\textcolor{funkcionalni}{defun} \textcolor{pojmenovan}{tnum-log} (tnum1 tnum2)
  (\textcolor{moje}{tnum/} (\textcolor{moje}{tnum-ln} tnum2) (\textcolor{moje}{tnum-ln} tnum1)))
\end{lispcode}

Uživatelská funkce potom podobně jako u odmocniny prohazuje argumenty, protože mluvíme vždy o nějakém logaritmu něčeho, například \uv{devítkový logaritmus dvou}. Ten je i předmětem následujícího testu.

\begin{lisptest}{\texttt{tnum-log}}{Vyčíslení devítkového logaritmu dvou}
* (tnum-to-string (tnum-log (num-to-tnum 9) (num-to-tnum 2)) 50)
"0.31546487678572871854976355717138042714979282006594..."
\end{lisptest}

Gomiometrické funkce jsou z velké části napsány též uživatelsky, takže by mělo být jasné, jak se s nimi z tohoto pohledu pracuje. Ukážu tedy jen vyčíslení, aby bylo vidět, že funkce opravdu fungují. Následuje výpočet sinu jedničky.

\begin{lisptest}{\texttt{tnum-sin}}{Představení funkce na výpočet sinu tnumu}
* (coerce (tnum-to-num (tnum-sin (num-to-tnum 1)) -20)
    'long-float)
0.8414709848078965d0
\end{lisptest}

Pokud má čtenář pocit, že takovéto číslo už někdy viděl, je tento pocit správný, protože přesně tímto způsobem jsem naplnil tabulku \ref{tab:sinus_input_output}.

\subsection{Rychlost}
Tabulka \ref{tab:rychlost} zobrazuje, jak dlouho běžně trvá vyhodnocení výrazů. Hodnoty budou vždy závislé na konkrétním stroji a jeho momentálním zatížení, obecnou představu o časové náročnosti výpočtů by ale měly poskytnout.

\begin{table}[H]
\begin{mdframed}[backgroundcolor=lightpink,innertopmargin=-2.5pt,innerbottommargin=2.5pt]
\centering
\caption{Doba výpočtů daných výrazů}
\label{tab:rychlost}
\begin{tabular}{| >{\columncolor[gray]{1}} l |>{\columncolor[gray]{1}}p{2.5cm}|}
\hline
\multicolumn{1}{|>{\columncolor[gray]{1}}c|}{Výraz} & Doba vyhodnocení~(s)\\\hline\hline
\texttt{((let ((tn}& \cellcolor[gray]{1} \\
\texttt{~~~~~~(tnum/ (tnum-pi) (tnum-e) (tnum-phi)))} & \multirow{-2}{*}{-}\\ \hline
\texttt{~~~(tnum-to-string tn 50)} & 1.367339 \\ \hline
\texttt{~~~(tnum-to-string (tnum-sin tn) 50)} & 28.180623 \\ \hline
\texttt{~~~(tnum-to-string (tnum-csc tn) 50)} & 3600.500439 \\ \hline
\texttt{~~~(tnum-to-string (tnum-ctan tn) 50))} & 25431.702944 \\ \hline
\end{tabular}

Tabulka v prvním sloupci zobrazuje výrazy, které byly vyhodnocovány a ve druhém čas, který toto vyhodnocení zabralo. Doba byla měřena makrem \texttt{time} a hodnoty jsou z řádku \uv{(\ldots) seconds of total run time}.
\end{mdframed}
\end{table}

\subsection{Vnější volání}
Pro přehlednost, jaké rozhraní knihovna \texttt{tnums} nabízí následuje souhrnná tabulka zobrazující všechny funkce určené k volání uživatelem. 

\begin{table}[H]
\begin{mdframed}[backgroundcolor=lightpink,innertopmargin=-2.5pt,innerbottommargin=2.5pt]
\centering
\caption{Funkce nabízené knihovnou \texttt{tnums}}
\label{tab:funkce_rozhrani}
\begin{tabular}{| >{\columncolor[gray]{1}} c |>{\columncolor[gray]{1}}c|>{\columncolor[gray]{1}}p{4.8cm}|}
\hline
Název & Argumenty & Význam\\ \hline \hline
\texttt{tnum-to-num} & \texttt{tnum:tnum}, \texttt{eps:num} & převod \texttt{tnum}u na číslo s~přesností \texttt{eps}\\
\texttt{tnum-to-string} & \texttt{tnum:tnum}, \texttt{count:num} & převod \texttt{tnum}u na textový řetězec o \texttt{count} desetinných místech\\\hline
\texttt{num-to-tnum}&\texttt{num:num}&převod \texttt{num}u na tnum\\
\texttt{tnum-pi}&$\emptyset$&Ludolfovo číslo jako tnum\\
\texttt{tnum-e}&$\emptyset$&Eulerovo číslo jako tnum\\
\texttt{tnum-phi}&$\emptyset$&Zlatý řez jako tnum\\\hline
\texttt{-tnum}&\texttt{tnum:tnum}&$-\texttt{tnum}$\\
\texttt{tnum+}&0+ \texttt{tnum}ů&součet tnumů\\
\texttt{tnum-}&1+ \texttt{tnum}ů&rozdíl tnumů\\
\texttt{/tnum}&\texttt{tnum:tnum}&$1/\texttt{tnum}$\\
\texttt{tnum*}&0+ \texttt{tnum}ů&součin tnumů\\
\texttt{tnum/}&1+ \texttt{tnum}ů&podíl tnumů\\
\texttt{tnum-expt}&\texttt{arg1:tnum}, \texttt{arg2:tnum}&$\texttt{arg1}^{\texttt{arg2}}$\\
\texttt{tnum-sqrt}&\texttt{arg1:tnum}, \texttt{arg2:tnum}&$\sqrt[\texttt{arg1}]{\texttt{arg2}}$\\
\texttt{tnum-log}&\texttt{arg1:tnum}, \texttt{arg2:tnum}&$\mathrm{log}_{\texttt{arg1}}(\texttt{arg2})$\\
\texttt{tnum-1+}&\texttt{tnum:tnum}&$\texttt{tnum} + 1$\\
\texttt{tnum-1-}&\texttt{tnum:tnum}&$\texttt{tnum} - 1$\\\hline
\texttt{tnum-exp}&\texttt{tnum:tnum}&přirozená mnocnina \texttt{tnum}u\\
\texttt{tnum-ln}&\texttt{tnum:tnum}&přirozený logaritmus \texttt{tnum}u\\
\texttt{tnum-sqrt}&\texttt{tnum:tnum}&druhá odmocnina \texttt{tnum}u\\
\texttt{tnum-sin}&\texttt{tnum:tnum}&sinus \texttt{tnum}u\\
\texttt{tnum-cos}&\texttt{tnum:tnum}&kosinus \texttt{tnum}u\\
\texttt{tnum-tan}&\texttt{tnum:tnum}&tangens \texttt{tnum}u\\
\texttt{tnum-csc}&\texttt{tnum:tnum}&kotangens \texttt{tnum}u\\
\texttt{tnum-sec}&\texttt{tnum:tnum}&sekans \texttt{tnum}u\\
\texttt{tnum-ctan}&\texttt{tnum:tnum}&kosekans \texttt{tnum}u\\
\hline\end{tabular}

Tabulka v prvním sloupci zobrazuje funkční symbol, v posledním význam funkce aplikované na argumenty z prostředního sloupce. Čtyři části rozdělené horizontálními čarami jsou po řadě funkce pro převody, konstanty, operace a matematické funkce. Součástí jsou i uživatelské funkce.
\end{mdframed}
\end{table}
\clearpage
\section{Diskuze}
\input{perspektiva.tex}
\begin{kiconclusions}
Popsal jsem, jak jsou vytvořena přirozená čísla pomocí teorie množin, také jak na tomto základě vznikají další číselné obory. Dále jsem popsal, jak se s \textbf{čísly} pracuje v paměti v počítače.

Také bylo zmíněno, že číselná osa je tvořena \textbf{reálnými čísly} a pokud použijeme více os, dostáváme strukturovaná čísla. Zamysleli jsme se, jestli jsou všechna reálná čísla rekurzivní a bohužel jsme dostali negativní odpověď.

Představil jsem, jak vypadají \textbf{výpočty s reálnými čísly}. Kromě matematických operací to byly matematické funkce. Zjistili jsme, že všechny tyto výpočty, včetně samotných reálných konstant lze reprezentovat jako funkce.

V textu jsem se věnoval i produktu celého tohoto snažení a sice programování Lispovské knihovny \texttt{tnums} implementující \textbf{přesné výpočty s reálnými čísly}. Také jsem přinesl několik příkladů, jak uživatelsky funkcionalitu rozšiřovat.
\end{kiconclusions}
\begin{kiconclusions}[english]
I have described how natural \textbf{numbers} are created using set theory, as well as how other number systems are created on this basis. I have also described how it works with numbers in a computer memory.

It was also mentioned that the number line is made up of \textbf{real numbers} and if we use more axes, we get structured numbers. We wondered if all of the real numbers are recursive and unfortunately we got a negative answer.

I have presented what \textbf{computation of real numbers} looks like. In addition to mathematical operations, there were mathematical functions. We have found out that all these calculations, including the real constants themselves, can be represented as functions.

In the text, I also focused on the product of all this effort, namely programming Lisp \texttt{tnums} library implementing \textbf{precise computation of real numbers}. I have also come up with some examples of how user can extend the functionality.
\end{kiconclusions}
\clearpage
\printbibliography[heading=bibintoc, title={Seznam literatury}]
\appendix
\section{Obsah přiloženého CD/DVD}\label{pril:adresar}
Na samotném konci textu práce je uveden stručný popis obsahu
přiloženého CD/DVD, tj.~jeho závazné adresářové struktury, důležitých
souborů apod.

\begin{description}
\item[\texttt{doc/}] \hfill \\
Adresář se soubory
\begin{itemize}
\item{\texttt{OndrejSlavikBP.pdf} -- tento text ve formátu PDF a}
\item{\texttt{bak/} -- adresář se všemi soubory pro vysázení tohoto textu, stačí dvakrát přeložit \texttt{PDFLaTex}em.}
\end{itemize}

\item[\texttt{load.lisp}] \hfill \\
Přeložením tohoto souboru ve vašem oblíbeném interpretu/kompilátoru Lispu získáte plnou funkcionalitu knihovny \texttt{tnums}, kterou jsme právě doprogramovali. Načítá soubory \texttt{src/tnums.lisp} a \texttt{src/user-functions.lisp}.

\item[\texttt{src/}] \hfill \\
Adresář se soubory
\begin{itemize}
\item{\texttt{tnums.lisp} -- základ knihovny z části 2 této práce,}
\item{\texttt{user-function.lisp} -- rozšíření knihovny o vědomě uživatelské funkce ze šesté kapitoly a}
\item{\texttt{tests.lisp} -- zakomentované výrazy, které zde byly popsány jako Lispový test i s výsledky, na které se vyhodnotí.}
\end{itemize}

\item[\texttt{README.md}] \hfill \\
Soubor představující knihovnu \texttt{tnums} a obsahující i několik příkladů výpočtů, které podporuje. Součástí je i kapitola o načtení knihovny evaluací souboru \texttt{load.lisp}, nejedná se tedy o instalaci v pravém slova smyslu.

\item[\texttt{install/}] \hfill \\
Adresář se soubory
\begin{itemize}
\item{\texttt{sbcl-2.1.6-source.tar.bz2} -- intalátor SBCL, konzolového kompilátoru ANSI Common Lispu,}
\item{\texttt{code-1.57.1-1623937013-amd64.deb} -- instalátor VS code, rozšiřitelného textového editoru a}
\item{\texttt{2gua.rainbow-brackets-0.0.6.vsix} -- instalátor rozšíření Rainbow Brackets pro přehledné obarvování závorek.}
\end{itemize}

\item[\texttt{LISENCE}] \hfill \\
Licenší soubor -- knihovna je publikována pod GNU GPLv3.
\end{description}

\end{document}