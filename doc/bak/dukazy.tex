Zde uvedu některé důkazy, které jsou mimo těžiště práce a pro pochopení práce s tnumy nejsou zapotřebí. Nějaké zájemce by ale mohlo zajímat, jak se některé důkazy vedou a ikdyž se nehodí je napsat přímo v textu práce, zde mohou být.
\subsection{Fakt \ref{vet:o_castecnem_souctu_geometricke_rady} -- O částečném součtu geometrické řady}\label{duk:o_castecnem_souctu_geometricke_rady}
\begin{proof}
Částečný součet je z definice $s^a_n=a+aq+\ldots+aq^n$. Užitím vztahu
\begin{equation}
(1-q)(1+q+q^2+\ldots+q^n)=(1-q)^{n+1}
\end{equation}
dostaneme
\begin{equation}
s^a_n=a(1+q+q^2+\ldots+q^n)=a\frac{1-q^{n+1}}{1-q}.
\end{equation}
\end{proof}

\subsection{Fakt \ref{vet:o_geometricke_rade} -- O geometrické řadě}\label{duk:o_geometricke_rade}
\begin{proof}
Po užití limity na obě strany ze lemmatu \ref{vet:o_castecnem_souctu_geometricke_rady} dostáváme 
\begin{equation}
\sum_{i\in\mathbb{N}}aq^i=\lim_{n\to\infty}s^a_n=\lim_{n\to\infty}a\frac{1-q^{n+1}}{1-q}=\frac{a}{1-q}.
\end{equation}
\end{proof}

\subsection{Fakt \ref{vet:o_zbytku_geometricke_rady} -- O zbytku geometrické řady}\label{duk:o_zbytku_geometricke_rady}	
\begin{proof}
\begin{equation}
R_n = \sum_{i\in\mathbb{N}}aq^i -s^a_n =\frac{a}{1-q} - a\frac{1-q^{n+1}}{1-q} = \frac{a-a+aq^{n+1}}{1-q} = \frac{aq^{n+1}}{1-q}
\end{equation}
\end{proof}

\subsection{Fakt \ref{vet:exp_jako_rada} -- O exponenciále jako Maclaurinově řadě}\label{duk:exp_jako_rada}
\begin{proof}
\begin{equation}
\begin{split}
e^x = \sum_{i\in\mathbb{N}}\frac{exp^{(i)}(0)}{i!}x^i = \left[\begin{matrix}\frac{d}{dx} exp(x) = exp(x) \\ \end{matrix}\right] = \frac{exp(0)}{1} + \frac{exp(0)}{1}x +\\ +\frac{exp(0)}{2}x^2 + \ldots= \left[\begin{matrix} exp(0)=1 \end{matrix}\right] = \frac{1}{1} + \frac{x}{1} + \frac{x^2}{2} + \ldots=\sum_{i\in\mathbb{N}}\frac{x^i}{i!}
\end{split}
\end{equation}
\end{proof}

\subsection{Fakt \ref{vet:sin_jako_rada} -- O sinu jako Maclaurinově řadě}\label{duk:sin_jako_rada}
\begin{proof}
\begin{equation}
\begin{split}
sin(x)=\sum_{i\in\mathbb{N}}\frac{sin^{(i)}(0)}{i!}x^i=\left[\begin{matrix}\frac{d}{dx} sin(x)=cos(x) \\ \frac{d}{dx} cos(x)=-sin(x) \end{matrix} \right] =\\= \frac{sin(0)}{1} + \frac{cos(0)}{1}x-\frac{sin(0)}{2}x^2-\frac{cos(0)}{6}x^3+\frac{sin(0)}{24}x^4+\frac{cos(0)}{120}x^5\ldots=\\=\left[\begin{matrix} sin(0)=0\\ cos(0)=1 \end{matrix}\right]=\frac{x}{1}-\frac{x^3}{6}+\frac{x^5}{120}+\ldots=\sum_{i\in\mathbb{N}}(-1)^i\frac{x^{2i+1}}{(2i+1)!}
\end{split}
\end{equation}
\end{proof}

\subsection{Fakt \ref{vet:cos_jako_rada} -- O kosinu jako Maclaurinově řadě}\label{duk:cos_jako_rada}
\begin{proof}
\begin{equation}
\begin{split}
cos(x)=\sum_{i\in\mathbb{N}}\frac{cos^{(i)}(0)}{i!}x^i=\left[\begin{matrix}\frac{d}{dx} cos(x)=-sin(x) \\ \frac{d}{dx} sin(x)=cos(x) \end{matrix} \right] = \\ =\frac{cos(0)}{1} - \frac{sin(0)}{1}x - \frac{cos(0)}{2}x^2+\frac{sin(0)}{6}x^3+\frac{cos(0)}{24}x^4\ldots=\\=\left[\begin{matrix} cos(0)=1 \\ sin(0)=0 \end{matrix}\right]=\frac{1}{1}-\frac{x^2}{2}+\frac{x^4}{24}+\ldots=\sum_{i\in\mathbb{N}}(-1)^i\frac{x^{2i}}{(2i)!}
\end{split}
\end{equation}
\end{proof}
